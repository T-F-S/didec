% \LaTeX-Main\
% !TeX encoding=UTF-8
% !TeX spellcheck=en_US
%%
%% The LaTeX package didec - version 1.0.0 (2024/02/28)
%% didec.tex: Manual
%%
%% -------------------------------------------------------------------------------------------
%% Copyright (c) 2024-2024 by Prof. Dr. Dr. Thomas F. Sturm <thomas dot sturm at unibw dot de>
%% -------------------------------------------------------------------------------------------
%%
%% This work may be distributed and/or modified under the
%% conditions of the LaTeX Project Public License, either version 1.3
%% of this license or (at your option) any later version.
%% The latest version of this license is in
%%   http://www.latex-project.org/lppl.txt
%% and version 1.3 or later is part of all distributions of LaTeX
%% version 2005/12/01 or later.
%%
%% This work has the LPPL maintenance status `author-maintained'.
%%
%% This work consists of all files listed in README
%%
\documentclass[a4paper,11pt]{ltxdoc}
\usepackage{didec.doc}

\hypersetup{
  pdftitle={Manual for the didec package},
  pdfauthor={Thomas F. Sturm},
  pdfsubject={Calculations with two decimal places},
  pdfkeywords={fixed-point arithmetic, two decimal places, calculations, cash book, bookkeeping, accountancy}
}

\makeindex

\pdfsuppresswarningpagegroup=1

\DeclareUnicodeCharacter{20AC}{\euro{}}
\didecsetup{
    decimal-separator  = {,},
    grouping-separator = {.},
    currency           = {}{\;€},
}

\didecnew{A}
\didecnew{B}
\didecnew{C}


%%%%%%%%%%%%%%%%%%%%%%%%%%%%%%%%%%%%%%%%%%%%%%%%%
\begin{document}

\begin{center}
\begin{tcolorbox}[enhanced,hbox,tikznode,left=8mm,right=8mm,boxrule=0.4pt,
  colback=white,colframe=Blue_Gray,
  drop lifted shadow=Blue_Gray!50,arc is angular,
  before=\par\vspace*{5mm},after=\par\bigskip]
{\bfseries\LARGE The didec package}\\[3mm]
{\large Manual for version \version\ (\datum)}
\end{tcolorbox}
{\large Thomas F.~Sturm%
  \footnote{Prof.~Dr.~Dr.~Thomas F.~Sturm, Institut f\"{u}r Mathematik und Informatik,
    University of the Bundeswehr Munich, D-85577 Neubiberg, Germany;
     email: \href{mailto:thomas.sturm@unibw.de}{thomas.sturm@unibw.de}}\par\medskip
\normalsize%
\url{https://www.ctan.org/pkg/didec}\par
\url{https://github.com/T-F-S/didec}
}
\end{center}
\bigskip
\begin{absquote}
  \begin{center}\bfseries Abstract\end{center}
  The \texttt{didec} package supports fixed-point arithmetic with two
  decimal places (\emph{di-decimal}) which is typical for financial transactions
  in many currencies. The intended use case is (personal) bookkeeping.
\end{absquote}

\tableofcontents

%%%%%%%%%%%%%%%%%%%%%%%%%%%%%%%%%%%%%%%%%%%%%%%%%%%%%%%%%%%%%%%%%%%%%%%%%%%%%%%%
\clearpage
\section{Quick start}%
For the impatient: this package provides fixed-point arithmetic with
two decimal places. You may use it for any purpose where exactly two decimal
places are needed or suffice, but the main application case is (personal) bookkeeping.

Say, John wants to keep track about his money. With
\begin{dispListing}
\didecnew{John}
\end{dispListing}
\didecnew{John}
a so-called \emph{didec variable} is created to store currency values.

Now, lets fill in some money:
\begin{dispListing}
\didecset{John}{1000}
\end{dispListing}
\didecset{John}{1000}

How much money has John now?
\begin{dispExample}
\didecuse{John}
\end{dispExample}

Obviously, John has Euros and the amount is displayed in a German style manor.
Of course, this can be adapted to your liking, see \refCom{didecsetup}.
But now, let us spend some money.
\begin{dispExample}
\didecsub{John}{19.75}
\didecuse{John}
\end{dispExample}

Preferably, the package functions are used inside convenient user commands.
You can choose between \LaTeX2e and \LaTeX3 programming layer functions.

For this quick start, we make some convenience commands for John.
|\transaction| shall be one purchase made in cash and |\transaction*| one made
by credit card.
\begin{dispExample*}{breakable}
%\usepackage{booktabs}
\didecsetup{english,
  currency          = {\pounds}{},
  currency-negative = {-\pounds}{},
}

%\didecnew{John}
\didecnew{cash}
\didecnew{credit}
\didecnew{transaction}

% Let's keep record in a table
\NewDocumentEnvironment{householdbook}{}{%
    \begin{center}
    \begin{tabular}{lp{9cm}rr}\toprule%
    & Transaction & Expenses & Budget\\\midrule
  }{%
    \midrule%
    \multicolumn{3}{r}{Cash} & \dideccoluse{cash}\\%
    \multicolumn{3}{r}{Credit card} & \dideccoluse{credit}\\\bottomrule%
    \end{tabular}%
    \end{center}
  }

% One transaction
\NewDocumentCommand \transaction{ s m m }
  {
    \didecset{transaction}{#3}%
    \didecsub{John}{transaction}%
    \IfBooleanTF {#1}%
      {\didecsub{credit}{transaction}CC}%
      {\didecsub{cash}{transaction}}%
    & #2
    & \dideccolinvuse{transaction}
    & \dideccoluse{John}\\%
  }

% Ready to start our tiny accountancy

\didecset{John}{1000}        % John's money
\didecsetequal{cash}{John}   % in cash
\didecset{credit}{0}         % blank credit card

\begin{householdbook}
  \transaction{Coffee break with snack}{19.75}
  \transaction*{Refuel}{62.87}
  \transaction{Gift from Aunt Mary for helping her}{-30}
  \transaction{Parking meter}{4.50}
  \transaction*{Shopping for weekend}{147.23}
  \transaction*{Fancy thing on the Internet}{270}
  \transaction*{Cinema}{17.70}
\end{householdbook}

\end{dispExample*}




%%%%%%%%%%%%%%%%%%%%%%%%%%%%%%%%%%%%%%%%%%%%%%%%%%%%%%%%%%%%%%%%%%%%%%%%%%%%%%%%
\clearpage
\section{Didec kernels and didec expressions}%

All calculations are done on cent basis as integer operations, but all
displayed figures have two decimal places which give the name for the package
(\emph{di}-\emph{dec}imal).

The package provides two numerical kernels which can be selected mutually
by package options \refKey{didec/int} or \refKey{didec/fp}.
The \refKey{didec/int} kernel is faster, but provides a smaller number range,
while the \refKey{didec/int} kernel is slower with a larger number range.
For ordinary people like you and me, the \refKey{didec/int} kernel will suffice
to do all personal financial calculations.
Upgrading later from \refKey{didec/int} to \refKey{didec/fp} is a matter of
just switching the package option setting.

%-------------------------------------------------------------------------------
\begin{docPackageOptions}{{
    doc name = int,
    doc description = {no value, initially set},
  }}
Selects the \meta{int} (integer) based numerical kernel with up to
9.33\footnote{joke for the mathematicians} significant figures and fast computation.
The number range for valid figures $n$ is:
\begin{center}
$ -21\;474\;836.47 \le n \le 21\;474\;836.47 $
\end{center}
\begin{dispListing}
\usepackage[int]{didec}
\end{dispListing}
Using figures outside the valid range will result in \LaTeX\ errors complaining
about too large numbers.
\end{docPackageOptions}
%-------------------------------------------------------------------------------


%-------------------------------------------------------------------------------
\begin{docPackageOptions}{{
    doc name = fp,
    doc description = {no value, initially unset},
  }}
Selects the \meta{fp} (floating-point) based numerical kernel with up to
16 significant figures and somewhat slower computation.
The number range for valid figures $n$ is:
\begin{center}
$ -99\;999\;999\;999\;999.99 \le n \le 99\;999\;999\;999\;999.99 $
\end{center}
\begin{dispListing}
\usepackage[fp]{didec}
\end{dispListing}
Using figures outside the valid range will result in \emph{silent calculation errors},
because \LaTeX3 \meta{fp} can use much larger numbers but is restricted to 16
significant figures.
\end{docPackageOptions}
%-------------------------------------------------------------------------------

\medskip

In the following, a \meta{didec expr} (didec expression) denotes one of
the following:
\begin{itemize}
\item a number in floating-point notation, e.g. |123.45|
\item a number in floating-comma notation, e.g. |123,45|
\item a \meta{didec var} (didec variable), e.g. |expenses|.
\end{itemize}
Note that the notation \meta{didec expr} is inspired by \meta{int expr} and \meta{fp expr}
from \LaTeX3 but is in comparison very restricted and allows only the three choices above.
A \meta{didec expr} is always expanded and spaces are trimmed.

\medskip

Many provided commands or functions of the package come in three flavors, for
example:
\begin{itemize}
\item \refCom{didecadd}: This is a user command where arguments are space trimmed
  and some check on variable existence is done.
  Not existing variables are reported by speaking error messages (not in all cases!).
\item \refCom{didec_gadd_check:nn}: This a programming layer function
  with no space trimming for arguments, but some check on variable existence is done.
  Not existing variables are reported by speaking error messages (not in all cases!).
\item \refCom{didec_gadd:nn}: This a programming layer function
  with no space trimming for arguments and no check on variable existence.
  Not existing variables give strange errors.
  This is the fasted function and base of the others above.
\end{itemize}


\clearpage
The following tables compare the computation time for selected functions of
the package for the two numerical kernels. Time values will differ on other
computers and also depend on selected values for the examples calculations.
Nevertheless, you get an impression of the differences.


%-------------------------------------------------------------------------------
\begin{tcbverbatimwrite}{\jobname.temp}
\begin{tcblisting}{
  blankest,
  pdf comment,
  comment style={raster columns=1,blankest,enhanced/.style=},
  comment only,
  freeze pdf,
  compilable listing,
  run pdflatex
}
\IfFileExists{didec.doc.cfg}{% !TeX root = didec.tex
% !TeX encoding=UTF-8
%% The LaTeX package didec - version 1.0.0 (2024/02/28)
%% didec.doc.sty: style file for the documentation
%%
%% -------------------------------------------------------------------------------------------
%% Copyright (c) 2024-2024 by Prof. Dr. Dr. Thomas F. Sturm <thomas dot sturm at unibw dot de>
%% -------------------------------------------------------------------------------------------
%%
%% This work may be distributed and/or modified under the
%% conditions of the LaTeX Project Public License, either version 1.3
%% of this license or (at your option) any later version.
%% The latest version of this license is in
%%   http://www.latex-project.org/lppl.txt
%% and version 1.3 or later is part of all distributions of LaTeX
%% version 2005/12/01 or later.
%%
%% This work has the LPPL maintenance status `author-maintained'.
%%
%% This work consists of all files listed in README.md
%%
% include file (style) of didec.tex (manual of the LaTeX package didec)

\def\version{1.0.0}%
\def\datum{2024/02/28}%

\IfFileExists{didec.doc.cfg}{% !TeX root = didec.tex
% !TeX encoding=UTF-8
%% The LaTeX package didec - version 1.0.0 (2024/02/28)
%% didec.doc.sty: style file for the documentation
%%
%% -------------------------------------------------------------------------------------------
%% Copyright (c) 2024-2024 by Prof. Dr. Dr. Thomas F. Sturm <thomas dot sturm at unibw dot de>
%% -------------------------------------------------------------------------------------------
%%
%% This work may be distributed and/or modified under the
%% conditions of the LaTeX Project Public License, either version 1.3
%% of this license or (at your option) any later version.
%% The latest version of this license is in
%%   http://www.latex-project.org/lppl.txt
%% and version 1.3 or later is part of all distributions of LaTeX
%% version 2005/12/01 or later.
%%
%% This work has the LPPL maintenance status `author-maintained'.
%%
%% This work consists of all files listed in README.md
%%
% include file (style) of didec.tex (manual of the LaTeX package didec)

\def\version{1.0.0}%
\def\datum{2024/02/28}%

\IfFileExists{didec.doc.cfg}{% !TeX root = didec.tex
% !TeX encoding=UTF-8
%% The LaTeX package didec - version 1.0.0 (2024/02/28)
%% didec.doc.sty: style file for the documentation
%%
%% -------------------------------------------------------------------------------------------
%% Copyright (c) 2024-2024 by Prof. Dr. Dr. Thomas F. Sturm <thomas dot sturm at unibw dot de>
%% -------------------------------------------------------------------------------------------
%%
%% This work may be distributed and/or modified under the
%% conditions of the LaTeX Project Public License, either version 1.3
%% of this license or (at your option) any later version.
%% The latest version of this license is in
%%   http://www.latex-project.org/lppl.txt
%% and version 1.3 or later is part of all distributions of LaTeX
%% version 2005/12/01 or later.
%%
%% This work has the LPPL maintenance status `author-maintained'.
%%
%% This work consists of all files listed in README.md
%%
% include file (style) of didec.tex (manual of the LaTeX package didec)

\def\version{1.0.0}%
\def\datum{2024/02/28}%

\IfFileExists{didec.doc.cfg}{\input{didec.doc.cfg}}{}

\renewcommand*\l@subsection{\@dottedtocline{2}{1.5em}{2.7em}}
\renewcommand*\l@subsubsection{\@dottedtocline{3}{4.2em}{3.2em}}

\RequirePackage[T1]{fontenc}
\RequirePackage[english]{babel}
\RequirePackage{lmodern,parskip,array}
\RequirePackage[svgnames,table]{xcolor}
\RequirePackage{tikz}
\RequirePackage{varioref}
\RequirePackage[makeindex]{imakeidx}
\RequirePackage[bookmarks,raiselinks,pageanchor,hyperindex,colorlinks]{hyperref}
\RequirePackage{cleveref}
\RequirePackage{varwidth,incgraph}
\RequirePackage[official]{eurosym}

\RequirePackage[a4paper,left=2.5cm,right=2.5cm,top=1.5cm,bottom=1.5cm,
    marginparsep=3mm,marginparwidth=18mm,
    headheight=0mm,headsep=0cm,
    footskip=1.5cm,includeheadfoot%,showframe
    ]{geometry}
\RequirePackage{fancyhdr}
\fancyhf{}
\fancyfoot[C]{\thepage}%
\renewcommand{\headrulewidth}{0pt}
\renewcommand{\footrulewidth}{0pt}
\pagestyle{fancy}
\tolerance=2000%
\setlength{\emergencystretch}{20pt}%

\RequirePackage{array,tabularx,booktabs}
\RequirePackage{lipsum}
\RequirePackage[extendedchars,encoding,filenameencoding=utf-8]{grffile}
\RequirePackage{siunitx,xfp}

\RequirePackage{tcolorbox}
\tcbuselibrary{skins,minted,breakable,documentation,raster}

\ifdefined\didecpackageprefix
  \RequirePackage{\didecpackageprefix didec}
\else
  \RequirePackage{didec}
\fi

\definecolor{Green_Dark}{rgb}{0.078431,0.407843,0.176471}
\definecolor{Blue_Dark}{rgb}{0.090196,0.211765,0.364706}
\definecolor{Blue_Bright}{rgb}{0.858824,0.898039,0.945098}

\colorlet{Blue_Gray}{blue!50!gray}

\tcbset{skin=enhanced,
  doc head={colback=yellow!10!white,interior style=fill},
  doc head key={colback=magenta!5!white,interior style=fill},
  color key=DarkViolet,
  color value=Teal,
  color color=Teal,
  color counter=Orange!85!black,
  color length=Orange!85!black,
  index colorize,
  index annotate,
  beforeafter example/.style={
    before skip=4pt plus 2pt minus 1pt,
    after skip=8pt plus 4pt minus 2pt
  },
  docexample/.style={bicolor,
    beforeafter example,
    arc=0.66mm,
    boxrule=0.33mm,
    fonttitle=\bfseries,
    fontlower=\footnotesize,
    colframe=Blue_Gray,
    colback=Blue_Gray!5!white,
    colbacklower=white,
    drop small lifted shadow,
    listing engine=minted,
    documentation minted options={tabsize=2,fontsize=\small,breaklines,autogobble},
    documentation minted style=colorful,
    },
}
\urlstyle{sf}

\newtcblisting{fullexample}[1]{docexample,minted style=colorful,
  listing and comment,pdf comment,freeze pdf,compilable listing,#1}

\DeclareTotalTCBox{\myverb}{ O{} v }{tile,fontupper=\ttfamily,nobeforeafter,
  tcbox raise base,boxsep=0.5mm,top=0pt,bottom=0pt,left=0pt,right=0pt,boxrule=0.3mm,
  colback=yellow!10,
  borderline horizontal={0.3mm}{0pt}{red!50},
  #1}{#2}

\renewcommand*{\tcbdocnew}[1]{\textcolor{green!50!black}{\sffamily\bfseries N} #1}
\renewcommand*{\tcbdocupdated}[1]{\textcolor{blue!75!black}{\sffamily\bfseries U} #1}

\newcommand{\formatPackageOption}[1]{package option}
\newcommand{\formatPackageOptions}[1]{Package options}

\tcbmakedocSubKey[doc key prefix=]{docDidecKey}{didec}
\tcbmakedocSubKeys[doc key prefix=]{docDidecKeys}{didec}
\tcbmakedocSubKey[doc key prefix=,
    index key formatter=\formatPackageOption,
    index keys formatter=\formatPackageOptions
]{docPackageOption}{didec}
\tcbmakedocSubKeys[doc key prefix=,
    index key formatter=\formatPackageOption,
    index keys formatter=\formatPackageOptions
]{docPackageOptions}{didec}


\tcbset{
  deprecated*/.style={%
    doc head={colback=black!1!white,interior style=fill},
    doc head key={colback=black!3!white,interior style=fill},
    color command={black!80!white},
    color option={black!60!white},
  },
  deprecated/.style={%
    deprecated*,
    before doc body command={\textcolor{black!80!white}{\textsc{Deprecated! }}}
  },
}

\ExplSyntaxOn
\NewDocumentCommand \docColorExample { m }
  {
    \color_group_begin:
    \color_select:n { #1 }
    \rule[-1mm]{5mm}{4mm}~\docColor{#1}
    \color_group_end:
  }
\ExplSyntaxOff
}{}

\renewcommand*\l@subsection{\@dottedtocline{2}{1.5em}{2.7em}}
\renewcommand*\l@subsubsection{\@dottedtocline{3}{4.2em}{3.2em}}

\RequirePackage[T1]{fontenc}
\RequirePackage[english]{babel}
\RequirePackage{lmodern,parskip,array}
\RequirePackage[svgnames,table]{xcolor}
\RequirePackage{tikz}
\RequirePackage{varioref}
\RequirePackage[makeindex]{imakeidx}
\RequirePackage[bookmarks,raiselinks,pageanchor,hyperindex,colorlinks]{hyperref}
\RequirePackage{cleveref}
\RequirePackage{varwidth,incgraph}
\RequirePackage[official]{eurosym}

\RequirePackage[a4paper,left=2.5cm,right=2.5cm,top=1.5cm,bottom=1.5cm,
    marginparsep=3mm,marginparwidth=18mm,
    headheight=0mm,headsep=0cm,
    footskip=1.5cm,includeheadfoot%,showframe
    ]{geometry}
\RequirePackage{fancyhdr}
\fancyhf{}
\fancyfoot[C]{\thepage}%
\renewcommand{\headrulewidth}{0pt}
\renewcommand{\footrulewidth}{0pt}
\pagestyle{fancy}
\tolerance=2000%
\setlength{\emergencystretch}{20pt}%

\RequirePackage{array,tabularx,booktabs}
\RequirePackage{lipsum}
\RequirePackage[extendedchars,encoding,filenameencoding=utf-8]{grffile}
\RequirePackage{siunitx,xfp}

\RequirePackage{tcolorbox}
\tcbuselibrary{skins,minted,breakable,documentation,raster}

\ifdefined\didecpackageprefix
  \RequirePackage{\didecpackageprefix didec}
\else
  \RequirePackage{didec}
\fi

\definecolor{Green_Dark}{rgb}{0.078431,0.407843,0.176471}
\definecolor{Blue_Dark}{rgb}{0.090196,0.211765,0.364706}
\definecolor{Blue_Bright}{rgb}{0.858824,0.898039,0.945098}

\colorlet{Blue_Gray}{blue!50!gray}

\tcbset{skin=enhanced,
  doc head={colback=yellow!10!white,interior style=fill},
  doc head key={colback=magenta!5!white,interior style=fill},
  color key=DarkViolet,
  color value=Teal,
  color color=Teal,
  color counter=Orange!85!black,
  color length=Orange!85!black,
  index colorize,
  index annotate,
  beforeafter example/.style={
    before skip=4pt plus 2pt minus 1pt,
    after skip=8pt plus 4pt minus 2pt
  },
  docexample/.style={bicolor,
    beforeafter example,
    arc=0.66mm,
    boxrule=0.33mm,
    fonttitle=\bfseries,
    fontlower=\footnotesize,
    colframe=Blue_Gray,
    colback=Blue_Gray!5!white,
    colbacklower=white,
    drop small lifted shadow,
    listing engine=minted,
    documentation minted options={tabsize=2,fontsize=\small,breaklines,autogobble},
    documentation minted style=colorful,
    },
}
\urlstyle{sf}

\newtcblisting{fullexample}[1]{docexample,minted style=colorful,
  listing and comment,pdf comment,freeze pdf,compilable listing,#1}

\DeclareTotalTCBox{\myverb}{ O{} v }{tile,fontupper=\ttfamily,nobeforeafter,
  tcbox raise base,boxsep=0.5mm,top=0pt,bottom=0pt,left=0pt,right=0pt,boxrule=0.3mm,
  colback=yellow!10,
  borderline horizontal={0.3mm}{0pt}{red!50},
  #1}{#2}

\renewcommand*{\tcbdocnew}[1]{\textcolor{green!50!black}{\sffamily\bfseries N} #1}
\renewcommand*{\tcbdocupdated}[1]{\textcolor{blue!75!black}{\sffamily\bfseries U} #1}

\newcommand{\formatPackageOption}[1]{package option}
\newcommand{\formatPackageOptions}[1]{Package options}

\tcbmakedocSubKey[doc key prefix=]{docDidecKey}{didec}
\tcbmakedocSubKeys[doc key prefix=]{docDidecKeys}{didec}
\tcbmakedocSubKey[doc key prefix=,
    index key formatter=\formatPackageOption,
    index keys formatter=\formatPackageOptions
]{docPackageOption}{didec}
\tcbmakedocSubKeys[doc key prefix=,
    index key formatter=\formatPackageOption,
    index keys formatter=\formatPackageOptions
]{docPackageOptions}{didec}


\tcbset{
  deprecated*/.style={%
    doc head={colback=black!1!white,interior style=fill},
    doc head key={colback=black!3!white,interior style=fill},
    color command={black!80!white},
    color option={black!60!white},
  },
  deprecated/.style={%
    deprecated*,
    before doc body command={\textcolor{black!80!white}{\textsc{Deprecated! }}}
  },
}

\ExplSyntaxOn
\NewDocumentCommand \docColorExample { m }
  {
    \color_group_begin:
    \color_select:n { #1 }
    \rule[-1mm]{5mm}{4mm}~\docColor{#1}
    \color_group_end:
  }
\ExplSyntaxOff
}{}

\renewcommand*\l@subsection{\@dottedtocline{2}{1.5em}{2.7em}}
\renewcommand*\l@subsubsection{\@dottedtocline{3}{4.2em}{3.2em}}

\RequirePackage[T1]{fontenc}
\RequirePackage[english]{babel}
\RequirePackage{lmodern,parskip,array}
\RequirePackage[svgnames,table]{xcolor}
\RequirePackage{tikz}
\RequirePackage{varioref}
\RequirePackage[makeindex]{imakeidx}
\RequirePackage[bookmarks,raiselinks,pageanchor,hyperindex,colorlinks]{hyperref}
\RequirePackage{cleveref}
\RequirePackage{varwidth,incgraph}
\RequirePackage[official]{eurosym}

\RequirePackage[a4paper,left=2.5cm,right=2.5cm,top=1.5cm,bottom=1.5cm,
    marginparsep=3mm,marginparwidth=18mm,
    headheight=0mm,headsep=0cm,
    footskip=1.5cm,includeheadfoot%,showframe
    ]{geometry}
\RequirePackage{fancyhdr}
\fancyhf{}
\fancyfoot[C]{\thepage}%
\renewcommand{\headrulewidth}{0pt}
\renewcommand{\footrulewidth}{0pt}
\pagestyle{fancy}
\tolerance=2000%
\setlength{\emergencystretch}{20pt}%

\RequirePackage{array,tabularx,booktabs}
\RequirePackage{lipsum}
\RequirePackage[extendedchars,encoding,filenameencoding=utf-8]{grffile}
\RequirePackage{siunitx,xfp}

\RequirePackage{tcolorbox}
\tcbuselibrary{skins,minted,breakable,documentation,raster}

\ifdefined\didecpackageprefix
  \RequirePackage{\didecpackageprefix didec}
\else
  \RequirePackage{didec}
\fi

\definecolor{Green_Dark}{rgb}{0.078431,0.407843,0.176471}
\definecolor{Blue_Dark}{rgb}{0.090196,0.211765,0.364706}
\definecolor{Blue_Bright}{rgb}{0.858824,0.898039,0.945098}

\colorlet{Blue_Gray}{blue!50!gray}

\tcbset{skin=enhanced,
  doc head={colback=yellow!10!white,interior style=fill},
  doc head key={colback=magenta!5!white,interior style=fill},
  color key=DarkViolet,
  color value=Teal,
  color color=Teal,
  color counter=Orange!85!black,
  color length=Orange!85!black,
  index colorize,
  index annotate,
  beforeafter example/.style={
    before skip=4pt plus 2pt minus 1pt,
    after skip=8pt plus 4pt minus 2pt
  },
  docexample/.style={bicolor,
    beforeafter example,
    arc=0.66mm,
    boxrule=0.33mm,
    fonttitle=\bfseries,
    fontlower=\footnotesize,
    colframe=Blue_Gray,
    colback=Blue_Gray!5!white,
    colbacklower=white,
    drop small lifted shadow,
    listing engine=minted,
    documentation minted options={tabsize=2,fontsize=\small,breaklines,autogobble},
    documentation minted style=colorful,
    },
}
\urlstyle{sf}

\newtcblisting{fullexample}[1]{docexample,minted style=colorful,
  listing and comment,pdf comment,freeze pdf,compilable listing,#1}

\DeclareTotalTCBox{\myverb}{ O{} v }{tile,fontupper=\ttfamily,nobeforeafter,
  tcbox raise base,boxsep=0.5mm,top=0pt,bottom=0pt,left=0pt,right=0pt,boxrule=0.3mm,
  colback=yellow!10,
  borderline horizontal={0.3mm}{0pt}{red!50},
  #1}{#2}

\renewcommand*{\tcbdocnew}[1]{\textcolor{green!50!black}{\sffamily\bfseries N} #1}
\renewcommand*{\tcbdocupdated}[1]{\textcolor{blue!75!black}{\sffamily\bfseries U} #1}

\newcommand{\formatPackageOption}[1]{package option}
\newcommand{\formatPackageOptions}[1]{Package options}

\tcbmakedocSubKey[doc key prefix=]{docDidecKey}{didec}
\tcbmakedocSubKeys[doc key prefix=]{docDidecKeys}{didec}
\tcbmakedocSubKey[doc key prefix=,
    index key formatter=\formatPackageOption,
    index keys formatter=\formatPackageOptions
]{docPackageOption}{didec}
\tcbmakedocSubKeys[doc key prefix=,
    index key formatter=\formatPackageOption,
    index keys formatter=\formatPackageOptions
]{docPackageOptions}{didec}


\tcbset{
  deprecated*/.style={%
    doc head={colback=black!1!white,interior style=fill},
    doc head key={colback=black!3!white,interior style=fill},
    color command={black!80!white},
    color option={black!60!white},
  },
  deprecated/.style={%
    deprecated*,
    before doc body command={\textcolor{black!80!white}{\textsc{Deprecated! }}}
  },
}

\ExplSyntaxOn
\NewDocumentCommand \docColorExample { m }
  {
    \color_group_begin:
    \color_select:n { #1 }
    \rule[-1mm]{5mm}{4mm}~\docColor{#1}
    \color_group_end:
  }
\ExplSyntaxOff
}{\def\didecpackageprefix{}}
\documentclass[a4paper,11pt]{ltxdoc}
\usepackage[a4paper,textwidth=8cm]{geometry}
\usepackage{lmodern,array,booktabs,incgraph}
\usepackage[skins,documentation]{tcolorbox}
% \LaTeX-Main\
% !TeX encoding=UTF-8
% !TeX spellcheck=en_US
%%
%% The LaTeX package didec - version 1.0.0 (2024/02/28)
%% didec.tex: Manual
%%
%% -------------------------------------------------------------------------------------------
%% Copyright (c) 2024-2024 by Prof. Dr. Dr. Thomas F. Sturm <thomas dot sturm at unibw dot de>
%% -------------------------------------------------------------------------------------------
%%
%% This work may be distributed and/or modified under the
%% conditions of the LaTeX Project Public License, either version 1.3
%% of this license or (at your option) any later version.
%% The latest version of this license is in
%%   http://www.latex-project.org/lppl.txt
%% and version 1.3 or later is part of all distributions of LaTeX
%% version 2005/12/01 or later.
%%
%% This work has the LPPL maintenance status `author-maintained'.
%%
%% This work consists of all files listed in README
%%
\documentclass[a4paper,11pt]{ltxdoc}
\usepackage{didec.doc}

\hypersetup{
  pdftitle={Manual for the didec package},
  pdfauthor={Thomas F. Sturm},
  pdfsubject={Calculations with two decimal places},
  pdfkeywords={fixed-point arithmetic, two decimal places, calculations, cash book, bookkeeping, accountancy}
}

\makeindex

\pdfsuppresswarningpagegroup=1

\DeclareUnicodeCharacter{20AC}{\euro{}}
\didecsetup{
    decimal-separator  = {,},
    grouping-separator = {.},
    currency           = {}{\;€},
}

\didecnew{A}
\didecnew{B}
\didecnew{C}


%%%%%%%%%%%%%%%%%%%%%%%%%%%%%%%%%%%%%%%%%%%%%%%%%
\begin{document}

\begin{center}
\begin{tcolorbox}[enhanced,hbox,tikznode,left=8mm,right=8mm,boxrule=0.4pt,
  colback=white,colframe=Blue_Gray,
  drop lifted shadow=Blue_Gray!50,arc is angular,
  before=\par\vspace*{5mm},after=\par\bigskip]
{\bfseries\LARGE The didec package}\\[3mm]
{\large Manual for version \version\ (\datum)}
\end{tcolorbox}
{\large Thomas F.~Sturm%
  \footnote{Prof.~Dr.~Dr.~Thomas F.~Sturm, Institut f\"{u}r Mathematik und Informatik,
    University of the Bundeswehr Munich, D-85577 Neubiberg, Germany;
     email: \href{mailto:thomas.sturm@unibw.de}{thomas.sturm@unibw.de}}\par\medskip
\normalsize%
\url{https://www.ctan.org/pkg/didec}\par
\url{https://github.com/T-F-S/didec}
}
\end{center}
\bigskip
\begin{absquote}
  \begin{center}\bfseries Abstract\end{center}
  The \texttt{didec} package supports fixed-point arithmetic with two
  decimal places (\emph{di-decimal}) which is typical for financial transactions
  in many currencies. The intended use case is (personal) bookkeeping.
\end{absquote}

\tableofcontents

%%%%%%%%%%%%%%%%%%%%%%%%%%%%%%%%%%%%%%%%%%%%%%%%%%%%%%%%%%%%%%%%%%%%%%%%%%%%%%%%
\clearpage
\section{Quick start}%
For the impatient: this package provides fixed-point arithmetic with
two decimal places. You may use it for any purpose where exactly two decimal
places are needed or suffice, but the main application case is (personal) bookkeeping.

Say, John wants to keep track about his money. With
\begin{dispListing}
\didecnew{John}
\end{dispListing}
\didecnew{John}
a so-called \emph{didec variable} is created to store currency values.

Now, lets fill in some money:
\begin{dispListing}
\didecset{John}{1000}
\end{dispListing}
\didecset{John}{1000}

How much money has John now?
\begin{dispExample}
\didecuse{John}
\end{dispExample}

Obviously, John has Euros and the amount is displayed in a German style manor.
Of course, this can be adapted to your liking, see \refCom{didecsetup}.
But now, let us spend some money.
\begin{dispExample}
\didecsub{John}{19.75}
\didecuse{John}
\end{dispExample}

Preferably, the package functions are used inside convenient user commands.
You can choose between \LaTeX2e and \LaTeX3 programming layer functions.

For this quick start, we make some convenience commands for John.
|\transaction| shall be one purchase made in cash and |\transaction*| one made
by credit card.
\begin{dispExample*}{breakable}
%\usepackage{booktabs}
\didecsetup{english,
  currency          = {\pounds}{},
  currency-negative = {-\pounds}{},
}

%\didecnew{John}
\didecnew{cash}
\didecnew{credit}
\didecnew{transaction}

% Let's keep record in a table
\NewDocumentEnvironment{householdbook}{}{%
    \begin{center}
    \begin{tabular}{lp{9cm}rr}\toprule%
    & Transaction & Expenses & Budget\\\midrule
  }{%
    \midrule%
    \multicolumn{3}{r}{Cash} & \dideccoluse{cash}\\%
    \multicolumn{3}{r}{Credit card} & \dideccoluse{credit}\\\bottomrule%
    \end{tabular}%
    \end{center}
  }

% One transaction
\NewDocumentCommand \transaction{ s m m }
  {
    \didecset{transaction}{#3}%
    \didecsub{John}{transaction}%
    \IfBooleanTF {#1}%
      {\didecsub{credit}{transaction}CC}%
      {\didecsub{cash}{transaction}}%
    & #2
    & \dideccolinvuse{transaction}
    & \dideccoluse{John}\\%
  }

% Ready to start our tiny accountancy

\didecset{John}{1000}        % John's money
\didecsetequal{cash}{John}   % in cash
\didecset{credit}{0}         % blank credit card

\begin{householdbook}
  \transaction{Coffee break with snack}{19.75}
  \transaction*{Refuel}{62.87}
  \transaction{Gift from Aunt Mary for helping her}{-30}
  \transaction{Parking meter}{4.50}
  \transaction*{Shopping for weekend}{147.23}
  \transaction*{Fancy thing on the Internet}{270}
  \transaction*{Cinema}{17.70}
\end{householdbook}

\end{dispExample*}




%%%%%%%%%%%%%%%%%%%%%%%%%%%%%%%%%%%%%%%%%%%%%%%%%%%%%%%%%%%%%%%%%%%%%%%%%%%%%%%%
\clearpage
\section{Didec kernels and didec expressions}%

All calculations are done on cent basis as integer operations, but all
displayed figures have two decimal places which give the name for the package
(\emph{di}-\emph{dec}imal).

The package provides two numerical kernels which can be selected mutually
by package options \refKey{didec/int} or \refKey{didec/fp}.
The \refKey{didec/int} kernel is faster, but provides a smaller number range,
while the \refKey{didec/int} kernel is slower with a larger number range.
For ordinary people like you and me, the \refKey{didec/int} kernel will suffice
to do all personal financial calculations.
Upgrading later from \refKey{didec/int} to \refKey{didec/fp} is a matter of
just switching the package option setting.

%-------------------------------------------------------------------------------
\begin{docPackageOptions}{{
    doc name = int,
    doc description = {no value, initially set},
  }}
Selects the \meta{int} (integer) based numerical kernel with up to
9.33\footnote{joke for the mathematicians} significant figures and fast computation.
The number range for valid figures $n$ is:
\begin{center}
$ -21\;474\;836.47 \le n \le 21\;474\;836.47 $
\end{center}
\begin{dispListing}
\usepackage[int]{didec}
\end{dispListing}
Using figures outside the valid range will result in \LaTeX\ errors complaining
about too large numbers.
\end{docPackageOptions}
%-------------------------------------------------------------------------------


%-------------------------------------------------------------------------------
\begin{docPackageOptions}{{
    doc name = fp,
    doc description = {no value, initially unset},
  }}
Selects the \meta{fp} (floating-point) based numerical kernel with up to
16 significant figures and somewhat slower computation.
The number range for valid figures $n$ is:
\begin{center}
$ -99\;999\;999\;999\;999.99 \le n \le 99\;999\;999\;999\;999.99 $
\end{center}
\begin{dispListing}
\usepackage[fp]{didec}
\end{dispListing}
Using figures outside the valid range will result in \emph{silent calculation errors},
because \LaTeX3 \meta{fp} can use much larger numbers but is restricted to 16
significant figures.
\end{docPackageOptions}
%-------------------------------------------------------------------------------

\medskip

In the following, a \meta{didec expr} (didec expression) denotes one of
the following:
\begin{itemize}
\item a number in floating-point notation, e.g. |123.45|
\item a number in floating-comma notation, e.g. |123,45|
\item a \meta{didec var} (didec variable), e.g. |expenses|.
\end{itemize}
Note that the notation \meta{didec expr} is inspired by \meta{int expr} and \meta{fp expr}
from \LaTeX3 but is in comparison very restricted and allows only the three choices above.
A \meta{didec expr} is always expanded and spaces are trimmed.

\medskip

Many provided commands or functions of the package come in three flavors, for
example:
\begin{itemize}
\item \refCom{didecadd}: This is a user command where arguments are space trimmed
  and some check on variable existence is done.
  Not existing variables are reported by speaking error messages (not in all cases!).
\item \refCom{didec_gadd_check:nn}: This a programming layer function
  with no space trimming for arguments, but some check on variable existence is done.
  Not existing variables are reported by speaking error messages (not in all cases!).
\item \refCom{didec_gadd:nn}: This a programming layer function
  with no space trimming for arguments and no check on variable existence.
  Not existing variables give strange errors.
  This is the fasted function and base of the others above.
\end{itemize}


\clearpage
The following tables compare the computation time for selected functions of
the package for the two numerical kernels. Time values will differ on other
computers and also depend on selected values for the examples calculations.
Nevertheless, you get an impression of the differences.


%-------------------------------------------------------------------------------
\begin{tcbverbatimwrite}{\jobname.temp}
\begin{tcblisting}{
  blankest,
  pdf comment,
  comment style={raster columns=1,blankest,enhanced/.style=},
  comment only,
  freeze pdf,
  compilable listing,
  run pdflatex
}
\IfFileExists{didec.doc.cfg}{% !TeX root = didec.tex
% !TeX encoding=UTF-8
%% The LaTeX package didec - version 1.0.0 (2024/02/28)
%% didec.doc.sty: style file for the documentation
%%
%% -------------------------------------------------------------------------------------------
%% Copyright (c) 2024-2024 by Prof. Dr. Dr. Thomas F. Sturm <thomas dot sturm at unibw dot de>
%% -------------------------------------------------------------------------------------------
%%
%% This work may be distributed and/or modified under the
%% conditions of the LaTeX Project Public License, either version 1.3
%% of this license or (at your option) any later version.
%% The latest version of this license is in
%%   http://www.latex-project.org/lppl.txt
%% and version 1.3 or later is part of all distributions of LaTeX
%% version 2005/12/01 or later.
%%
%% This work has the LPPL maintenance status `author-maintained'.
%%
%% This work consists of all files listed in README.md
%%
% include file (style) of didec.tex (manual of the LaTeX package didec)

\def\version{1.0.0}%
\def\datum{2024/02/28}%

\IfFileExists{didec.doc.cfg}{% !TeX root = didec.tex
% !TeX encoding=UTF-8
%% The LaTeX package didec - version 1.0.0 (2024/02/28)
%% didec.doc.sty: style file for the documentation
%%
%% -------------------------------------------------------------------------------------------
%% Copyright (c) 2024-2024 by Prof. Dr. Dr. Thomas F. Sturm <thomas dot sturm at unibw dot de>
%% -------------------------------------------------------------------------------------------
%%
%% This work may be distributed and/or modified under the
%% conditions of the LaTeX Project Public License, either version 1.3
%% of this license or (at your option) any later version.
%% The latest version of this license is in
%%   http://www.latex-project.org/lppl.txt
%% and version 1.3 or later is part of all distributions of LaTeX
%% version 2005/12/01 or later.
%%
%% This work has the LPPL maintenance status `author-maintained'.
%%
%% This work consists of all files listed in README.md
%%
% include file (style) of didec.tex (manual of the LaTeX package didec)

\def\version{1.0.0}%
\def\datum{2024/02/28}%

\IfFileExists{didec.doc.cfg}{\input{didec.doc.cfg}}{}

\renewcommand*\l@subsection{\@dottedtocline{2}{1.5em}{2.7em}}
\renewcommand*\l@subsubsection{\@dottedtocline{3}{4.2em}{3.2em}}

\RequirePackage[T1]{fontenc}
\RequirePackage[english]{babel}
\RequirePackage{lmodern,parskip,array}
\RequirePackage[svgnames,table]{xcolor}
\RequirePackage{tikz}
\RequirePackage{varioref}
\RequirePackage[makeindex]{imakeidx}
\RequirePackage[bookmarks,raiselinks,pageanchor,hyperindex,colorlinks]{hyperref}
\RequirePackage{cleveref}
\RequirePackage{varwidth,incgraph}
\RequirePackage[official]{eurosym}

\RequirePackage[a4paper,left=2.5cm,right=2.5cm,top=1.5cm,bottom=1.5cm,
    marginparsep=3mm,marginparwidth=18mm,
    headheight=0mm,headsep=0cm,
    footskip=1.5cm,includeheadfoot%,showframe
    ]{geometry}
\RequirePackage{fancyhdr}
\fancyhf{}
\fancyfoot[C]{\thepage}%
\renewcommand{\headrulewidth}{0pt}
\renewcommand{\footrulewidth}{0pt}
\pagestyle{fancy}
\tolerance=2000%
\setlength{\emergencystretch}{20pt}%

\RequirePackage{array,tabularx,booktabs}
\RequirePackage{lipsum}
\RequirePackage[extendedchars,encoding,filenameencoding=utf-8]{grffile}
\RequirePackage{siunitx,xfp}

\RequirePackage{tcolorbox}
\tcbuselibrary{skins,minted,breakable,documentation,raster}

\ifdefined\didecpackageprefix
  \RequirePackage{\didecpackageprefix didec}
\else
  \RequirePackage{didec}
\fi

\definecolor{Green_Dark}{rgb}{0.078431,0.407843,0.176471}
\definecolor{Blue_Dark}{rgb}{0.090196,0.211765,0.364706}
\definecolor{Blue_Bright}{rgb}{0.858824,0.898039,0.945098}

\colorlet{Blue_Gray}{blue!50!gray}

\tcbset{skin=enhanced,
  doc head={colback=yellow!10!white,interior style=fill},
  doc head key={colback=magenta!5!white,interior style=fill},
  color key=DarkViolet,
  color value=Teal,
  color color=Teal,
  color counter=Orange!85!black,
  color length=Orange!85!black,
  index colorize,
  index annotate,
  beforeafter example/.style={
    before skip=4pt plus 2pt minus 1pt,
    after skip=8pt plus 4pt minus 2pt
  },
  docexample/.style={bicolor,
    beforeafter example,
    arc=0.66mm,
    boxrule=0.33mm,
    fonttitle=\bfseries,
    fontlower=\footnotesize,
    colframe=Blue_Gray,
    colback=Blue_Gray!5!white,
    colbacklower=white,
    drop small lifted shadow,
    listing engine=minted,
    documentation minted options={tabsize=2,fontsize=\small,breaklines,autogobble},
    documentation minted style=colorful,
    },
}
\urlstyle{sf}

\newtcblisting{fullexample}[1]{docexample,minted style=colorful,
  listing and comment,pdf comment,freeze pdf,compilable listing,#1}

\DeclareTotalTCBox{\myverb}{ O{} v }{tile,fontupper=\ttfamily,nobeforeafter,
  tcbox raise base,boxsep=0.5mm,top=0pt,bottom=0pt,left=0pt,right=0pt,boxrule=0.3mm,
  colback=yellow!10,
  borderline horizontal={0.3mm}{0pt}{red!50},
  #1}{#2}

\renewcommand*{\tcbdocnew}[1]{\textcolor{green!50!black}{\sffamily\bfseries N} #1}
\renewcommand*{\tcbdocupdated}[1]{\textcolor{blue!75!black}{\sffamily\bfseries U} #1}

\newcommand{\formatPackageOption}[1]{package option}
\newcommand{\formatPackageOptions}[1]{Package options}

\tcbmakedocSubKey[doc key prefix=]{docDidecKey}{didec}
\tcbmakedocSubKeys[doc key prefix=]{docDidecKeys}{didec}
\tcbmakedocSubKey[doc key prefix=,
    index key formatter=\formatPackageOption,
    index keys formatter=\formatPackageOptions
]{docPackageOption}{didec}
\tcbmakedocSubKeys[doc key prefix=,
    index key formatter=\formatPackageOption,
    index keys formatter=\formatPackageOptions
]{docPackageOptions}{didec}


\tcbset{
  deprecated*/.style={%
    doc head={colback=black!1!white,interior style=fill},
    doc head key={colback=black!3!white,interior style=fill},
    color command={black!80!white},
    color option={black!60!white},
  },
  deprecated/.style={%
    deprecated*,
    before doc body command={\textcolor{black!80!white}{\textsc{Deprecated! }}}
  },
}

\ExplSyntaxOn
\NewDocumentCommand \docColorExample { m }
  {
    \color_group_begin:
    \color_select:n { #1 }
    \rule[-1mm]{5mm}{4mm}~\docColor{#1}
    \color_group_end:
  }
\ExplSyntaxOff
}{}

\renewcommand*\l@subsection{\@dottedtocline{2}{1.5em}{2.7em}}
\renewcommand*\l@subsubsection{\@dottedtocline{3}{4.2em}{3.2em}}

\RequirePackage[T1]{fontenc}
\RequirePackage[english]{babel}
\RequirePackage{lmodern,parskip,array}
\RequirePackage[svgnames,table]{xcolor}
\RequirePackage{tikz}
\RequirePackage{varioref}
\RequirePackage[makeindex]{imakeidx}
\RequirePackage[bookmarks,raiselinks,pageanchor,hyperindex,colorlinks]{hyperref}
\RequirePackage{cleveref}
\RequirePackage{varwidth,incgraph}
\RequirePackage[official]{eurosym}

\RequirePackage[a4paper,left=2.5cm,right=2.5cm,top=1.5cm,bottom=1.5cm,
    marginparsep=3mm,marginparwidth=18mm,
    headheight=0mm,headsep=0cm,
    footskip=1.5cm,includeheadfoot%,showframe
    ]{geometry}
\RequirePackage{fancyhdr}
\fancyhf{}
\fancyfoot[C]{\thepage}%
\renewcommand{\headrulewidth}{0pt}
\renewcommand{\footrulewidth}{0pt}
\pagestyle{fancy}
\tolerance=2000%
\setlength{\emergencystretch}{20pt}%

\RequirePackage{array,tabularx,booktabs}
\RequirePackage{lipsum}
\RequirePackage[extendedchars,encoding,filenameencoding=utf-8]{grffile}
\RequirePackage{siunitx,xfp}

\RequirePackage{tcolorbox}
\tcbuselibrary{skins,minted,breakable,documentation,raster}

\ifdefined\didecpackageprefix
  \RequirePackage{\didecpackageprefix didec}
\else
  \RequirePackage{didec}
\fi

\definecolor{Green_Dark}{rgb}{0.078431,0.407843,0.176471}
\definecolor{Blue_Dark}{rgb}{0.090196,0.211765,0.364706}
\definecolor{Blue_Bright}{rgb}{0.858824,0.898039,0.945098}

\colorlet{Blue_Gray}{blue!50!gray}

\tcbset{skin=enhanced,
  doc head={colback=yellow!10!white,interior style=fill},
  doc head key={colback=magenta!5!white,interior style=fill},
  color key=DarkViolet,
  color value=Teal,
  color color=Teal,
  color counter=Orange!85!black,
  color length=Orange!85!black,
  index colorize,
  index annotate,
  beforeafter example/.style={
    before skip=4pt plus 2pt minus 1pt,
    after skip=8pt plus 4pt minus 2pt
  },
  docexample/.style={bicolor,
    beforeafter example,
    arc=0.66mm,
    boxrule=0.33mm,
    fonttitle=\bfseries,
    fontlower=\footnotesize,
    colframe=Blue_Gray,
    colback=Blue_Gray!5!white,
    colbacklower=white,
    drop small lifted shadow,
    listing engine=minted,
    documentation minted options={tabsize=2,fontsize=\small,breaklines,autogobble},
    documentation minted style=colorful,
    },
}
\urlstyle{sf}

\newtcblisting{fullexample}[1]{docexample,minted style=colorful,
  listing and comment,pdf comment,freeze pdf,compilable listing,#1}

\DeclareTotalTCBox{\myverb}{ O{} v }{tile,fontupper=\ttfamily,nobeforeafter,
  tcbox raise base,boxsep=0.5mm,top=0pt,bottom=0pt,left=0pt,right=0pt,boxrule=0.3mm,
  colback=yellow!10,
  borderline horizontal={0.3mm}{0pt}{red!50},
  #1}{#2}

\renewcommand*{\tcbdocnew}[1]{\textcolor{green!50!black}{\sffamily\bfseries N} #1}
\renewcommand*{\tcbdocupdated}[1]{\textcolor{blue!75!black}{\sffamily\bfseries U} #1}

\newcommand{\formatPackageOption}[1]{package option}
\newcommand{\formatPackageOptions}[1]{Package options}

\tcbmakedocSubKey[doc key prefix=]{docDidecKey}{didec}
\tcbmakedocSubKeys[doc key prefix=]{docDidecKeys}{didec}
\tcbmakedocSubKey[doc key prefix=,
    index key formatter=\formatPackageOption,
    index keys formatter=\formatPackageOptions
]{docPackageOption}{didec}
\tcbmakedocSubKeys[doc key prefix=,
    index key formatter=\formatPackageOption,
    index keys formatter=\formatPackageOptions
]{docPackageOptions}{didec}


\tcbset{
  deprecated*/.style={%
    doc head={colback=black!1!white,interior style=fill},
    doc head key={colback=black!3!white,interior style=fill},
    color command={black!80!white},
    color option={black!60!white},
  },
  deprecated/.style={%
    deprecated*,
    before doc body command={\textcolor{black!80!white}{\textsc{Deprecated! }}}
  },
}

\ExplSyntaxOn
\NewDocumentCommand \docColorExample { m }
  {
    \color_group_begin:
    \color_select:n { #1 }
    \rule[-1mm]{5mm}{4mm}~\docColor{#1}
    \color_group_end:
  }
\ExplSyntaxOff
}{\def\didecpackageprefix{}}
\documentclass[a4paper,11pt]{ltxdoc}
\usepackage[a4paper,textwidth=8cm]{geometry}
\usepackage{lmodern,array,booktabs,incgraph}
\usepackage[skins,documentation]{tcolorbox}
% \LaTeX-Main\
% !TeX encoding=UTF-8
% !TeX spellcheck=en_US
%%
%% The LaTeX package didec - version 1.0.0 (2024/02/28)
%% didec.tex: Manual
%%
%% -------------------------------------------------------------------------------------------
%% Copyright (c) 2024-2024 by Prof. Dr. Dr. Thomas F. Sturm <thomas dot sturm at unibw dot de>
%% -------------------------------------------------------------------------------------------
%%
%% This work may be distributed and/or modified under the
%% conditions of the LaTeX Project Public License, either version 1.3
%% of this license or (at your option) any later version.
%% The latest version of this license is in
%%   http://www.latex-project.org/lppl.txt
%% and version 1.3 or later is part of all distributions of LaTeX
%% version 2005/12/01 or later.
%%
%% This work has the LPPL maintenance status `author-maintained'.
%%
%% This work consists of all files listed in README
%%
\documentclass[a4paper,11pt]{ltxdoc}
\usepackage{didec.doc}

\hypersetup{
  pdftitle={Manual for the didec package},
  pdfauthor={Thomas F. Sturm},
  pdfsubject={Calculations with two decimal places},
  pdfkeywords={fixed-point arithmetic, two decimal places, calculations, cash book, bookkeeping, accountancy}
}

\makeindex

\pdfsuppresswarningpagegroup=1

\DeclareUnicodeCharacter{20AC}{\euro{}}
\didecsetup{
    decimal-separator  = {,},
    grouping-separator = {.},
    currency           = {}{\;€},
}

\didecnew{A}
\didecnew{B}
\didecnew{C}


%%%%%%%%%%%%%%%%%%%%%%%%%%%%%%%%%%%%%%%%%%%%%%%%%
\begin{document}

\begin{center}
\begin{tcolorbox}[enhanced,hbox,tikznode,left=8mm,right=8mm,boxrule=0.4pt,
  colback=white,colframe=Blue_Gray,
  drop lifted shadow=Blue_Gray!50,arc is angular,
  before=\par\vspace*{5mm},after=\par\bigskip]
{\bfseries\LARGE The didec package}\\[3mm]
{\large Manual for version \version\ (\datum)}
\end{tcolorbox}
{\large Thomas F.~Sturm%
  \footnote{Prof.~Dr.~Dr.~Thomas F.~Sturm, Institut f\"{u}r Mathematik und Informatik,
    University of the Bundeswehr Munich, D-85577 Neubiberg, Germany;
     email: \href{mailto:thomas.sturm@unibw.de}{thomas.sturm@unibw.de}}\par\medskip
\normalsize%
\url{https://www.ctan.org/pkg/didec}\par
\url{https://github.com/T-F-S/didec}
}
\end{center}
\bigskip
\begin{absquote}
  \begin{center}\bfseries Abstract\end{center}
  The \texttt{didec} package supports fixed-point arithmetic with two
  decimal places (\emph{di-decimal}) which is typical for financial transactions
  in many currencies. The intended use case is (personal) bookkeeping.
\end{absquote}

\tableofcontents

%%%%%%%%%%%%%%%%%%%%%%%%%%%%%%%%%%%%%%%%%%%%%%%%%%%%%%%%%%%%%%%%%%%%%%%%%%%%%%%%
\clearpage
\section{Quick start}%
For the impatient: this package provides fixed-point arithmetic with
two decimal places. You may use it for any purpose where exactly two decimal
places are needed or suffice, but the main application case is (personal) bookkeeping.

Say, John wants to keep track about his money. With
\begin{dispListing}
\didecnew{John}
\end{dispListing}
\didecnew{John}
a so-called \emph{didec variable} is created to store currency values.

Now, lets fill in some money:
\begin{dispListing}
\didecset{John}{1000}
\end{dispListing}
\didecset{John}{1000}

How much money has John now?
\begin{dispExample}
\didecuse{John}
\end{dispExample}

Obviously, John has Euros and the amount is displayed in a German style manor.
Of course, this can be adapted to your liking, see \refCom{didecsetup}.
But now, let us spend some money.
\begin{dispExample}
\didecsub{John}{19.75}
\didecuse{John}
\end{dispExample}

Preferably, the package functions are used inside convenient user commands.
You can choose between \LaTeX2e and \LaTeX3 programming layer functions.

For this quick start, we make some convenience commands for John.
|\transaction| shall be one purchase made in cash and |\transaction*| one made
by credit card.
\begin{dispExample*}{breakable}
%\usepackage{booktabs}
\didecsetup{english,
  currency          = {\pounds}{},
  currency-negative = {-\pounds}{},
}

%\didecnew{John}
\didecnew{cash}
\didecnew{credit}
\didecnew{transaction}

% Let's keep record in a table
\NewDocumentEnvironment{householdbook}{}{%
    \begin{center}
    \begin{tabular}{lp{9cm}rr}\toprule%
    & Transaction & Expenses & Budget\\\midrule
  }{%
    \midrule%
    \multicolumn{3}{r}{Cash} & \dideccoluse{cash}\\%
    \multicolumn{3}{r}{Credit card} & \dideccoluse{credit}\\\bottomrule%
    \end{tabular}%
    \end{center}
  }

% One transaction
\NewDocumentCommand \transaction{ s m m }
  {
    \didecset{transaction}{#3}%
    \didecsub{John}{transaction}%
    \IfBooleanTF {#1}%
      {\didecsub{credit}{transaction}CC}%
      {\didecsub{cash}{transaction}}%
    & #2
    & \dideccolinvuse{transaction}
    & \dideccoluse{John}\\%
  }

% Ready to start our tiny accountancy

\didecset{John}{1000}        % John's money
\didecsetequal{cash}{John}   % in cash
\didecset{credit}{0}         % blank credit card

\begin{householdbook}
  \transaction{Coffee break with snack}{19.75}
  \transaction*{Refuel}{62.87}
  \transaction{Gift from Aunt Mary for helping her}{-30}
  \transaction{Parking meter}{4.50}
  \transaction*{Shopping for weekend}{147.23}
  \transaction*{Fancy thing on the Internet}{270}
  \transaction*{Cinema}{17.70}
\end{householdbook}

\end{dispExample*}




%%%%%%%%%%%%%%%%%%%%%%%%%%%%%%%%%%%%%%%%%%%%%%%%%%%%%%%%%%%%%%%%%%%%%%%%%%%%%%%%
\clearpage
\section{Didec kernels and didec expressions}%

All calculations are done on cent basis as integer operations, but all
displayed figures have two decimal places which give the name for the package
(\emph{di}-\emph{dec}imal).

The package provides two numerical kernels which can be selected mutually
by package options \refKey{didec/int} or \refKey{didec/fp}.
The \refKey{didec/int} kernel is faster, but provides a smaller number range,
while the \refKey{didec/int} kernel is slower with a larger number range.
For ordinary people like you and me, the \refKey{didec/int} kernel will suffice
to do all personal financial calculations.
Upgrading later from \refKey{didec/int} to \refKey{didec/fp} is a matter of
just switching the package option setting.

%-------------------------------------------------------------------------------
\begin{docPackageOptions}{{
    doc name = int,
    doc description = {no value, initially set},
  }}
Selects the \meta{int} (integer) based numerical kernel with up to
9.33\footnote{joke for the mathematicians} significant figures and fast computation.
The number range for valid figures $n$ is:
\begin{center}
$ -21\;474\;836.47 \le n \le 21\;474\;836.47 $
\end{center}
\begin{dispListing}
\usepackage[int]{didec}
\end{dispListing}
Using figures outside the valid range will result in \LaTeX\ errors complaining
about too large numbers.
\end{docPackageOptions}
%-------------------------------------------------------------------------------


%-------------------------------------------------------------------------------
\begin{docPackageOptions}{{
    doc name = fp,
    doc description = {no value, initially unset},
  }}
Selects the \meta{fp} (floating-point) based numerical kernel with up to
16 significant figures and somewhat slower computation.
The number range for valid figures $n$ is:
\begin{center}
$ -99\;999\;999\;999\;999.99 \le n \le 99\;999\;999\;999\;999.99 $
\end{center}
\begin{dispListing}
\usepackage[fp]{didec}
\end{dispListing}
Using figures outside the valid range will result in \emph{silent calculation errors},
because \LaTeX3 \meta{fp} can use much larger numbers but is restricted to 16
significant figures.
\end{docPackageOptions}
%-------------------------------------------------------------------------------

\medskip

In the following, a \meta{didec expr} (didec expression) denotes one of
the following:
\begin{itemize}
\item a number in floating-point notation, e.g. |123.45|
\item a number in floating-comma notation, e.g. |123,45|
\item a \meta{didec var} (didec variable), e.g. |expenses|.
\end{itemize}
Note that the notation \meta{didec expr} is inspired by \meta{int expr} and \meta{fp expr}
from \LaTeX3 but is in comparison very restricted and allows only the three choices above.
A \meta{didec expr} is always expanded and spaces are trimmed.

\medskip

Many provided commands or functions of the package come in three flavors, for
example:
\begin{itemize}
\item \refCom{didecadd}: This is a user command where arguments are space trimmed
  and some check on variable existence is done.
  Not existing variables are reported by speaking error messages (not in all cases!).
\item \refCom{didec_gadd_check:nn}: This a programming layer function
  with no space trimming for arguments, but some check on variable existence is done.
  Not existing variables are reported by speaking error messages (not in all cases!).
\item \refCom{didec_gadd:nn}: This a programming layer function
  with no space trimming for arguments and no check on variable existence.
  Not existing variables give strange errors.
  This is the fasted function and base of the others above.
\end{itemize}


\clearpage
The following tables compare the computation time for selected functions of
the package for the two numerical kernels. Time values will differ on other
computers and also depend on selected values for the examples calculations.
Nevertheless, you get an impression of the differences.


%-------------------------------------------------------------------------------
\begin{tcbverbatimwrite}{\jobname.temp}
\begin{tcblisting}{
  blankest,
  pdf comment,
  comment style={raster columns=1,blankest,enhanced/.style=},
  comment only,
  freeze pdf,
  compilable listing,
  run pdflatex
}
\IfFileExists{didec.doc.cfg}{% !TeX root = didec.tex
% !TeX encoding=UTF-8
%% The LaTeX package didec - version 1.0.0 (2024/02/28)
%% didec.doc.sty: style file for the documentation
%%
%% -------------------------------------------------------------------------------------------
%% Copyright (c) 2024-2024 by Prof. Dr. Dr. Thomas F. Sturm <thomas dot sturm at unibw dot de>
%% -------------------------------------------------------------------------------------------
%%
%% This work may be distributed and/or modified under the
%% conditions of the LaTeX Project Public License, either version 1.3
%% of this license or (at your option) any later version.
%% The latest version of this license is in
%%   http://www.latex-project.org/lppl.txt
%% and version 1.3 or later is part of all distributions of LaTeX
%% version 2005/12/01 or later.
%%
%% This work has the LPPL maintenance status `author-maintained'.
%%
%% This work consists of all files listed in README.md
%%
% include file (style) of didec.tex (manual of the LaTeX package didec)

\def\version{1.0.0}%
\def\datum{2024/02/28}%

\IfFileExists{didec.doc.cfg}{\input{didec.doc.cfg}}{}

\renewcommand*\l@subsection{\@dottedtocline{2}{1.5em}{2.7em}}
\renewcommand*\l@subsubsection{\@dottedtocline{3}{4.2em}{3.2em}}

\RequirePackage[T1]{fontenc}
\RequirePackage[english]{babel}
\RequirePackage{lmodern,parskip,array}
\RequirePackage[svgnames,table]{xcolor}
\RequirePackage{tikz}
\RequirePackage{varioref}
\RequirePackage[makeindex]{imakeidx}
\RequirePackage[bookmarks,raiselinks,pageanchor,hyperindex,colorlinks]{hyperref}
\RequirePackage{cleveref}
\RequirePackage{varwidth,incgraph}
\RequirePackage[official]{eurosym}

\RequirePackage[a4paper,left=2.5cm,right=2.5cm,top=1.5cm,bottom=1.5cm,
    marginparsep=3mm,marginparwidth=18mm,
    headheight=0mm,headsep=0cm,
    footskip=1.5cm,includeheadfoot%,showframe
    ]{geometry}
\RequirePackage{fancyhdr}
\fancyhf{}
\fancyfoot[C]{\thepage}%
\renewcommand{\headrulewidth}{0pt}
\renewcommand{\footrulewidth}{0pt}
\pagestyle{fancy}
\tolerance=2000%
\setlength{\emergencystretch}{20pt}%

\RequirePackage{array,tabularx,booktabs}
\RequirePackage{lipsum}
\RequirePackage[extendedchars,encoding,filenameencoding=utf-8]{grffile}
\RequirePackage{siunitx,xfp}

\RequirePackage{tcolorbox}
\tcbuselibrary{skins,minted,breakable,documentation,raster}

\ifdefined\didecpackageprefix
  \RequirePackage{\didecpackageprefix didec}
\else
  \RequirePackage{didec}
\fi

\definecolor{Green_Dark}{rgb}{0.078431,0.407843,0.176471}
\definecolor{Blue_Dark}{rgb}{0.090196,0.211765,0.364706}
\definecolor{Blue_Bright}{rgb}{0.858824,0.898039,0.945098}

\colorlet{Blue_Gray}{blue!50!gray}

\tcbset{skin=enhanced,
  doc head={colback=yellow!10!white,interior style=fill},
  doc head key={colback=magenta!5!white,interior style=fill},
  color key=DarkViolet,
  color value=Teal,
  color color=Teal,
  color counter=Orange!85!black,
  color length=Orange!85!black,
  index colorize,
  index annotate,
  beforeafter example/.style={
    before skip=4pt plus 2pt minus 1pt,
    after skip=8pt plus 4pt minus 2pt
  },
  docexample/.style={bicolor,
    beforeafter example,
    arc=0.66mm,
    boxrule=0.33mm,
    fonttitle=\bfseries,
    fontlower=\footnotesize,
    colframe=Blue_Gray,
    colback=Blue_Gray!5!white,
    colbacklower=white,
    drop small lifted shadow,
    listing engine=minted,
    documentation minted options={tabsize=2,fontsize=\small,breaklines,autogobble},
    documentation minted style=colorful,
    },
}
\urlstyle{sf}

\newtcblisting{fullexample}[1]{docexample,minted style=colorful,
  listing and comment,pdf comment,freeze pdf,compilable listing,#1}

\DeclareTotalTCBox{\myverb}{ O{} v }{tile,fontupper=\ttfamily,nobeforeafter,
  tcbox raise base,boxsep=0.5mm,top=0pt,bottom=0pt,left=0pt,right=0pt,boxrule=0.3mm,
  colback=yellow!10,
  borderline horizontal={0.3mm}{0pt}{red!50},
  #1}{#2}

\renewcommand*{\tcbdocnew}[1]{\textcolor{green!50!black}{\sffamily\bfseries N} #1}
\renewcommand*{\tcbdocupdated}[1]{\textcolor{blue!75!black}{\sffamily\bfseries U} #1}

\newcommand{\formatPackageOption}[1]{package option}
\newcommand{\formatPackageOptions}[1]{Package options}

\tcbmakedocSubKey[doc key prefix=]{docDidecKey}{didec}
\tcbmakedocSubKeys[doc key prefix=]{docDidecKeys}{didec}
\tcbmakedocSubKey[doc key prefix=,
    index key formatter=\formatPackageOption,
    index keys formatter=\formatPackageOptions
]{docPackageOption}{didec}
\tcbmakedocSubKeys[doc key prefix=,
    index key formatter=\formatPackageOption,
    index keys formatter=\formatPackageOptions
]{docPackageOptions}{didec}


\tcbset{
  deprecated*/.style={%
    doc head={colback=black!1!white,interior style=fill},
    doc head key={colback=black!3!white,interior style=fill},
    color command={black!80!white},
    color option={black!60!white},
  },
  deprecated/.style={%
    deprecated*,
    before doc body command={\textcolor{black!80!white}{\textsc{Deprecated! }}}
  },
}

\ExplSyntaxOn
\NewDocumentCommand \docColorExample { m }
  {
    \color_group_begin:
    \color_select:n { #1 }
    \rule[-1mm]{5mm}{4mm}~\docColor{#1}
    \color_group_end:
  }
\ExplSyntaxOff
}{\def\didecpackageprefix{}}
\documentclass[a4paper,11pt]{ltxdoc}
\usepackage[a4paper,textwidth=8cm]{geometry}
\usepackage{lmodern,array,booktabs,incgraph}
\usepackage[skins,documentation]{tcolorbox}
% \LaTeX-Main\
% !TeX encoding=UTF-8
% !TeX spellcheck=en_US
%%
%% The LaTeX package didec - version 1.0.0 (2024/02/28)
%% didec.tex: Manual
%%
%% -------------------------------------------------------------------------------------------
%% Copyright (c) 2024-2024 by Prof. Dr. Dr. Thomas F. Sturm <thomas dot sturm at unibw dot de>
%% -------------------------------------------------------------------------------------------
%%
%% This work may be distributed and/or modified under the
%% conditions of the LaTeX Project Public License, either version 1.3
%% of this license or (at your option) any later version.
%% The latest version of this license is in
%%   http://www.latex-project.org/lppl.txt
%% and version 1.3 or later is part of all distributions of LaTeX
%% version 2005/12/01 or later.
%%
%% This work has the LPPL maintenance status `author-maintained'.
%%
%% This work consists of all files listed in README
%%
\documentclass[a4paper,11pt]{ltxdoc}
\usepackage{didec.doc}

\hypersetup{
  pdftitle={Manual for the didec package},
  pdfauthor={Thomas F. Sturm},
  pdfsubject={Calculations with two decimal places},
  pdfkeywords={fixed-point arithmetic, two decimal places, calculations, cash book, bookkeeping, accountancy}
}

\makeindex

\pdfsuppresswarningpagegroup=1

\DeclareUnicodeCharacter{20AC}{\euro{}}
\didecsetup{
    decimal-separator  = {,},
    grouping-separator = {.},
    currency           = {}{\;€},
}

\didecnew{A}
\didecnew{B}
\didecnew{C}


%%%%%%%%%%%%%%%%%%%%%%%%%%%%%%%%%%%%%%%%%%%%%%%%%
\begin{document}

\begin{center}
\begin{tcolorbox}[enhanced,hbox,tikznode,left=8mm,right=8mm,boxrule=0.4pt,
  colback=white,colframe=Blue_Gray,
  drop lifted shadow=Blue_Gray!50,arc is angular,
  before=\par\vspace*{5mm},after=\par\bigskip]
{\bfseries\LARGE The didec package}\\[3mm]
{\large Manual for version \version\ (\datum)}
\end{tcolorbox}
{\large Thomas F.~Sturm%
  \footnote{Prof.~Dr.~Dr.~Thomas F.~Sturm, Institut f\"{u}r Mathematik und Informatik,
    University of the Bundeswehr Munich, D-85577 Neubiberg, Germany;
     email: \href{mailto:thomas.sturm@unibw.de}{thomas.sturm@unibw.de}}\par\medskip
\normalsize%
\url{https://www.ctan.org/pkg/didec}\par
\url{https://github.com/T-F-S/didec}
}
\end{center}
\bigskip
\begin{absquote}
  \begin{center}\bfseries Abstract\end{center}
  The \texttt{didec} package supports fixed-point arithmetic with two
  decimal places (\emph{di-decimal}) which is typical for financial transactions
  in many currencies. The intended use case is (personal) bookkeeping.
\end{absquote}

\tableofcontents

%%%%%%%%%%%%%%%%%%%%%%%%%%%%%%%%%%%%%%%%%%%%%%%%%%%%%%%%%%%%%%%%%%%%%%%%%%%%%%%%
\clearpage
\section{Quick start}%
For the impatient: this package provides fixed-point arithmetic with
two decimal places. You may use it for any purpose where exactly two decimal
places are needed or suffice, but the main application case is (personal) bookkeeping.

Say, John wants to keep track about his money. With
\begin{dispListing}
\didecnew{John}
\end{dispListing}
\didecnew{John}
a so-called \emph{didec variable} is created to store currency values.

Now, lets fill in some money:
\begin{dispListing}
\didecset{John}{1000}
\end{dispListing}
\didecset{John}{1000}

How much money has John now?
\begin{dispExample}
\didecuse{John}
\end{dispExample}

Obviously, John has Euros and the amount is displayed in a German style manor.
Of course, this can be adapted to your liking, see \refCom{didecsetup}.
But now, let us spend some money.
\begin{dispExample}
\didecsub{John}{19.75}
\didecuse{John}
\end{dispExample}

Preferably, the package functions are used inside convenient user commands.
You can choose between \LaTeX2e and \LaTeX3 programming layer functions.

For this quick start, we make some convenience commands for John.
|\transaction| shall be one purchase made in cash and |\transaction*| one made
by credit card.
\begin{dispExample*}{breakable}
%\usepackage{booktabs}
\didecsetup{english,
  currency          = {\pounds}{},
  currency-negative = {-\pounds}{},
}

%\didecnew{John}
\didecnew{cash}
\didecnew{credit}
\didecnew{transaction}

% Let's keep record in a table
\NewDocumentEnvironment{householdbook}{}{%
    \begin{center}
    \begin{tabular}{lp{9cm}rr}\toprule%
    & Transaction & Expenses & Budget\\\midrule
  }{%
    \midrule%
    \multicolumn{3}{r}{Cash} & \dideccoluse{cash}\\%
    \multicolumn{3}{r}{Credit card} & \dideccoluse{credit}\\\bottomrule%
    \end{tabular}%
    \end{center}
  }

% One transaction
\NewDocumentCommand \transaction{ s m m }
  {
    \didecset{transaction}{#3}%
    \didecsub{John}{transaction}%
    \IfBooleanTF {#1}%
      {\didecsub{credit}{transaction}CC}%
      {\didecsub{cash}{transaction}}%
    & #2
    & \dideccolinvuse{transaction}
    & \dideccoluse{John}\\%
  }

% Ready to start our tiny accountancy

\didecset{John}{1000}        % John's money
\didecsetequal{cash}{John}   % in cash
\didecset{credit}{0}         % blank credit card

\begin{householdbook}
  \transaction{Coffee break with snack}{19.75}
  \transaction*{Refuel}{62.87}
  \transaction{Gift from Aunt Mary for helping her}{-30}
  \transaction{Parking meter}{4.50}
  \transaction*{Shopping for weekend}{147.23}
  \transaction*{Fancy thing on the Internet}{270}
  \transaction*{Cinema}{17.70}
\end{householdbook}

\end{dispExample*}




%%%%%%%%%%%%%%%%%%%%%%%%%%%%%%%%%%%%%%%%%%%%%%%%%%%%%%%%%%%%%%%%%%%%%%%%%%%%%%%%
\clearpage
\section{Didec kernels and didec expressions}%

All calculations are done on cent basis as integer operations, but all
displayed figures have two decimal places which give the name for the package
(\emph{di}-\emph{dec}imal).

The package provides two numerical kernels which can be selected mutually
by package options \refKey{didec/int} or \refKey{didec/fp}.
The \refKey{didec/int} kernel is faster, but provides a smaller number range,
while the \refKey{didec/int} kernel is slower with a larger number range.
For ordinary people like you and me, the \refKey{didec/int} kernel will suffice
to do all personal financial calculations.
Upgrading later from \refKey{didec/int} to \refKey{didec/fp} is a matter of
just switching the package option setting.

%-------------------------------------------------------------------------------
\begin{docPackageOptions}{{
    doc name = int,
    doc description = {no value, initially set},
  }}
Selects the \meta{int} (integer) based numerical kernel with up to
9.33\footnote{joke for the mathematicians} significant figures and fast computation.
The number range for valid figures $n$ is:
\begin{center}
$ -21\;474\;836.47 \le n \le 21\;474\;836.47 $
\end{center}
\begin{dispListing}
\usepackage[int]{didec}
\end{dispListing}
Using figures outside the valid range will result in \LaTeX\ errors complaining
about too large numbers.
\end{docPackageOptions}
%-------------------------------------------------------------------------------


%-------------------------------------------------------------------------------
\begin{docPackageOptions}{{
    doc name = fp,
    doc description = {no value, initially unset},
  }}
Selects the \meta{fp} (floating-point) based numerical kernel with up to
16 significant figures and somewhat slower computation.
The number range for valid figures $n$ is:
\begin{center}
$ -99\;999\;999\;999\;999.99 \le n \le 99\;999\;999\;999\;999.99 $
\end{center}
\begin{dispListing}
\usepackage[fp]{didec}
\end{dispListing}
Using figures outside the valid range will result in \emph{silent calculation errors},
because \LaTeX3 \meta{fp} can use much larger numbers but is restricted to 16
significant figures.
\end{docPackageOptions}
%-------------------------------------------------------------------------------

\medskip

In the following, a \meta{didec expr} (didec expression) denotes one of
the following:
\begin{itemize}
\item a number in floating-point notation, e.g. |123.45|
\item a number in floating-comma notation, e.g. |123,45|
\item a \meta{didec var} (didec variable), e.g. |expenses|.
\end{itemize}
Note that the notation \meta{didec expr} is inspired by \meta{int expr} and \meta{fp expr}
from \LaTeX3 but is in comparison very restricted and allows only the three choices above.
A \meta{didec expr} is always expanded and spaces are trimmed.

\medskip

Many provided commands or functions of the package come in three flavors, for
example:
\begin{itemize}
\item \refCom{didecadd}: This is a user command where arguments are space trimmed
  and some check on variable existence is done.
  Not existing variables are reported by speaking error messages (not in all cases!).
\item \refCom{didec_gadd_check:nn}: This a programming layer function
  with no space trimming for arguments, but some check on variable existence is done.
  Not existing variables are reported by speaking error messages (not in all cases!).
\item \refCom{didec_gadd:nn}: This a programming layer function
  with no space trimming for arguments and no check on variable existence.
  Not existing variables give strange errors.
  This is the fasted function and base of the others above.
\end{itemize}


\clearpage
The following tables compare the computation time for selected functions of
the package for the two numerical kernels. Time values will differ on other
computers and also depend on selected values for the examples calculations.
Nevertheless, you get an impression of the differences.


%-------------------------------------------------------------------------------
\begin{tcbverbatimwrite}{\jobname.temp}
\begin{tcblisting}{
  blankest,
  pdf comment,
  comment style={raster columns=1,blankest,enhanced/.style=},
  comment only,
  freeze pdf,
  compilable listing,
  run pdflatex
}
\IfFileExists{didec.doc.cfg}{\input{didec.doc.cfg}}{\def\didecpackageprefix{}}
\documentclass[a4paper,11pt]{ltxdoc}
\usepackage[a4paper,textwidth=8cm]{geometry}
\usepackage{lmodern,array,booktabs,incgraph}
\usepackage[skins,documentation]{tcolorbox}
\input{didec.tempinc}
\usepackage{l3benchmark,siunitx}
\begin{document}
\begin{inctext}
\begin{tcolorbox}[blankest]
\makeatletter
\ExplSyntaxOn
\cs_new_protected:Npn \didec_benchmark:n #1
{
  \benchmark:n { #1 }
  \tablenum[table-format=4.0]{ \fp_to_decimal:n { round( \g_benchmark_time_fp * 1 000 000 ) } }\,\unit{\micro\second}
}
\didecnew{A}\didecnew{B}
\fp_set:Nn \g_benchmark_duration_target_fp { 1 }
\begin{center}
\begin{tabular}{lr}\toprule
\multicolumn{2}{c}{
  kernel:~\texttt{\c_didec_kernel_str},~engine:~\texttt{\c_sys_engine_str}
}\\\midrule
\docAuxCommand*{didec_gset:nn} & \didec_benchmark:n{ \didec_gset:nn{A}{-9876,324} }
\\\docAuxCommand*{didec_gset_check:nn} & \didec_benchmark:n{ \didec_gset_check:nn{A}{-9876,324} }
\\\docAuxCommand*{didecset} & \didec_benchmark:n{ \didecset{A}{-9876,324} }
\\\midrule
\docAuxCommand*{didec_gset_eq:nn} & \didecset{B}{1234.56} \didec_benchmark:n{ \didec_gset_eq:nn{A}{B} }
\\\docAuxCommand*{didec_gset_eq_check:nn} & \didecset{B}{1234.56} \didec_benchmark:n{ \didec_gset_eq_check:nn{A}{B} }
\\\docAuxCommand*{didecsetequal} & \didecset{B}{1234.56} \didec_benchmark:n{ \didecsetequal{A}{B} }
\\\midrule
\docAuxCommand*{didec_gset_fp:nn} & \didec_benchmark:n{ \didec_gset_fp:nn{A}{51234.56} }
\\\docAuxCommand*{didec_gset_fp_check:nn} & \didec_benchmark:n{ \didec_gset_fp_check:nn{A}{51234.56} }
\\\docAuxCommand*{didecsetfp} & \didec_benchmark:n{ \didecsetfp{A}{51234.56} }
\\\midrule
\docAuxCommand*{didec_gadd:nn} & \didecset{A}{-987.34} \didec_benchmark:n{ \didec_gadd:nn{A}{12345} }
\\\docAuxCommand*{didec_gadd_check:nn} &  \didecset{A}{-987.34}\didec_benchmark:n{ \didec_gadd_check:nn{A}{12345} }
\\\docAuxCommand*{didecadd} &  \didecset{A}{-987.34} \didec_benchmark:n{ \didecadd{A}{12345} }
\\\midrule
\docAuxCommand*{didec_gadd_to:nnn} & \didec_benchmark:n{ \didec_gadd_to:nnn{A}{-4564}{12345} }
\\\docAuxCommand*{didec_gadd_to_check:nnn} & \didec_benchmark:n{ \didec_gadd_to_check:nnn{A}{-4564}{12345} }
\\\docAuxCommand*{didecadd} & \didec_benchmark:n{ \didecadd[A]{-4564}{12345} }
\\\midrule
\docAuxCommand*{didec_gmul_fp:nn} & \didecset{A}{-987.34} \didec_benchmark:n{ \didec_gmul_fp:nn{A}{0.99} }
\\\docAuxCommand*{didec_gmul_fp_check:nn} &  \didecset{A}{-987.34}\didec_benchmark:n{ \didec_gmul_fp_check:nn{A}{0.99} }
\\\docAuxCommand*{didecmulfp} &  \didecset{A}{-987.34} \didec_benchmark:n{ \didecmulfp{A}{0.99} }
\\\midrule
%
\docAuxCommand*{didec_if_positive:nTF} & \didecset{A}{-987.34} \didec_benchmark:n{ \didec_if_positive:nTF{A}{}{} }
\\\docAuxCommand*{didecifpositive} & \didecset{A}{-987.34} \didec_benchmark:n{ \didecifpositive{A}{}{} }
\\\docAuxCommand*{didec_compare:nNnTF} & \didecset{A}{-987.34} \didec_benchmark:n{ \didec_compare:nNnTF{A}>{12345}{}{} }
\\\docAuxCommand*{didecifgreaterthan} & \didecset{A}{-987.34} \didec_benchmark:n{ \didecifgreaterthan{A}{12345}{}{} }
\\\midrule
\docAuxCommand*{didec_to_fp:n} & \didecset{A}{-1234987.34} \didec_benchmark:n{ \setbox\z@\hbox{\didec_to_fp:n{A}} }
\\\docAuxCommand*{didec_to_fp_check:n} & \didecset{A}{-1234987.34} \didec_benchmark:n{ \setbox\z@\hbox{\didec_to_fp_check:n{A}} }
\\\docAuxCommand*{didectofp} & \didecset{A}{-1234987.34} \didec_benchmark:n{ \setbox\z@\hbox{\didectofp{A}} }
\\\midrule
\docAuxCommand*{didec_use:n} & \didecset{A}{-1234987.34} \didec_benchmark:n{ \setbox\z@\hbox{\didec_use:n{A}} }
\\\docAuxCommand*{didec_use_check:n} & \didecset{A}{-1234987.34} \didec_benchmark:n{ \setbox\z@\hbox{\didec_use_check:n{A}} }
\\\docAuxCommand*{didecuse} & \didecset{A}{-1234987.34} \didec_benchmark:n{ \setbox\z@\hbox{\didecuse{A}} }
\\\docAuxCommand*{didecformat} & \didec_benchmark:n{ \setbox\z@\hbox{\didecformat{-1234987.34}} }
\\\docAuxCommand*{didec_color_use:n} & \didecset{A}{-1234987.34} \didec_benchmark:n{ \setbox\z@\hbox{\didec_color_use:n{A}} }
\\\docAuxCommand*{didec_color_use_check:n} & \didecset{A}{-1234987.34} \didec_benchmark:n{ \setbox\z@\hbox{\didec_color_use_check:n{A}} }
\\\docAuxCommand*{dideccoluse} & \didecset{A}{-1234987.34} \didec_benchmark:n{ \setbox\z@\hbox{\dideccoluse{A}} }
\\\docAuxCommand*{dideccolformat} & \didec_benchmark:n{ \setbox\z@\hbox{\dideccolformat{-1234987.34}} }
\\\bottomrule
\end{tabular}
\end{center}
\ExplSyntaxOff
\makeatother
\end{tcolorbox}
\end{inctext}
\end{document}
\end{tcblisting}
\end{tcbverbatimwrite}
%
\begin{tcbraster}[raster column skip=0pt]
\begin{tcbverbatimwrite}{\jobname.tempinc}
\usepackage[int]{\didecpackageprefix didec}
\end{tcbverbatimwrite}
\input{\jobname.temp}
%
\begin{tcbverbatimwrite}{\jobname.tempinc}
\usepackage[fp]{\didecpackageprefix didec}
\end{tcbverbatimwrite}
\input{\jobname.temp}
\end{tcbraster}
%-------------------------------------------------------------------------------


\clearpage
If needed, the selected kernel can be questioned by the following:

%-------------------------------------------------------------------------------
\begin{docCommands}[
    % doc parameter = \marg{didec var},
  ]
{
   { doc name = c_didec_kernel_str },
}
The current kernel given as a lower case string: one of
\docValue*{int} or \docValue*{fp}.
\begin{dispExample}
\ExplSyntaxOn
\c_didec_kernel_str
\ExplSyntaxOff
\end{dispExample}
\end{docCommands}
%-------------------------------------------------------------------------------



%-------------------------------------------------------------------------------
\begin{docCommands}[
    % doc parameter = \marg{didec var},
  ]
{
   { doc name = didec_if_kernel_int_p: },
   { doc name = didec_if_kernel_int:T, doc parameter = ~\marg{true code} },
   { doc name = didec_if_kernel_int:TF, doc parameter = ~\marg{true code}\marg{false code} },
   { doc name = didec_if_kernel_fp_p: },
   { doc name = didec_if_kernel_fp:T, doc parameter = ~\marg{true code} },
   { doc name = didec_if_kernel_fp:TF, doc parameter = ~\marg{true code}\marg{false code} },
}
Conditionals which allow kernel-specific code to be used. The names follow
naturally from those of the kernels.
\begin{dispExample}
\ExplSyntaxOn
\didec_if_kernel_int:T { Integer~kernel~used. }\par
Floating~point~kernel
\didec_if_kernel_fp:TF { ~used. }{ ~not~used. }
\ExplSyntaxOff
\end{dispExample}
\end{docCommands}
%-------------------------------------------------------------------------------





%%%%%%%%%%%%%%%%%%%%%%%%%%%%%%%%%%%%%%%%%%%%%%%%%%%%%%%%%%%%%%%%%%%%%%%%%%%%%%%%
\clearpage
\section{Creating didec variables}%

%-------------------------------------------------------------------------------
\begin{docCommands}[
    doc parameter = \marg{didec var},
  ]
{
  { doc name = didecnew },
  { doc name = didec_new:n },
 }
Creates a new \meta{didec var} or raises an error if the name is already taken.
\refCom{didecnew} trims spaces while \refCom{didec_new:n} does not.
\begin{dispExample}
\didecnew{konto}
\didecset{konto}{99.75}
\didecuse{konto}
\end{dispExample}
\end{docCommands}
%-------------------------------------------------------------------------------


%\clearpage
%%%%%%%%%%%%%%%%%%%%%%%%%%%%%%%%%%%%%%%%%%%%%%%%%%%%%%%%%%%%%%%%%%%%%%%%%%%%%%%%
\section{Setting didec variables}%


%-------------------------------------------------------------------------------
\begin{docCommands}[
    doc parameter = \marg{didec var}\marg{didec expr},
  ]
{
  { doc name = didecset },
  { doc name = didec_gset:nn },
  { doc name = didec_gset_check:nn },
 }
Sets \meta{didec var} to the value of \meta{didec expr} which can be
\begin{itemize}
\item a number in floating-point notation,
\item a number in floating-comma notation,
\item another \meta{didec var}.
\end{itemize}
\refCom{didecset} trims spaces and performs an existence check for \meta{didec var}.
\refCom{didec_gset_check:nn} performs an existence check for \meta{didec var}.
Decimals places 3 and beyond are cut not rounded.
If rounding is an issue, use \refCom{didecsetfp} instead.
\begin{dispExample}
\didecset{A}{1234.56}
\didecuse{A}

\didecset{A}{2345,6789}
\didecuse{A}

\didecset{B}{-3500}
\didecset{A}{B}
\didecuse{A}
\end{dispExample}
\end{docCommands}
%-------------------------------------------------------------------------------



%-------------------------------------------------------------------------------
\begin{docCommands}[
    doc parameter = \marg{didec var\textsubscript{1}}\marg{didec var\textsubscript{2}},
  ]
{
  { doc name = didecsetequal },
  { doc name = didec_gset_eq:nn },
  { doc name = didec_gset_eq_check:nn },
 }
Sets the \meta{didec var\textsubscript{1}} to the current value of \meta{didec var\textsubscript{2}}
\begin{dispExample}
\didecset{A}{1234.56}
\didecsetequal{B}{A}
\didecuse{A}
\didecuse{B}
\end{dispExample}
\end{docCommands}
%-------------------------------------------------------------------------------


\clearpage
%-------------------------------------------------------------------------------
\begin{docCommands}[
    doc parameter = \marg{didec var}\marg{didec expr},
  ]
{
  { doc name = didecsetnegative },
  { doc name = didec_gset_negative:nn },
  { doc name = didec_gset_negative_check:nn },
 }
Sets \meta{didec var} to the negated (opposite) value of \meta{didec expr}.
\refCom{didecsetnegative} trims spaces and performs an existence check for \meta{didec var}.
\begin{dispExample}
\didecsetnegative{A}{1234.56}
\didecuse{A}

\didecsetnegative{A}{-42.55}
\didecuse{A}

\didecset{B}{-3500}
\didecsetnegative{A}{B}
\didecuse{A}
\end{dispExample}
\end{docCommands}
%-------------------------------------------------------------------------------



%-------------------------------------------------------------------------------
\begin{docCommands}[
    doc parameter = \marg{didec var}\marg{fp expr},
  ]
{
  { doc name = didecsetfp },
  { doc name = didec_gset_fp:nn },
  { doc name = didec_gset_fp_check:nn },
 }
Sets \meta{didec var} to the value of \meta{fp expr} which can
by any \LaTeX3 floating-point expression.
\refCom{didecsetfp} trims spaces and performs an existence check for \meta{didec var}.
Other didec variables can be used inside \meta{fp expr} if guarded with \refCom{didectofp}.
The result is rounded to 2 decimal places.
\refCom{didec_gset_fp_check:nn} performs an existence check for \meta{didec var}.
\begin{dispExample}
\didecsetfp{A}{2345.6789}
\didecuse{A}

\didecsetfp{A}{ ln( 12345678 ) }
\didecuse{A}

\didecset{A}{123456,78}
\didecsetfp{B}{ \didectofp{A} * 2.35 / 100 }
2.35\% of \didecuse{A} are \didecuse{B}.
\end{dispExample}
\end{docCommands}
%-------------------------------------------------------------------------------


\clearpage
%-------------------------------------------------------------------------------
\begin{docCommands}
{
  { doc name = didecadd,                doc parameter = \oarg{didec var}\marg{didec expr\textsubscript{1}}\marg{didec expr\textsubscript{2}} },
  { doc name = didec_gadd:nn,     doc parameter = \marg{didec var}\marg{didec expr} },
  { doc name = didec_gadd_check:nn,     doc parameter = \marg{didec var}\marg{didec expr} },
  { doc name = didec_gadd_to:nnn, doc parameter = \marg{didec var}\marg{didec expr\textsubscript{1}}\marg{didec expr\textsubscript{2}} },
  { doc name = didec_gadd_to_check:nnn, doc parameter = \marg{didec var}\marg{didec expr\textsubscript{1}}\marg{didec expr\textsubscript{2}} },
 }
\begin{itemize}
\item Adds the result of computing the \meta{didec expr} to the \meta{didec var}
\item or sets \meta{didec var} to the sum of \meta{didec expr\textsubscript{1}} and \meta{didec expr\textsubscript{2}}.
\item For \refCom{didecadd}, if the optional \meta{didec var} is not available,
  the sum is stored into \meta{didec expr\textsubscript{1}} which has to be a \meta{didec var}
  in this case.
\end{itemize}
\begin{dispExample}
\didecset{A}{123}
\didecset{B}{5,88}
\didecadd{A}{B}
\didecuse{A}

\didecadd[A]{B}{1000}
\didecuse{A}

\didecadd{A}{-2750}
\didecuse{A}
\end{dispExample}
\end{docCommands}
%-------------------------------------------------------------------------------


%\clearpage
%-------------------------------------------------------------------------------
\begin{docCommands}
{
  { doc name = didecsub,                doc parameter = \oarg{didec var}\marg{didec expr\textsubscript{1}}\marg{didec expr\textsubscript{2}} },
  { doc name = didec_gsub:nn,     doc parameter = \marg{didec var}\marg{didec expr} },
  { doc name = didec_gsub_check:nn,     doc parameter = \marg{didec var}\marg{didec expr} },
  { doc name = didec_gsub_to:nnn, doc parameter = \marg{didec var}\marg{didec expr\textsubscript{1}}\marg{didec expr\textsubscript{2}} },
  { doc name = didec_gsub_to_check:nnn, doc parameter = \marg{didec var}\marg{didec expr\textsubscript{1}}\marg{didec expr\textsubscript{2}} },
 }
\begin{itemize}
\item Subtracts the result of computing the \meta{didec expr} to the \meta{didec var}
\item or sets \meta{didec var} to the difference of \meta{didec expr\textsubscript{1}} and \meta{didec expr\textsubscript{2}}.
\item For \refCom{didecsub}, if the optional \meta{didec var} is not available,
  the difference is stored into \meta{didec expr\textsubscript{1}} which has to be a \meta{didec var}
  in this case.
\end{itemize}
\begin{dispExample}
\didecset{A}{123}
\didecset{B}{5,88}
\didecsub{A}{B}
\didecuse{A}

\didecsub[A]{B}{1000}
\didecuse{A}

\didecsub{A}{-2750}
\didecuse{A}
\end{dispExample}
\end{docCommands}
%-------------------------------------------------------------------------------


\clearpage
%-------------------------------------------------------------------------------
\begin{docCommands}
{
  { doc name = didecmulfp,                doc parameter = \oarg{didec var\textsubscript{2}}\marg{didec var}\marg{fp expr} },
  { doc name = didec_gmul_fp:nn,     doc parameter = \marg{didec var}\marg{fp expr} },
  { doc name = didec_gmul_fp_check:nn,     doc parameter = \marg{didec var}\marg{fp expr} },
  { doc name = didec_gmul_fp_to:nnn, doc parameter = \marg{didec var\textsubscript{2}}\marg{didec var}\marg{fp expr} },
  { doc name = didec_gmul_fp_to_check:nnn, doc parameter = \marg{didec var\textsubscript{2}}\marg{didec var}\marg{fp expr} },
 }
\begin{itemize}
\item Multiplies \meta{didec var} with the result of computing the \meta{fp expr}
\item and sets \meta{didec var} or respectively \meta{didec var\textsubscript{2}} to the result.
\end{itemize}
\begin{dispExample}
\didecset{A}{123}
\didecmulfp{A}{0.9675}
\didecuse{A}

\didecmulfp[B]{A}{ln(42)}
\didecuse{B}
\end{dispExample}
\end{docCommands}
%-------------------------------------------------------------------------------



%-------------------------------------------------------------------------------
\begin{docCommands}
{
  { doc name = didecdivfp,                doc parameter = \oarg{didec var\textsubscript{2}}\marg{didec var}\marg{fp expr} },
  { doc name = didec_gdiv_fp:nn,     doc parameter = \marg{didec var}\marg{fp expr} },
  { doc name = didec_gdiv_fp_check:nn,     doc parameter = \marg{didec var}\marg{fp expr} },
  { doc name = didec_gdiv_fp_to:nnn, doc parameter = \marg{didec var\textsubscript{2}}\marg{didec var}\marg{fp expr} },
  { doc name = didec_gdiv_fp_to_check:nnn, doc parameter = \marg{didec var\textsubscript{2}}\marg{didec var}\marg{fp expr} },
 }
\begin{itemize}
\item Divides \meta{didec var} by the result of computing the \meta{fp expr}
\item and sets \meta{didec var} or respectively \meta{didec var\textsubscript{2}} to the result.
\end{itemize}
\begin{dispExample}
\didecset{A}{123}
\didecdivfp{A}{0.9675}
\didecuse{A}

\didecdivfp[B]{A}{ln(42)}
\didecuse{B}
\end{dispExample}
\end{docCommands}
%-------------------------------------------------------------------------------


%-------------------------------------------------------------------------------
\begin{docCommands}
{
  { doc name = didecsetsum, doc parameter = \oarg{didec var}\marg{sum of didec exp} },
}
Sets \meta{didec var} to the result of computing the given \meta{sum of didec exp}.\par
Here, \meta{sum of didec exp} =  \meta{didec exp\textsubscript{1}}
+ \meta{didec exp\textsubscript{2}} + \ldots + \meta{didec exp\textsubscript{n}}
\begin{dispExample}
\didecset{A}{123}
\didecset{B}{-32.15}
\didecsetsum{A}{ A + B + -22.5 }
\didecuse{A}
\end{dispExample}
\end{docCommands}
%-------------------------------------------------------------------------------



\clearpage
%%%%%%%%%%%%%%%%%%%%%%%%%%%%%%%%%%%%%%%%%%%%%%%%%%%%%%%%%%%%%%%%%%%%%%%%%%%%%%%%
\section{Using didecs}%


%-------------------------------------------------------------------------------
\begin{docCommands}
{
  { doc name = didectoint, doc parameter = \marg{didec var} },
  { doc name = didec_to_int:n, doc parameter = \marg{didec var} },
  { doc name = didec_to_int_check:n, doc parameter = \marg{didec var} },
}
Expresses the \meta{didec var} as Cent integer value, i.e. 100 times the
value.
All functions are expandable.
\begin{dispExample}
\didecset{A}{27123.45}
\didectoint{A}

\didecset{A}{-17}
\didectoint{A}
\end{dispExample}
\end{docCommands}
%-------------------------------------------------------------------------------



%-------------------------------------------------------------------------------
\begin{docCommands}
{
  { doc name = didectofp, doc parameter = \marg{didec var} },
  { doc name = didec_to_fp:n, doc parameter = \marg{didec var} },
  { doc name = didec_to_fp_check:n, doc parameter = \marg{didec var} },
}
Expresses the \meta{didec var} as floating-point value.
All functions are expandable.
\begin{dispExample}
\didecset{A}{27123.45}
\didectofp{A}

\didecset{A}{-17}
\didectofp{A}
\end{dispExample}
\end{docCommands}
%-------------------------------------------------------------------------------



%-------------------------------------------------------------------------------
\begin{docCommands}
{
  { doc name = didectofc, doc parameter = \marg{didec var} },
  { doc name = didec_to_fc:n, doc parameter = \marg{didec var} },
  { doc name = didec_to_fc_check:n, doc parameter = \marg{didec var} },
}
Expresses the \meta{didec var} as floating-comma value.
All functions are expandable.
\begin{dispExample}
\didecset{A}{27123.45}
\didectofc{A}

\didecset{A}{-17}
\didectofc{A}
\end{dispExample}
\end{docCommands}
%-------------------------------------------------------------------------------


\clearpage
%-------------------------------------------------------------------------------
\begin{docCommands}
{
  { doc name = didecuse, doc parameter = \oarg{key list}\marg{didec var} },
  { doc name = didec_use:n, doc parameter = \marg{didec var} },
  { doc name = didec_use_check:n, doc parameter = \marg{didec var} },
}
Expresses the \meta{didec var} as formatted value.
With \refCom{didecsetup}, the standard format can be set.
This standard format can be overwritten by \meta{key list}.
\begin{dispExample}
\didecset{A}{123456.78}
\didecuse{A}

\didecuse[
    currency           = {\pounds}{},
    decimal-separator  = {.},
    grouping-separator = {,},
  ]{A}
\end{dispExample}
\end{docCommands}
%-------------------------------------------------------------------------------



%-------------------------------------------------------------------------------
\begin{docCommands}
{
  { doc name = dideccoluse, doc parameter = \oarg{key list}\marg{didec var} },
  { doc name = didec_color_use:n, doc parameter = \marg{didec var} },
  { doc name = didec_color_use_check:n, doc parameter = \marg{didec var} },
}
Expresses the \meta{didec var} as colorized formatted value.
With \refCom{didecsetup}, the standard format can be set.
This standard format can be overwritten by \meta{key list}.
\begin{dispExample}
\didecset{A}{123456.78}
\dideccoluse{A}

\didecset{A}{-125}
\dideccoluse{A}

\dideccoluse[color-negative=didec-blue]{A}
\end{dispExample}
\end{docCommands}
%-------------------------------------------------------------------------------


%-------------------------------------------------------------------------------
\begin{docCommands}
{
  { doc name = dideccolinvuse, doc parameter = \oarg{key list}\marg{didec var} },
  { doc name = didec_color_inverse_use:n, doc parameter = \marg{didec var} },
  { doc name = didec_color_inverse_use_check:n, doc parameter = \marg{didec var} },
}
Expresses the \meta{didec var} as colorized formatted value.
The coloring is switched between positive and negative.
The standard
coloring and format can be overwritten by \meta{key list}.
\begin{dispExample}
\didecset{A}{123456.78}
\dideccolinvuse{A}

\didecset{A}{-125}
\dideccolinvuse{A}
\end{dispExample}
\end{docCommands}
%-------------------------------------------------------------------------------


\clearpage
%-------------------------------------------------------------------------------
\begin{docCommands}
{
  { doc name = didecformat, doc parameter = \oarg{key list}\marg{didec expr} },
  { doc name = dideccolformat, doc parameter = \oarg{key list}\marg{didec expr} },
  { doc name = dideccolinvformat, doc parameter = \oarg{key list}\marg{didec expr} },
}
Like \refCom{didecuse}, \refCom{dideccoluse}, \refCom{dideccolinvuse},
but accepts a \meta{didec expr} instead of a \meta{didec var}.
If the \meta{didec expr} is a \meta{didec var},
\refCom{didecuse}, \refCom{dideccoluse}, \refCom{dideccolinvuse} are more efficient.
\begin{dispExample}
\didecformat{123456.78}\\
\didecformat{A}\\
\didecformat[english]{123456.78}\\
\dideccolformat{123456.78}\\
\dideccolinvformat{123456.78}
\end{dispExample}
\end{docCommands}
%-------------------------------------------------------------------------------


%-------------------------------------------------------------------------------
\begin{docCommands}
{
  { doc name = didecsetup, doc parameter = \marg{key list} },
}
Sets all keys of the given \meta{key list}. See the following documentation
for available settings.
\begin{dispExample}
\didecsetup{
    currency           = {\pounds}{},
    decimal-separator  = {.},
    grouping-separator = {,},
}
\didecset{A}{123456.78}
\didecuse{A}
\end{dispExample}
\end{docCommands}
%-------------------------------------------------------------------------------


\clearpage
%-------------------------------------------------------------------------------
\begin{docDidecKeys}
{
  {
    doc name = decimal-separator,
    doc parameter = {=\marg{separator}},
    doc description = {initially |,|},
  },
}
Sets some \meta{separator} as decimal separator.
\begin{dispExample}
\didecset{A}{123456.78}
\didecuse[ decimal-separator={\#} ]{A} \par
\end{dispExample}
\end{docDidecKeys}
%-------------------------------------------------------------------------------



%-------------------------------------------------------------------------------
\begin{docDidecKeys}
{
  {
    doc name = grouping-separator,
    doc parameter = {=\marg{separator}},
    doc description = {initially |.|},
  },
}
Sets some \meta{separator} as grouping separator.
\begin{dispExample}
\didecset{A}{123456.78}
\didecuse[ grouping-separator={'} ]{A}
\end{dispExample}
\end{docDidecKeys}
%-------------------------------------------------------------------------------



%-------------------------------------------------------------------------------
\begin{docDidecKeys}
{
  {
    doc name = currency,
    doc parameter = {=\marg{prefix}\marg{postfix}},
    doc description = {initially empty},
  },
}
Sets some \meta{prefix} and \meta{postfix} to denote the currency of the
didec variable.
This also sets\\
\refKey{didec/currency-negative}\brackets{\meta{prefix}-}\marg{postfix}
\begin{dispExample}
\didecset{A}{123456.78}
\didecuse[ currency = {\pounds}{} ]{A} \par
\didecuse[ currency = {}{\:Gulden} ]{A}
\end{dispExample}
\end{docDidecKeys}
%-------------------------------------------------------------------------------



%-------------------------------------------------------------------------------
\begin{docDidecKeys}
{
  {
    doc name = currency-negative,
    doc parameter = {=\marg{prefix}\marg{postfix}},
    doc description = {initially empty},
  },
}
Sets some \meta{prefix} and \meta{postfix} to denote the currency of the
didec variable, if the resulting value is negative. Otherwise, the settings of
\refKey{didec/currency} are used.
Note that you need to set a minus sign |-| explicitly, if you want to see it.
Also note that setting \refKey{didec/currency} overwrites values given
by \refKey{didec/currency-negative}.
\begin{dispExample}
\didecset{A}{-123456.78}
\didecuse{A}\par
\didecuse[ currency-negative = {$-$}{\;€} ]{A} \par
\didecuse[ currency-negative = {(}{\;€)} ]{A} \par
\didecuse[ currency-negative = {(}{)\;€} ]{A} \par
\didecuse[ currency-negative = {-€}{} ]{A} \par
\end{dispExample}
\end{docDidecKeys}
%-------------------------------------------------------------------------------


\clearpage
%-------------------------------------------------------------------------------
\begin{docDidecKeys}[
    doc parameter   = ,
    doc description = style,
]
{
  { doc name = german },
  { doc name = english },
  { doc name = french },
  { doc name = float },
}
Styles to set some format preferences combined.
Any currency settings are removed and should be applied afterwards,
if needed.
Note that \refCom{didectofp} is more efficient than \refCom{didecuse} with
style \refKey{didec/float}.
\begin{center}
\begin{tabular}{>{\ttfamily}l>{\ttfamily}c>{\ttfamily}c>{\ttfamily}c>{\ttfamily}c}\toprule
                   & german & english & french & float\\\midrule
decimal-separator  &   ,    &    .    &    ,   &   .  \\ 
grouping-separator &   .    &    ,    &   |\;|   &      \\\bottomrule
\end{tabular}
\end{center}
\begin{dispExample}
\didecset{A}{12345678.90}
\didecuse[german]{A}  \par
\didecuse[english]{A} \par
\didecuse[french]{A}  \par
\didecuse[float]{A}   \par
\end{dispExample}
\end{docDidecKeys}
%-------------------------------------------------------------------------------


%-------------------------------------------------------------------------------
\begin{docDidecKeys}
{
  {
    doc name = color-positive,
    doc parameter = {=\meta{positive color}},
    doc description = {initially |didec-green|},
  },
  {
    doc name = color-negative,
    doc parameter = {=\meta{negative color}},
    doc description = {initially |didec-red|},
  },
}
Sets \meta{positive color} to denote positive (and zero) values
and \meta{negative color} to denote negative values.
Any valid |l3color| \meta{color expression} can be used.
The package defines additional colors
\begin{itemize}
\item\docColorExample{didec-green}
\item\docColorExample{didec-red}
\item\docColorExample{didec-blue}
\end{itemize}
\begin{dispExample}
\didecset{A}{123456.78}
\dideccoluse[ color-positive = magenta ]{A}
\end{dispExample}
\end{docDidecKeys}
%-------------------------------------------------------------------------------




%-------------------------------------------------------------------------------
\begin{docCommands}
{
  { doc name = didecwrite, doc parameter = \marg{didec var}\marg{stream} },
  { doc name = didec_write:nn, doc parameter = \marg{didec var}\marg{stream} },
  { doc name = didec_write_check:nn, doc parameter = \marg{didec var}\marg{stream} },
}
Writes \refCom{didecset}\marg{didec var}\marg{current value} to the given
already opened output \meta{stream}.
\begin{dispListing}
\didecwrite{A}{output}
% writes to output:
% \didecset{A}{VALUE}
\end{dispListing}
\end{docCommands}
%-------------------------------------------------------------------------------


\clearpage
%%%%%%%%%%%%%%%%%%%%%%%%%%%%%%%%%%%%%%%%%%%%%%%%%%%%%%%%%%%%%%%%%%%%%%%%%%%%%%%%
\section{Didec conditionals}%
%-------------------------------------------------------------------------------
\begin{docCommands}[
    doc parameter = \marg{didec var}\marg{true code}\marg{false code},
  ]
{
  { doc name = didecifpositive },
  { doc name = didec_if_positive_p:n, doc parameter = \marg{didec var} },
  { doc name = didec_if_positive:nTF, },
  { doc name = didec_if_positive:nT, doc parameter = \marg{didec var}\marg{true code} },
  { doc name = didec_if_positive:nF, doc parameter = \marg{didec var}\marg{false code} },
 }
Evaluates the \meta{didec var}
and returns |true| or executes the \meta{true code} if the value is positive,
otherwise returns |false| or executes the \meta{false code}.
\begin{dispExample}
\didecset{A}{2799.50}
\didecuse{A} is \didecifpositive{A}{positive}{not positive}.

\didecset{B}{-584}
\didecuse{B} is \didecifpositive{B}{positive}{not positive}.

\didecset{A}{0}
\didecuse{A} is \didecifpositive{A}{positive}{not positive}.
\end{dispExample}
\end{docCommands}
%-------------------------------------------------------------------------------


%-------------------------------------------------------------------------------
\begin{docCommands}[
    doc parameter = \marg{didec var}\marg{true code}\marg{false code},
  ]
{
  { doc name = didecifnegative },
  { doc name = didec_if_negative_p:n, doc parameter = \marg{didec var} },
  { doc name = didec_if_negative:nTF, },
  { doc name = didec_if_negative:nT, doc parameter = \marg{didec var}\marg{true code} },
  { doc name = didec_if_negative:nF, doc parameter = \marg{didec var}\marg{false code} },
 }
Evaluates the \meta{didec var}
and returns |true| or executes the \meta{true code} if the value is negative,
otherwise returns |false| or executes the \meta{false code}.
\begin{dispExample}
\didecset{A}{2799.50}
\didecuse{A} is \didecifnegative{A}{negative}{not negative}.

\didecset{B}{-584}
\didecuse{B} is \didecifnegative{B}{negative}{not negative}.

\didecset{A}{0}
\didecuse{A} is \didecifnegative{A}{negative}{not negative}.
\end{dispExample}
\end{docCommands}
%-------------------------------------------------------------------------------


\clearpage
%-------------------------------------------------------------------------------
\begin{docCommands}[
    doc parameter = \marg{didec var}\marg{true code}\marg{false code},
  ]
{
  { doc name = didecifzero  },
  { doc name = didec_if_zero_p:n, doc parameter = \marg{didec var} },
  { doc name = didec_if_zero:nTF, },
  { doc name = didec_if_zero:nT, doc parameter = \marg{didec var}\marg{true code} },
  { doc name = didec_if_zero:nF, doc parameter = \marg{didec var}\marg{false code} },
 }
Evaluates the \meta{didec var}
and returns |true| or executes the \meta{true code} if the value is zero,
otherwise returns |false| or executes the \meta{false code}.
\begin{dispExample}
\didecset{A}{2799.50}
\didecuse{A} is \didecifzero{A}{zero}{not zero}.

\didecset{B}{-584}
\didecuse{B} is \didecifzero{B}{zero}{not zero}.

\didecset{A}{0}
\didecuse{A} is \didecifzero{A}{zero}{not zero}.
\end{dispExample}
\end{docCommands}
%-------------------------------------------------------------------------------






%-------------------------------------------------------------------------------
\begin{docCommands}[
    doc parameter = \marg{didec expr\textsubscript{1}}\marg{didec expr\textsubscript{2}}\marg{true code}\marg{false code},
  ]
{
  { doc name = dideciflowerthan },
  { doc name = didecifequal  },
  { doc name = didecifgreaterthan },
  { doc name = didec_compare_p:nNn, doc parameter = \marg{didec expr\textsubscript{1}}\meta{relation}\marg{didec expr\textsubscript{2}} },
  { doc name = didec_compare:nNnTF, doc parameter = \marg{didec expr\textsubscript{1}}\meta{relation}\marg{didec expr\textsubscript{2}}\marg{true code}\marg{false code}  },
  { doc name = didec_compare:nNnT, doc parameter = \marg{didec expr\textsubscript{1}}\meta{relation}\marg{didec expr\textsubscript{2}}\marg{true code} },
  { doc name = didec_compare:nNnF, doc parameter = \marg{didec expr\textsubscript{1}}\meta{relation}\marg{didec expr\textsubscript{2}}\marg{false code} },
 }
Compares the \meta{didec expr\textsubscript{1}} and the \meta{didec expr\textsubscript{2}},
and returns |true| or executes the \meta{true code} if the \meta{relation} (given by function respectively)
is obeyed, otherwise returns |false| or executes the \meta{false code}.
\begin{dispExample}
\didecset{A}{2799.50}
\didecset{B}{-584}
\didecuse{A} is \dideciflowerthan{A}{B}{lower}{not lower} than \didecuse{B}

\didecuse{A} is \didecifequal{A}{B}{equal}{not equal} to \didecuse{B}

\didecuse{A} is \didecifequal{A}{A}{equal}{not equal} to \didecuse{A}

\didecuse{A} is \didecifgreaterthan{A}{B}{greater}{not greater} than \didecuse{B}
\end{dispExample}
\end{docCommands}
%-------------------------------------------------------------------------------


\clearpage
%%%%%%%%%%%%%%%%%%%%%%%%%%%%%%%%%%%%%%%%%%%%%%%%%%%%%%%%%%%%%%%%%%%%%%%%%%%%%%%%
\section{Viewing didecs}%

%-------------------------------------------------------------------------------
\begin{docCommands}
{
  { doc name = didec_show:n, doc parameter = \marg{didec var} },
 }
Displays the content of \meta{didec var} in the terminal.
\begin{dispListing}
\didecset{A}{2799.50}
\ExplSyntaxOn
\didec_show:n{A}
\ExplSyntaxOff
\end{dispListing}
\end{docCommands}
%-------------------------------------------------------------------------------



%%%%%%%%%%%%%%%%%%%%%%%%%%%%%%%%%%%%%%%%%%%%%%%%%%%%%%%%%%%%%%%%%%%%%%%%%%%%%%%%
\clearpage
\phantomsection
\printindex

\end{document}

\usepackage{l3benchmark,siunitx}
\begin{document}
\begin{inctext}
\begin{tcolorbox}[blankest]
\makeatletter
\ExplSyntaxOn
\cs_new_protected:Npn \didec_benchmark:n #1
{
  \benchmark:n { #1 }
  \tablenum[table-format=4.0]{ \fp_to_decimal:n { round( \g_benchmark_time_fp * 1 000 000 ) } }\,\unit{\micro\second}
}
\didecnew{A}\didecnew{B}
\fp_set:Nn \g_benchmark_duration_target_fp { 1 }
\begin{center}
\begin{tabular}{lr}\toprule
\multicolumn{2}{c}{
  kernel:~\texttt{\c_didec_kernel_str},~engine:~\texttt{\c_sys_engine_str}
}\\\midrule
\docAuxCommand*{didec_gset:nn} & \didec_benchmark:n{ \didec_gset:nn{A}{-9876,324} }
\\\docAuxCommand*{didec_gset_check:nn} & \didec_benchmark:n{ \didec_gset_check:nn{A}{-9876,324} }
\\\docAuxCommand*{didecset} & \didec_benchmark:n{ \didecset{A}{-9876,324} }
\\\midrule
\docAuxCommand*{didec_gset_eq:nn} & \didecset{B}{1234.56} \didec_benchmark:n{ \didec_gset_eq:nn{A}{B} }
\\\docAuxCommand*{didec_gset_eq_check:nn} & \didecset{B}{1234.56} \didec_benchmark:n{ \didec_gset_eq_check:nn{A}{B} }
\\\docAuxCommand*{didecsetequal} & \didecset{B}{1234.56} \didec_benchmark:n{ \didecsetequal{A}{B} }
\\\midrule
\docAuxCommand*{didec_gset_fp:nn} & \didec_benchmark:n{ \didec_gset_fp:nn{A}{51234.56} }
\\\docAuxCommand*{didec_gset_fp_check:nn} & \didec_benchmark:n{ \didec_gset_fp_check:nn{A}{51234.56} }
\\\docAuxCommand*{didecsetfp} & \didec_benchmark:n{ \didecsetfp{A}{51234.56} }
\\\midrule
\docAuxCommand*{didec_gadd:nn} & \didecset{A}{-987.34} \didec_benchmark:n{ \didec_gadd:nn{A}{12345} }
\\\docAuxCommand*{didec_gadd_check:nn} &  \didecset{A}{-987.34}\didec_benchmark:n{ \didec_gadd_check:nn{A}{12345} }
\\\docAuxCommand*{didecadd} &  \didecset{A}{-987.34} \didec_benchmark:n{ \didecadd{A}{12345} }
\\\midrule
\docAuxCommand*{didec_gadd_to:nnn} & \didec_benchmark:n{ \didec_gadd_to:nnn{A}{-4564}{12345} }
\\\docAuxCommand*{didec_gadd_to_check:nnn} & \didec_benchmark:n{ \didec_gadd_to_check:nnn{A}{-4564}{12345} }
\\\docAuxCommand*{didecadd} & \didec_benchmark:n{ \didecadd[A]{-4564}{12345} }
\\\midrule
\docAuxCommand*{didec_gmul_fp:nn} & \didecset{A}{-987.34} \didec_benchmark:n{ \didec_gmul_fp:nn{A}{0.99} }
\\\docAuxCommand*{didec_gmul_fp_check:nn} &  \didecset{A}{-987.34}\didec_benchmark:n{ \didec_gmul_fp_check:nn{A}{0.99} }
\\\docAuxCommand*{didecmulfp} &  \didecset{A}{-987.34} \didec_benchmark:n{ \didecmulfp{A}{0.99} }
\\\midrule
%
\docAuxCommand*{didec_if_positive:nTF} & \didecset{A}{-987.34} \didec_benchmark:n{ \didec_if_positive:nTF{A}{}{} }
\\\docAuxCommand*{didecifpositive} & \didecset{A}{-987.34} \didec_benchmark:n{ \didecifpositive{A}{}{} }
\\\docAuxCommand*{didec_compare:nNnTF} & \didecset{A}{-987.34} \didec_benchmark:n{ \didec_compare:nNnTF{A}>{12345}{}{} }
\\\docAuxCommand*{didecifgreaterthan} & \didecset{A}{-987.34} \didec_benchmark:n{ \didecifgreaterthan{A}{12345}{}{} }
\\\midrule
\docAuxCommand*{didec_to_fp:n} & \didecset{A}{-1234987.34} \didec_benchmark:n{ \setbox\z@\hbox{\didec_to_fp:n{A}} }
\\\docAuxCommand*{didec_to_fp_check:n} & \didecset{A}{-1234987.34} \didec_benchmark:n{ \setbox\z@\hbox{\didec_to_fp_check:n{A}} }
\\\docAuxCommand*{didectofp} & \didecset{A}{-1234987.34} \didec_benchmark:n{ \setbox\z@\hbox{\didectofp{A}} }
\\\midrule
\docAuxCommand*{didec_use:n} & \didecset{A}{-1234987.34} \didec_benchmark:n{ \setbox\z@\hbox{\didec_use:n{A}} }
\\\docAuxCommand*{didec_use_check:n} & \didecset{A}{-1234987.34} \didec_benchmark:n{ \setbox\z@\hbox{\didec_use_check:n{A}} }
\\\docAuxCommand*{didecuse} & \didecset{A}{-1234987.34} \didec_benchmark:n{ \setbox\z@\hbox{\didecuse{A}} }
\\\docAuxCommand*{didecformat} & \didec_benchmark:n{ \setbox\z@\hbox{\didecformat{-1234987.34}} }
\\\docAuxCommand*{didec_color_use:n} & \didecset{A}{-1234987.34} \didec_benchmark:n{ \setbox\z@\hbox{\didec_color_use:n{A}} }
\\\docAuxCommand*{didec_color_use_check:n} & \didecset{A}{-1234987.34} \didec_benchmark:n{ \setbox\z@\hbox{\didec_color_use_check:n{A}} }
\\\docAuxCommand*{dideccoluse} & \didecset{A}{-1234987.34} \didec_benchmark:n{ \setbox\z@\hbox{\dideccoluse{A}} }
\\\docAuxCommand*{dideccolformat} & \didec_benchmark:n{ \setbox\z@\hbox{\dideccolformat{-1234987.34}} }
\\\bottomrule
\end{tabular}
\end{center}
\ExplSyntaxOff
\makeatother
\end{tcolorbox}
\end{inctext}
\end{document}
\end{tcblisting}
\end{tcbverbatimwrite}
%
\begin{tcbraster}[raster column skip=0pt]
\begin{tcbverbatimwrite}{\jobname.tempinc}
\usepackage[int]{\didecpackageprefix didec}
\end{tcbverbatimwrite}
\input{\jobname.temp}
%
\begin{tcbverbatimwrite}{\jobname.tempinc}
\usepackage[fp]{\didecpackageprefix didec}
\end{tcbverbatimwrite}
\input{\jobname.temp}
\end{tcbraster}
%-------------------------------------------------------------------------------


\clearpage
If needed, the selected kernel can be questioned by the following:

%-------------------------------------------------------------------------------
\begin{docCommands}[
    % doc parameter = \marg{didec var},
  ]
{
   { doc name = c_didec_kernel_str },
}
The current kernel given as a lower case string: one of
\docValue*{int} or \docValue*{fp}.
\begin{dispExample}
\ExplSyntaxOn
\c_didec_kernel_str
\ExplSyntaxOff
\end{dispExample}
\end{docCommands}
%-------------------------------------------------------------------------------



%-------------------------------------------------------------------------------
\begin{docCommands}[
    % doc parameter = \marg{didec var},
  ]
{
   { doc name = didec_if_kernel_int_p: },
   { doc name = didec_if_kernel_int:T, doc parameter = ~\marg{true code} },
   { doc name = didec_if_kernel_int:TF, doc parameter = ~\marg{true code}\marg{false code} },
   { doc name = didec_if_kernel_fp_p: },
   { doc name = didec_if_kernel_fp:T, doc parameter = ~\marg{true code} },
   { doc name = didec_if_kernel_fp:TF, doc parameter = ~\marg{true code}\marg{false code} },
}
Conditionals which allow kernel-specific code to be used. The names follow
naturally from those of the kernels.
\begin{dispExample}
\ExplSyntaxOn
\didec_if_kernel_int:T { Integer~kernel~used. }\par
Floating~point~kernel
\didec_if_kernel_fp:TF { ~used. }{ ~not~used. }
\ExplSyntaxOff
\end{dispExample}
\end{docCommands}
%-------------------------------------------------------------------------------





%%%%%%%%%%%%%%%%%%%%%%%%%%%%%%%%%%%%%%%%%%%%%%%%%%%%%%%%%%%%%%%%%%%%%%%%%%%%%%%%
\clearpage
\section{Creating didec variables}%

%-------------------------------------------------------------------------------
\begin{docCommands}[
    doc parameter = \marg{didec var},
  ]
{
  { doc name = didecnew },
  { doc name = didec_new:n },
 }
Creates a new \meta{didec var} or raises an error if the name is already taken.
\refCom{didecnew} trims spaces while \refCom{didec_new:n} does not.
\begin{dispExample}
\didecnew{konto}
\didecset{konto}{99.75}
\didecuse{konto}
\end{dispExample}
\end{docCommands}
%-------------------------------------------------------------------------------


%\clearpage
%%%%%%%%%%%%%%%%%%%%%%%%%%%%%%%%%%%%%%%%%%%%%%%%%%%%%%%%%%%%%%%%%%%%%%%%%%%%%%%%
\section{Setting didec variables}%


%-------------------------------------------------------------------------------
\begin{docCommands}[
    doc parameter = \marg{didec var}\marg{didec expr},
  ]
{
  { doc name = didecset },
  { doc name = didec_gset:nn },
  { doc name = didec_gset_check:nn },
 }
Sets \meta{didec var} to the value of \meta{didec expr} which can be
\begin{itemize}
\item a number in floating-point notation,
\item a number in floating-comma notation,
\item another \meta{didec var}.
\end{itemize}
\refCom{didecset} trims spaces and performs an existence check for \meta{didec var}.
\refCom{didec_gset_check:nn} performs an existence check for \meta{didec var}.
Decimals places 3 and beyond are cut not rounded.
If rounding is an issue, use \refCom{didecsetfp} instead.
\begin{dispExample}
\didecset{A}{1234.56}
\didecuse{A}

\didecset{A}{2345,6789}
\didecuse{A}

\didecset{B}{-3500}
\didecset{A}{B}
\didecuse{A}
\end{dispExample}
\end{docCommands}
%-------------------------------------------------------------------------------



%-------------------------------------------------------------------------------
\begin{docCommands}[
    doc parameter = \marg{didec var\textsubscript{1}}\marg{didec var\textsubscript{2}},
  ]
{
  { doc name = didecsetequal },
  { doc name = didec_gset_eq:nn },
  { doc name = didec_gset_eq_check:nn },
 }
Sets the \meta{didec var\textsubscript{1}} to the current value of \meta{didec var\textsubscript{2}}
\begin{dispExample}
\didecset{A}{1234.56}
\didecsetequal{B}{A}
\didecuse{A}
\didecuse{B}
\end{dispExample}
\end{docCommands}
%-------------------------------------------------------------------------------


\clearpage
%-------------------------------------------------------------------------------
\begin{docCommands}[
    doc parameter = \marg{didec var}\marg{didec expr},
  ]
{
  { doc name = didecsetnegative },
  { doc name = didec_gset_negative:nn },
  { doc name = didec_gset_negative_check:nn },
 }
Sets \meta{didec var} to the negated (opposite) value of \meta{didec expr}.
\refCom{didecsetnegative} trims spaces and performs an existence check for \meta{didec var}.
\begin{dispExample}
\didecsetnegative{A}{1234.56}
\didecuse{A}

\didecsetnegative{A}{-42.55}
\didecuse{A}

\didecset{B}{-3500}
\didecsetnegative{A}{B}
\didecuse{A}
\end{dispExample}
\end{docCommands}
%-------------------------------------------------------------------------------



%-------------------------------------------------------------------------------
\begin{docCommands}[
    doc parameter = \marg{didec var}\marg{fp expr},
  ]
{
  { doc name = didecsetfp },
  { doc name = didec_gset_fp:nn },
  { doc name = didec_gset_fp_check:nn },
 }
Sets \meta{didec var} to the value of \meta{fp expr} which can
by any \LaTeX3 floating-point expression.
\refCom{didecsetfp} trims spaces and performs an existence check for \meta{didec var}.
Other didec variables can be used inside \meta{fp expr} if guarded with \refCom{didectofp}.
The result is rounded to 2 decimal places.
\refCom{didec_gset_fp_check:nn} performs an existence check for \meta{didec var}.
\begin{dispExample}
\didecsetfp{A}{2345.6789}
\didecuse{A}

\didecsetfp{A}{ ln( 12345678 ) }
\didecuse{A}

\didecset{A}{123456,78}
\didecsetfp{B}{ \didectofp{A} * 2.35 / 100 }
2.35\% of \didecuse{A} are \didecuse{B}.
\end{dispExample}
\end{docCommands}
%-------------------------------------------------------------------------------


\clearpage
%-------------------------------------------------------------------------------
\begin{docCommands}
{
  { doc name = didecadd,                doc parameter = \oarg{didec var}\marg{didec expr\textsubscript{1}}\marg{didec expr\textsubscript{2}} },
  { doc name = didec_gadd:nn,     doc parameter = \marg{didec var}\marg{didec expr} },
  { doc name = didec_gadd_check:nn,     doc parameter = \marg{didec var}\marg{didec expr} },
  { doc name = didec_gadd_to:nnn, doc parameter = \marg{didec var}\marg{didec expr\textsubscript{1}}\marg{didec expr\textsubscript{2}} },
  { doc name = didec_gadd_to_check:nnn, doc parameter = \marg{didec var}\marg{didec expr\textsubscript{1}}\marg{didec expr\textsubscript{2}} },
 }
\begin{itemize}
\item Adds the result of computing the \meta{didec expr} to the \meta{didec var}
\item or sets \meta{didec var} to the sum of \meta{didec expr\textsubscript{1}} and \meta{didec expr\textsubscript{2}}.
\item For \refCom{didecadd}, if the optional \meta{didec var} is not available,
  the sum is stored into \meta{didec expr\textsubscript{1}} which has to be a \meta{didec var}
  in this case.
\end{itemize}
\begin{dispExample}
\didecset{A}{123}
\didecset{B}{5,88}
\didecadd{A}{B}
\didecuse{A}

\didecadd[A]{B}{1000}
\didecuse{A}

\didecadd{A}{-2750}
\didecuse{A}
\end{dispExample}
\end{docCommands}
%-------------------------------------------------------------------------------


%\clearpage
%-------------------------------------------------------------------------------
\begin{docCommands}
{
  { doc name = didecsub,                doc parameter = \oarg{didec var}\marg{didec expr\textsubscript{1}}\marg{didec expr\textsubscript{2}} },
  { doc name = didec_gsub:nn,     doc parameter = \marg{didec var}\marg{didec expr} },
  { doc name = didec_gsub_check:nn,     doc parameter = \marg{didec var}\marg{didec expr} },
  { doc name = didec_gsub_to:nnn, doc parameter = \marg{didec var}\marg{didec expr\textsubscript{1}}\marg{didec expr\textsubscript{2}} },
  { doc name = didec_gsub_to_check:nnn, doc parameter = \marg{didec var}\marg{didec expr\textsubscript{1}}\marg{didec expr\textsubscript{2}} },
 }
\begin{itemize}
\item Subtracts the result of computing the \meta{didec expr} to the \meta{didec var}
\item or sets \meta{didec var} to the difference of \meta{didec expr\textsubscript{1}} and \meta{didec expr\textsubscript{2}}.
\item For \refCom{didecsub}, if the optional \meta{didec var} is not available,
  the difference is stored into \meta{didec expr\textsubscript{1}} which has to be a \meta{didec var}
  in this case.
\end{itemize}
\begin{dispExample}
\didecset{A}{123}
\didecset{B}{5,88}
\didecsub{A}{B}
\didecuse{A}

\didecsub[A]{B}{1000}
\didecuse{A}

\didecsub{A}{-2750}
\didecuse{A}
\end{dispExample}
\end{docCommands}
%-------------------------------------------------------------------------------


\clearpage
%-------------------------------------------------------------------------------
\begin{docCommands}
{
  { doc name = didecmulfp,                doc parameter = \oarg{didec var\textsubscript{2}}\marg{didec var}\marg{fp expr} },
  { doc name = didec_gmul_fp:nn,     doc parameter = \marg{didec var}\marg{fp expr} },
  { doc name = didec_gmul_fp_check:nn,     doc parameter = \marg{didec var}\marg{fp expr} },
  { doc name = didec_gmul_fp_to:nnn, doc parameter = \marg{didec var\textsubscript{2}}\marg{didec var}\marg{fp expr} },
  { doc name = didec_gmul_fp_to_check:nnn, doc parameter = \marg{didec var\textsubscript{2}}\marg{didec var}\marg{fp expr} },
 }
\begin{itemize}
\item Multiplies \meta{didec var} with the result of computing the \meta{fp expr}
\item and sets \meta{didec var} or respectively \meta{didec var\textsubscript{2}} to the result.
\end{itemize}
\begin{dispExample}
\didecset{A}{123}
\didecmulfp{A}{0.9675}
\didecuse{A}

\didecmulfp[B]{A}{ln(42)}
\didecuse{B}
\end{dispExample}
\end{docCommands}
%-------------------------------------------------------------------------------



%-------------------------------------------------------------------------------
\begin{docCommands}
{
  { doc name = didecdivfp,                doc parameter = \oarg{didec var\textsubscript{2}}\marg{didec var}\marg{fp expr} },
  { doc name = didec_gdiv_fp:nn,     doc parameter = \marg{didec var}\marg{fp expr} },
  { doc name = didec_gdiv_fp_check:nn,     doc parameter = \marg{didec var}\marg{fp expr} },
  { doc name = didec_gdiv_fp_to:nnn, doc parameter = \marg{didec var\textsubscript{2}}\marg{didec var}\marg{fp expr} },
  { doc name = didec_gdiv_fp_to_check:nnn, doc parameter = \marg{didec var\textsubscript{2}}\marg{didec var}\marg{fp expr} },
 }
\begin{itemize}
\item Divides \meta{didec var} by the result of computing the \meta{fp expr}
\item and sets \meta{didec var} or respectively \meta{didec var\textsubscript{2}} to the result.
\end{itemize}
\begin{dispExample}
\didecset{A}{123}
\didecdivfp{A}{0.9675}
\didecuse{A}

\didecdivfp[B]{A}{ln(42)}
\didecuse{B}
\end{dispExample}
\end{docCommands}
%-------------------------------------------------------------------------------


%-------------------------------------------------------------------------------
\begin{docCommands}
{
  { doc name = didecsetsum, doc parameter = \oarg{didec var}\marg{sum of didec exp} },
}
Sets \meta{didec var} to the result of computing the given \meta{sum of didec exp}.\par
Here, \meta{sum of didec exp} =  \meta{didec exp\textsubscript{1}}
+ \meta{didec exp\textsubscript{2}} + \ldots + \meta{didec exp\textsubscript{n}}
\begin{dispExample}
\didecset{A}{123}
\didecset{B}{-32.15}
\didecsetsum{A}{ A + B + -22.5 }
\didecuse{A}
\end{dispExample}
\end{docCommands}
%-------------------------------------------------------------------------------



\clearpage
%%%%%%%%%%%%%%%%%%%%%%%%%%%%%%%%%%%%%%%%%%%%%%%%%%%%%%%%%%%%%%%%%%%%%%%%%%%%%%%%
\section{Using didecs}%


%-------------------------------------------------------------------------------
\begin{docCommands}
{
  { doc name = didectoint, doc parameter = \marg{didec var} },
  { doc name = didec_to_int:n, doc parameter = \marg{didec var} },
  { doc name = didec_to_int_check:n, doc parameter = \marg{didec var} },
}
Expresses the \meta{didec var} as Cent integer value, i.e. 100 times the
value.
All functions are expandable.
\begin{dispExample}
\didecset{A}{27123.45}
\didectoint{A}

\didecset{A}{-17}
\didectoint{A}
\end{dispExample}
\end{docCommands}
%-------------------------------------------------------------------------------



%-------------------------------------------------------------------------------
\begin{docCommands}
{
  { doc name = didectofp, doc parameter = \marg{didec var} },
  { doc name = didec_to_fp:n, doc parameter = \marg{didec var} },
  { doc name = didec_to_fp_check:n, doc parameter = \marg{didec var} },
}
Expresses the \meta{didec var} as floating-point value.
All functions are expandable.
\begin{dispExample}
\didecset{A}{27123.45}
\didectofp{A}

\didecset{A}{-17}
\didectofp{A}
\end{dispExample}
\end{docCommands}
%-------------------------------------------------------------------------------



%-------------------------------------------------------------------------------
\begin{docCommands}
{
  { doc name = didectofc, doc parameter = \marg{didec var} },
  { doc name = didec_to_fc:n, doc parameter = \marg{didec var} },
  { doc name = didec_to_fc_check:n, doc parameter = \marg{didec var} },
}
Expresses the \meta{didec var} as floating-comma value.
All functions are expandable.
\begin{dispExample}
\didecset{A}{27123.45}
\didectofc{A}

\didecset{A}{-17}
\didectofc{A}
\end{dispExample}
\end{docCommands}
%-------------------------------------------------------------------------------


\clearpage
%-------------------------------------------------------------------------------
\begin{docCommands}
{
  { doc name = didecuse, doc parameter = \oarg{key list}\marg{didec var} },
  { doc name = didec_use:n, doc parameter = \marg{didec var} },
  { doc name = didec_use_check:n, doc parameter = \marg{didec var} },
}
Expresses the \meta{didec var} as formatted value.
With \refCom{didecsetup}, the standard format can be set.
This standard format can be overwritten by \meta{key list}.
\begin{dispExample}
\didecset{A}{123456.78}
\didecuse{A}

\didecuse[
    currency           = {\pounds}{},
    decimal-separator  = {.},
    grouping-separator = {,},
  ]{A}
\end{dispExample}
\end{docCommands}
%-------------------------------------------------------------------------------



%-------------------------------------------------------------------------------
\begin{docCommands}
{
  { doc name = dideccoluse, doc parameter = \oarg{key list}\marg{didec var} },
  { doc name = didec_color_use:n, doc parameter = \marg{didec var} },
  { doc name = didec_color_use_check:n, doc parameter = \marg{didec var} },
}
Expresses the \meta{didec var} as colorized formatted value.
With \refCom{didecsetup}, the standard format can be set.
This standard format can be overwritten by \meta{key list}.
\begin{dispExample}
\didecset{A}{123456.78}
\dideccoluse{A}

\didecset{A}{-125}
\dideccoluse{A}

\dideccoluse[color-negative=didec-blue]{A}
\end{dispExample}
\end{docCommands}
%-------------------------------------------------------------------------------


%-------------------------------------------------------------------------------
\begin{docCommands}
{
  { doc name = dideccolinvuse, doc parameter = \oarg{key list}\marg{didec var} },
  { doc name = didec_color_inverse_use:n, doc parameter = \marg{didec var} },
  { doc name = didec_color_inverse_use_check:n, doc parameter = \marg{didec var} },
}
Expresses the \meta{didec var} as colorized formatted value.
The coloring is switched between positive and negative.
The standard
coloring and format can be overwritten by \meta{key list}.
\begin{dispExample}
\didecset{A}{123456.78}
\dideccolinvuse{A}

\didecset{A}{-125}
\dideccolinvuse{A}
\end{dispExample}
\end{docCommands}
%-------------------------------------------------------------------------------


\clearpage
%-------------------------------------------------------------------------------
\begin{docCommands}
{
  { doc name = didecformat, doc parameter = \oarg{key list}\marg{didec expr} },
  { doc name = dideccolformat, doc parameter = \oarg{key list}\marg{didec expr} },
  { doc name = dideccolinvformat, doc parameter = \oarg{key list}\marg{didec expr} },
}
Like \refCom{didecuse}, \refCom{dideccoluse}, \refCom{dideccolinvuse},
but accepts a \meta{didec expr} instead of a \meta{didec var}.
If the \meta{didec expr} is a \meta{didec var},
\refCom{didecuse}, \refCom{dideccoluse}, \refCom{dideccolinvuse} are more efficient.
\begin{dispExample}
\didecformat{123456.78}\\
\didecformat{A}\\
\didecformat[english]{123456.78}\\
\dideccolformat{123456.78}\\
\dideccolinvformat{123456.78}
\end{dispExample}
\end{docCommands}
%-------------------------------------------------------------------------------


%-------------------------------------------------------------------------------
\begin{docCommands}
{
  { doc name = didecsetup, doc parameter = \marg{key list} },
}
Sets all keys of the given \meta{key list}. See the following documentation
for available settings.
\begin{dispExample}
\didecsetup{
    currency           = {\pounds}{},
    decimal-separator  = {.},
    grouping-separator = {,},
}
\didecset{A}{123456.78}
\didecuse{A}
\end{dispExample}
\end{docCommands}
%-------------------------------------------------------------------------------


\clearpage
%-------------------------------------------------------------------------------
\begin{docDidecKeys}
{
  {
    doc name = decimal-separator,
    doc parameter = {=\marg{separator}},
    doc description = {initially |,|},
  },
}
Sets some \meta{separator} as decimal separator.
\begin{dispExample}
\didecset{A}{123456.78}
\didecuse[ decimal-separator={\#} ]{A} \par
\end{dispExample}
\end{docDidecKeys}
%-------------------------------------------------------------------------------



%-------------------------------------------------------------------------------
\begin{docDidecKeys}
{
  {
    doc name = grouping-separator,
    doc parameter = {=\marg{separator}},
    doc description = {initially |.|},
  },
}
Sets some \meta{separator} as grouping separator.
\begin{dispExample}
\didecset{A}{123456.78}
\didecuse[ grouping-separator={'} ]{A}
\end{dispExample}
\end{docDidecKeys}
%-------------------------------------------------------------------------------



%-------------------------------------------------------------------------------
\begin{docDidecKeys}
{
  {
    doc name = currency,
    doc parameter = {=\marg{prefix}\marg{postfix}},
    doc description = {initially empty},
  },
}
Sets some \meta{prefix} and \meta{postfix} to denote the currency of the
didec variable.
This also sets\\
\refKey{didec/currency-negative}\brackets{\meta{prefix}-}\marg{postfix}
\begin{dispExample}
\didecset{A}{123456.78}
\didecuse[ currency = {\pounds}{} ]{A} \par
\didecuse[ currency = {}{\:Gulden} ]{A}
\end{dispExample}
\end{docDidecKeys}
%-------------------------------------------------------------------------------



%-------------------------------------------------------------------------------
\begin{docDidecKeys}
{
  {
    doc name = currency-negative,
    doc parameter = {=\marg{prefix}\marg{postfix}},
    doc description = {initially empty},
  },
}
Sets some \meta{prefix} and \meta{postfix} to denote the currency of the
didec variable, if the resulting value is negative. Otherwise, the settings of
\refKey{didec/currency} are used.
Note that you need to set a minus sign |-| explicitly, if you want to see it.
Also note that setting \refKey{didec/currency} overwrites values given
by \refKey{didec/currency-negative}.
\begin{dispExample}
\didecset{A}{-123456.78}
\didecuse{A}\par
\didecuse[ currency-negative = {$-$}{\;€} ]{A} \par
\didecuse[ currency-negative = {(}{\;€)} ]{A} \par
\didecuse[ currency-negative = {(}{)\;€} ]{A} \par
\didecuse[ currency-negative = {-€}{} ]{A} \par
\end{dispExample}
\end{docDidecKeys}
%-------------------------------------------------------------------------------


\clearpage
%-------------------------------------------------------------------------------
\begin{docDidecKeys}[
    doc parameter   = ,
    doc description = style,
]
{
  { doc name = german },
  { doc name = english },
  { doc name = french },
  { doc name = float },
}
Styles to set some format preferences combined.
Any currency settings are removed and should be applied afterwards,
if needed.
Note that \refCom{didectofp} is more efficient than \refCom{didecuse} with
style \refKey{didec/float}.
\begin{center}
\begin{tabular}{>{\ttfamily}l>{\ttfamily}c>{\ttfamily}c>{\ttfamily}c>{\ttfamily}c}\toprule
                   & german & english & french & float\\\midrule
decimal-separator  &   ,    &    .    &    ,   &   .  \\ 
grouping-separator &   .    &    ,    &   |\;|   &      \\\bottomrule
\end{tabular}
\end{center}
\begin{dispExample}
\didecset{A}{12345678.90}
\didecuse[german]{A}  \par
\didecuse[english]{A} \par
\didecuse[french]{A}  \par
\didecuse[float]{A}   \par
\end{dispExample}
\end{docDidecKeys}
%-------------------------------------------------------------------------------


%-------------------------------------------------------------------------------
\begin{docDidecKeys}
{
  {
    doc name = color-positive,
    doc parameter = {=\meta{positive color}},
    doc description = {initially |didec-green|},
  },
  {
    doc name = color-negative,
    doc parameter = {=\meta{negative color}},
    doc description = {initially |didec-red|},
  },
}
Sets \meta{positive color} to denote positive (and zero) values
and \meta{negative color} to denote negative values.
Any valid |l3color| \meta{color expression} can be used.
The package defines additional colors
\begin{itemize}
\item\docColorExample{didec-green}
\item\docColorExample{didec-red}
\item\docColorExample{didec-blue}
\end{itemize}
\begin{dispExample}
\didecset{A}{123456.78}
\dideccoluse[ color-positive = magenta ]{A}
\end{dispExample}
\end{docDidecKeys}
%-------------------------------------------------------------------------------




%-------------------------------------------------------------------------------
\begin{docCommands}
{
  { doc name = didecwrite, doc parameter = \marg{didec var}\marg{stream} },
  { doc name = didec_write:nn, doc parameter = \marg{didec var}\marg{stream} },
  { doc name = didec_write_check:nn, doc parameter = \marg{didec var}\marg{stream} },
}
Writes \refCom{didecset}\marg{didec var}\marg{current value} to the given
already opened output \meta{stream}.
\begin{dispListing}
\didecwrite{A}{output}
% writes to output:
% \didecset{A}{VALUE}
\end{dispListing}
\end{docCommands}
%-------------------------------------------------------------------------------


\clearpage
%%%%%%%%%%%%%%%%%%%%%%%%%%%%%%%%%%%%%%%%%%%%%%%%%%%%%%%%%%%%%%%%%%%%%%%%%%%%%%%%
\section{Didec conditionals}%
%-------------------------------------------------------------------------------
\begin{docCommands}[
    doc parameter = \marg{didec var}\marg{true code}\marg{false code},
  ]
{
  { doc name = didecifpositive },
  { doc name = didec_if_positive_p:n, doc parameter = \marg{didec var} },
  { doc name = didec_if_positive:nTF, },
  { doc name = didec_if_positive:nT, doc parameter = \marg{didec var}\marg{true code} },
  { doc name = didec_if_positive:nF, doc parameter = \marg{didec var}\marg{false code} },
 }
Evaluates the \meta{didec var}
and returns |true| or executes the \meta{true code} if the value is positive,
otherwise returns |false| or executes the \meta{false code}.
\begin{dispExample}
\didecset{A}{2799.50}
\didecuse{A} is \didecifpositive{A}{positive}{not positive}.

\didecset{B}{-584}
\didecuse{B} is \didecifpositive{B}{positive}{not positive}.

\didecset{A}{0}
\didecuse{A} is \didecifpositive{A}{positive}{not positive}.
\end{dispExample}
\end{docCommands}
%-------------------------------------------------------------------------------


%-------------------------------------------------------------------------------
\begin{docCommands}[
    doc parameter = \marg{didec var}\marg{true code}\marg{false code},
  ]
{
  { doc name = didecifnegative },
  { doc name = didec_if_negative_p:n, doc parameter = \marg{didec var} },
  { doc name = didec_if_negative:nTF, },
  { doc name = didec_if_negative:nT, doc parameter = \marg{didec var}\marg{true code} },
  { doc name = didec_if_negative:nF, doc parameter = \marg{didec var}\marg{false code} },
 }
Evaluates the \meta{didec var}
and returns |true| or executes the \meta{true code} if the value is negative,
otherwise returns |false| or executes the \meta{false code}.
\begin{dispExample}
\didecset{A}{2799.50}
\didecuse{A} is \didecifnegative{A}{negative}{not negative}.

\didecset{B}{-584}
\didecuse{B} is \didecifnegative{B}{negative}{not negative}.

\didecset{A}{0}
\didecuse{A} is \didecifnegative{A}{negative}{not negative}.
\end{dispExample}
\end{docCommands}
%-------------------------------------------------------------------------------


\clearpage
%-------------------------------------------------------------------------------
\begin{docCommands}[
    doc parameter = \marg{didec var}\marg{true code}\marg{false code},
  ]
{
  { doc name = didecifzero  },
  { doc name = didec_if_zero_p:n, doc parameter = \marg{didec var} },
  { doc name = didec_if_zero:nTF, },
  { doc name = didec_if_zero:nT, doc parameter = \marg{didec var}\marg{true code} },
  { doc name = didec_if_zero:nF, doc parameter = \marg{didec var}\marg{false code} },
 }
Evaluates the \meta{didec var}
and returns |true| or executes the \meta{true code} if the value is zero,
otherwise returns |false| or executes the \meta{false code}.
\begin{dispExample}
\didecset{A}{2799.50}
\didecuse{A} is \didecifzero{A}{zero}{not zero}.

\didecset{B}{-584}
\didecuse{B} is \didecifzero{B}{zero}{not zero}.

\didecset{A}{0}
\didecuse{A} is \didecifzero{A}{zero}{not zero}.
\end{dispExample}
\end{docCommands}
%-------------------------------------------------------------------------------






%-------------------------------------------------------------------------------
\begin{docCommands}[
    doc parameter = \marg{didec expr\textsubscript{1}}\marg{didec expr\textsubscript{2}}\marg{true code}\marg{false code},
  ]
{
  { doc name = dideciflowerthan },
  { doc name = didecifequal  },
  { doc name = didecifgreaterthan },
  { doc name = didec_compare_p:nNn, doc parameter = \marg{didec expr\textsubscript{1}}\meta{relation}\marg{didec expr\textsubscript{2}} },
  { doc name = didec_compare:nNnTF, doc parameter = \marg{didec expr\textsubscript{1}}\meta{relation}\marg{didec expr\textsubscript{2}}\marg{true code}\marg{false code}  },
  { doc name = didec_compare:nNnT, doc parameter = \marg{didec expr\textsubscript{1}}\meta{relation}\marg{didec expr\textsubscript{2}}\marg{true code} },
  { doc name = didec_compare:nNnF, doc parameter = \marg{didec expr\textsubscript{1}}\meta{relation}\marg{didec expr\textsubscript{2}}\marg{false code} },
 }
Compares the \meta{didec expr\textsubscript{1}} and the \meta{didec expr\textsubscript{2}},
and returns |true| or executes the \meta{true code} if the \meta{relation} (given by function respectively)
is obeyed, otherwise returns |false| or executes the \meta{false code}.
\begin{dispExample}
\didecset{A}{2799.50}
\didecset{B}{-584}
\didecuse{A} is \dideciflowerthan{A}{B}{lower}{not lower} than \didecuse{B}

\didecuse{A} is \didecifequal{A}{B}{equal}{not equal} to \didecuse{B}

\didecuse{A} is \didecifequal{A}{A}{equal}{not equal} to \didecuse{A}

\didecuse{A} is \didecifgreaterthan{A}{B}{greater}{not greater} than \didecuse{B}
\end{dispExample}
\end{docCommands}
%-------------------------------------------------------------------------------


\clearpage
%%%%%%%%%%%%%%%%%%%%%%%%%%%%%%%%%%%%%%%%%%%%%%%%%%%%%%%%%%%%%%%%%%%%%%%%%%%%%%%%
\section{Viewing didecs}%

%-------------------------------------------------------------------------------
\begin{docCommands}
{
  { doc name = didec_show:n, doc parameter = \marg{didec var} },
 }
Displays the content of \meta{didec var} in the terminal.
\begin{dispListing}
\didecset{A}{2799.50}
\ExplSyntaxOn
\didec_show:n{A}
\ExplSyntaxOff
\end{dispListing}
\end{docCommands}
%-------------------------------------------------------------------------------



%%%%%%%%%%%%%%%%%%%%%%%%%%%%%%%%%%%%%%%%%%%%%%%%%%%%%%%%%%%%%%%%%%%%%%%%%%%%%%%%
\clearpage
\phantomsection
\printindex

\end{document}

\usepackage{l3benchmark,siunitx}
\begin{document}
\begin{inctext}
\begin{tcolorbox}[blankest]
\makeatletter
\ExplSyntaxOn
\cs_new_protected:Npn \didec_benchmark:n #1
{
  \benchmark:n { #1 }
  \tablenum[table-format=4.0]{ \fp_to_decimal:n { round( \g_benchmark_time_fp * 1 000 000 ) } }\,\unit{\micro\second}
}
\didecnew{A}\didecnew{B}
\fp_set:Nn \g_benchmark_duration_target_fp { 1 }
\begin{center}
\begin{tabular}{lr}\toprule
\multicolumn{2}{c}{
  kernel:~\texttt{\c_didec_kernel_str},~engine:~\texttt{\c_sys_engine_str}
}\\\midrule
\docAuxCommand*{didec_gset:nn} & \didec_benchmark:n{ \didec_gset:nn{A}{-9876,324} }
\\\docAuxCommand*{didec_gset_check:nn} & \didec_benchmark:n{ \didec_gset_check:nn{A}{-9876,324} }
\\\docAuxCommand*{didecset} & \didec_benchmark:n{ \didecset{A}{-9876,324} }
\\\midrule
\docAuxCommand*{didec_gset_eq:nn} & \didecset{B}{1234.56} \didec_benchmark:n{ \didec_gset_eq:nn{A}{B} }
\\\docAuxCommand*{didec_gset_eq_check:nn} & \didecset{B}{1234.56} \didec_benchmark:n{ \didec_gset_eq_check:nn{A}{B} }
\\\docAuxCommand*{didecsetequal} & \didecset{B}{1234.56} \didec_benchmark:n{ \didecsetequal{A}{B} }
\\\midrule
\docAuxCommand*{didec_gset_fp:nn} & \didec_benchmark:n{ \didec_gset_fp:nn{A}{51234.56} }
\\\docAuxCommand*{didec_gset_fp_check:nn} & \didec_benchmark:n{ \didec_gset_fp_check:nn{A}{51234.56} }
\\\docAuxCommand*{didecsetfp} & \didec_benchmark:n{ \didecsetfp{A}{51234.56} }
\\\midrule
\docAuxCommand*{didec_gadd:nn} & \didecset{A}{-987.34} \didec_benchmark:n{ \didec_gadd:nn{A}{12345} }
\\\docAuxCommand*{didec_gadd_check:nn} &  \didecset{A}{-987.34}\didec_benchmark:n{ \didec_gadd_check:nn{A}{12345} }
\\\docAuxCommand*{didecadd} &  \didecset{A}{-987.34} \didec_benchmark:n{ \didecadd{A}{12345} }
\\\midrule
\docAuxCommand*{didec_gadd_to:nnn} & \didec_benchmark:n{ \didec_gadd_to:nnn{A}{-4564}{12345} }
\\\docAuxCommand*{didec_gadd_to_check:nnn} & \didec_benchmark:n{ \didec_gadd_to_check:nnn{A}{-4564}{12345} }
\\\docAuxCommand*{didecadd} & \didec_benchmark:n{ \didecadd[A]{-4564}{12345} }
\\\midrule
\docAuxCommand*{didec_gmul_fp:nn} & \didecset{A}{-987.34} \didec_benchmark:n{ \didec_gmul_fp:nn{A}{0.99} }
\\\docAuxCommand*{didec_gmul_fp_check:nn} &  \didecset{A}{-987.34}\didec_benchmark:n{ \didec_gmul_fp_check:nn{A}{0.99} }
\\\docAuxCommand*{didecmulfp} &  \didecset{A}{-987.34} \didec_benchmark:n{ \didecmulfp{A}{0.99} }
\\\midrule
%
\docAuxCommand*{didec_if_positive:nTF} & \didecset{A}{-987.34} \didec_benchmark:n{ \didec_if_positive:nTF{A}{}{} }
\\\docAuxCommand*{didecifpositive} & \didecset{A}{-987.34} \didec_benchmark:n{ \didecifpositive{A}{}{} }
\\\docAuxCommand*{didec_compare:nNnTF} & \didecset{A}{-987.34} \didec_benchmark:n{ \didec_compare:nNnTF{A}>{12345}{}{} }
\\\docAuxCommand*{didecifgreaterthan} & \didecset{A}{-987.34} \didec_benchmark:n{ \didecifgreaterthan{A}{12345}{}{} }
\\\midrule
\docAuxCommand*{didec_to_fp:n} & \didecset{A}{-1234987.34} \didec_benchmark:n{ \setbox\z@\hbox{\didec_to_fp:n{A}} }
\\\docAuxCommand*{didec_to_fp_check:n} & \didecset{A}{-1234987.34} \didec_benchmark:n{ \setbox\z@\hbox{\didec_to_fp_check:n{A}} }
\\\docAuxCommand*{didectofp} & \didecset{A}{-1234987.34} \didec_benchmark:n{ \setbox\z@\hbox{\didectofp{A}} }
\\\midrule
\docAuxCommand*{didec_use:n} & \didecset{A}{-1234987.34} \didec_benchmark:n{ \setbox\z@\hbox{\didec_use:n{A}} }
\\\docAuxCommand*{didec_use_check:n} & \didecset{A}{-1234987.34} \didec_benchmark:n{ \setbox\z@\hbox{\didec_use_check:n{A}} }
\\\docAuxCommand*{didecuse} & \didecset{A}{-1234987.34} \didec_benchmark:n{ \setbox\z@\hbox{\didecuse{A}} }
\\\docAuxCommand*{didecformat} & \didec_benchmark:n{ \setbox\z@\hbox{\didecformat{-1234987.34}} }
\\\docAuxCommand*{didec_color_use:n} & \didecset{A}{-1234987.34} \didec_benchmark:n{ \setbox\z@\hbox{\didec_color_use:n{A}} }
\\\docAuxCommand*{didec_color_use_check:n} & \didecset{A}{-1234987.34} \didec_benchmark:n{ \setbox\z@\hbox{\didec_color_use_check:n{A}} }
\\\docAuxCommand*{dideccoluse} & \didecset{A}{-1234987.34} \didec_benchmark:n{ \setbox\z@\hbox{\dideccoluse{A}} }
\\\docAuxCommand*{dideccolformat} & \didec_benchmark:n{ \setbox\z@\hbox{\dideccolformat{-1234987.34}} }
\\\bottomrule
\end{tabular}
\end{center}
\ExplSyntaxOff
\makeatother
\end{tcolorbox}
\end{inctext}
\end{document}
\end{tcblisting}
\end{tcbverbatimwrite}
%
\begin{tcbraster}[raster column skip=0pt]
\begin{tcbverbatimwrite}{\jobname.tempinc}
\usepackage[int]{\didecpackageprefix didec}
\end{tcbverbatimwrite}
\input{\jobname.temp}
%
\begin{tcbverbatimwrite}{\jobname.tempinc}
\usepackage[fp]{\didecpackageprefix didec}
\end{tcbverbatimwrite}
\input{\jobname.temp}
\end{tcbraster}
%-------------------------------------------------------------------------------


\clearpage
If needed, the selected kernel can be questioned by the following:

%-------------------------------------------------------------------------------
\begin{docCommands}[
    % doc parameter = \marg{didec var},
  ]
{
   { doc name = c_didec_kernel_str },
}
The current kernel given as a lower case string: one of
\docValue*{int} or \docValue*{fp}.
\begin{dispExample}
\ExplSyntaxOn
\c_didec_kernel_str
\ExplSyntaxOff
\end{dispExample}
\end{docCommands}
%-------------------------------------------------------------------------------



%-------------------------------------------------------------------------------
\begin{docCommands}[
    % doc parameter = \marg{didec var},
  ]
{
   { doc name = didec_if_kernel_int_p: },
   { doc name = didec_if_kernel_int:T, doc parameter = ~\marg{true code} },
   { doc name = didec_if_kernel_int:TF, doc parameter = ~\marg{true code}\marg{false code} },
   { doc name = didec_if_kernel_fp_p: },
   { doc name = didec_if_kernel_fp:T, doc parameter = ~\marg{true code} },
   { doc name = didec_if_kernel_fp:TF, doc parameter = ~\marg{true code}\marg{false code} },
}
Conditionals which allow kernel-specific code to be used. The names follow
naturally from those of the kernels.
\begin{dispExample}
\ExplSyntaxOn
\didec_if_kernel_int:T { Integer~kernel~used. }\par
Floating~point~kernel
\didec_if_kernel_fp:TF { ~used. }{ ~not~used. }
\ExplSyntaxOff
\end{dispExample}
\end{docCommands}
%-------------------------------------------------------------------------------





%%%%%%%%%%%%%%%%%%%%%%%%%%%%%%%%%%%%%%%%%%%%%%%%%%%%%%%%%%%%%%%%%%%%%%%%%%%%%%%%
\clearpage
\section{Creating didec variables}%

%-------------------------------------------------------------------------------
\begin{docCommands}[
    doc parameter = \marg{didec var},
  ]
{
  { doc name = didecnew },
  { doc name = didec_new:n },
 }
Creates a new \meta{didec var} or raises an error if the name is already taken.
\refCom{didecnew} trims spaces while \refCom{didec_new:n} does not.
\begin{dispExample}
\didecnew{konto}
\didecset{konto}{99.75}
\didecuse{konto}
\end{dispExample}
\end{docCommands}
%-------------------------------------------------------------------------------


%\clearpage
%%%%%%%%%%%%%%%%%%%%%%%%%%%%%%%%%%%%%%%%%%%%%%%%%%%%%%%%%%%%%%%%%%%%%%%%%%%%%%%%
\section{Setting didec variables}%


%-------------------------------------------------------------------------------
\begin{docCommands}[
    doc parameter = \marg{didec var}\marg{didec expr},
  ]
{
  { doc name = didecset },
  { doc name = didec_gset:nn },
  { doc name = didec_gset_check:nn },
 }
Sets \meta{didec var} to the value of \meta{didec expr} which can be
\begin{itemize}
\item a number in floating-point notation,
\item a number in floating-comma notation,
\item another \meta{didec var}.
\end{itemize}
\refCom{didecset} trims spaces and performs an existence check for \meta{didec var}.
\refCom{didec_gset_check:nn} performs an existence check for \meta{didec var}.
Decimals places 3 and beyond are cut not rounded.
If rounding is an issue, use \refCom{didecsetfp} instead.
\begin{dispExample}
\didecset{A}{1234.56}
\didecuse{A}

\didecset{A}{2345,6789}
\didecuse{A}

\didecset{B}{-3500}
\didecset{A}{B}
\didecuse{A}
\end{dispExample}
\end{docCommands}
%-------------------------------------------------------------------------------



%-------------------------------------------------------------------------------
\begin{docCommands}[
    doc parameter = \marg{didec var\textsubscript{1}}\marg{didec var\textsubscript{2}},
  ]
{
  { doc name = didecsetequal },
  { doc name = didec_gset_eq:nn },
  { doc name = didec_gset_eq_check:nn },
 }
Sets the \meta{didec var\textsubscript{1}} to the current value of \meta{didec var\textsubscript{2}}
\begin{dispExample}
\didecset{A}{1234.56}
\didecsetequal{B}{A}
\didecuse{A}
\didecuse{B}
\end{dispExample}
\end{docCommands}
%-------------------------------------------------------------------------------


\clearpage
%-------------------------------------------------------------------------------
\begin{docCommands}[
    doc parameter = \marg{didec var}\marg{didec expr},
  ]
{
  { doc name = didecsetnegative },
  { doc name = didec_gset_negative:nn },
  { doc name = didec_gset_negative_check:nn },
 }
Sets \meta{didec var} to the negated (opposite) value of \meta{didec expr}.
\refCom{didecsetnegative} trims spaces and performs an existence check for \meta{didec var}.
\begin{dispExample}
\didecsetnegative{A}{1234.56}
\didecuse{A}

\didecsetnegative{A}{-42.55}
\didecuse{A}

\didecset{B}{-3500}
\didecsetnegative{A}{B}
\didecuse{A}
\end{dispExample}
\end{docCommands}
%-------------------------------------------------------------------------------



%-------------------------------------------------------------------------------
\begin{docCommands}[
    doc parameter = \marg{didec var}\marg{fp expr},
  ]
{
  { doc name = didecsetfp },
  { doc name = didec_gset_fp:nn },
  { doc name = didec_gset_fp_check:nn },
 }
Sets \meta{didec var} to the value of \meta{fp expr} which can
by any \LaTeX3 floating-point expression.
\refCom{didecsetfp} trims spaces and performs an existence check for \meta{didec var}.
Other didec variables can be used inside \meta{fp expr} if guarded with \refCom{didectofp}.
The result is rounded to 2 decimal places.
\refCom{didec_gset_fp_check:nn} performs an existence check for \meta{didec var}.
\begin{dispExample}
\didecsetfp{A}{2345.6789}
\didecuse{A}

\didecsetfp{A}{ ln( 12345678 ) }
\didecuse{A}

\didecset{A}{123456,78}
\didecsetfp{B}{ \didectofp{A} * 2.35 / 100 }
2.35\% of \didecuse{A} are \didecuse{B}.
\end{dispExample}
\end{docCommands}
%-------------------------------------------------------------------------------


\clearpage
%-------------------------------------------------------------------------------
\begin{docCommands}
{
  { doc name = didecadd,                doc parameter = \oarg{didec var}\marg{didec expr\textsubscript{1}}\marg{didec expr\textsubscript{2}} },
  { doc name = didec_gadd:nn,     doc parameter = \marg{didec var}\marg{didec expr} },
  { doc name = didec_gadd_check:nn,     doc parameter = \marg{didec var}\marg{didec expr} },
  { doc name = didec_gadd_to:nnn, doc parameter = \marg{didec var}\marg{didec expr\textsubscript{1}}\marg{didec expr\textsubscript{2}} },
  { doc name = didec_gadd_to_check:nnn, doc parameter = \marg{didec var}\marg{didec expr\textsubscript{1}}\marg{didec expr\textsubscript{2}} },
 }
\begin{itemize}
\item Adds the result of computing the \meta{didec expr} to the \meta{didec var}
\item or sets \meta{didec var} to the sum of \meta{didec expr\textsubscript{1}} and \meta{didec expr\textsubscript{2}}.
\item For \refCom{didecadd}, if the optional \meta{didec var} is not available,
  the sum is stored into \meta{didec expr\textsubscript{1}} which has to be a \meta{didec var}
  in this case.
\end{itemize}
\begin{dispExample}
\didecset{A}{123}
\didecset{B}{5,88}
\didecadd{A}{B}
\didecuse{A}

\didecadd[A]{B}{1000}
\didecuse{A}

\didecadd{A}{-2750}
\didecuse{A}
\end{dispExample}
\end{docCommands}
%-------------------------------------------------------------------------------


%\clearpage
%-------------------------------------------------------------------------------
\begin{docCommands}
{
  { doc name = didecsub,                doc parameter = \oarg{didec var}\marg{didec expr\textsubscript{1}}\marg{didec expr\textsubscript{2}} },
  { doc name = didec_gsub:nn,     doc parameter = \marg{didec var}\marg{didec expr} },
  { doc name = didec_gsub_check:nn,     doc parameter = \marg{didec var}\marg{didec expr} },
  { doc name = didec_gsub_to:nnn, doc parameter = \marg{didec var}\marg{didec expr\textsubscript{1}}\marg{didec expr\textsubscript{2}} },
  { doc name = didec_gsub_to_check:nnn, doc parameter = \marg{didec var}\marg{didec expr\textsubscript{1}}\marg{didec expr\textsubscript{2}} },
 }
\begin{itemize}
\item Subtracts the result of computing the \meta{didec expr} to the \meta{didec var}
\item or sets \meta{didec var} to the difference of \meta{didec expr\textsubscript{1}} and \meta{didec expr\textsubscript{2}}.
\item For \refCom{didecsub}, if the optional \meta{didec var} is not available,
  the difference is stored into \meta{didec expr\textsubscript{1}} which has to be a \meta{didec var}
  in this case.
\end{itemize}
\begin{dispExample}
\didecset{A}{123}
\didecset{B}{5,88}
\didecsub{A}{B}
\didecuse{A}

\didecsub[A]{B}{1000}
\didecuse{A}

\didecsub{A}{-2750}
\didecuse{A}
\end{dispExample}
\end{docCommands}
%-------------------------------------------------------------------------------


\clearpage
%-------------------------------------------------------------------------------
\begin{docCommands}
{
  { doc name = didecmulfp,                doc parameter = \oarg{didec var\textsubscript{2}}\marg{didec var}\marg{fp expr} },
  { doc name = didec_gmul_fp:nn,     doc parameter = \marg{didec var}\marg{fp expr} },
  { doc name = didec_gmul_fp_check:nn,     doc parameter = \marg{didec var}\marg{fp expr} },
  { doc name = didec_gmul_fp_to:nnn, doc parameter = \marg{didec var\textsubscript{2}}\marg{didec var}\marg{fp expr} },
  { doc name = didec_gmul_fp_to_check:nnn, doc parameter = \marg{didec var\textsubscript{2}}\marg{didec var}\marg{fp expr} },
 }
\begin{itemize}
\item Multiplies \meta{didec var} with the result of computing the \meta{fp expr}
\item and sets \meta{didec var} or respectively \meta{didec var\textsubscript{2}} to the result.
\end{itemize}
\begin{dispExample}
\didecset{A}{123}
\didecmulfp{A}{0.9675}
\didecuse{A}

\didecmulfp[B]{A}{ln(42)}
\didecuse{B}
\end{dispExample}
\end{docCommands}
%-------------------------------------------------------------------------------



%-------------------------------------------------------------------------------
\begin{docCommands}
{
  { doc name = didecdivfp,                doc parameter = \oarg{didec var\textsubscript{2}}\marg{didec var}\marg{fp expr} },
  { doc name = didec_gdiv_fp:nn,     doc parameter = \marg{didec var}\marg{fp expr} },
  { doc name = didec_gdiv_fp_check:nn,     doc parameter = \marg{didec var}\marg{fp expr} },
  { doc name = didec_gdiv_fp_to:nnn, doc parameter = \marg{didec var\textsubscript{2}}\marg{didec var}\marg{fp expr} },
  { doc name = didec_gdiv_fp_to_check:nnn, doc parameter = \marg{didec var\textsubscript{2}}\marg{didec var}\marg{fp expr} },
 }
\begin{itemize}
\item Divides \meta{didec var} by the result of computing the \meta{fp expr}
\item and sets \meta{didec var} or respectively \meta{didec var\textsubscript{2}} to the result.
\end{itemize}
\begin{dispExample}
\didecset{A}{123}
\didecdivfp{A}{0.9675}
\didecuse{A}

\didecdivfp[B]{A}{ln(42)}
\didecuse{B}
\end{dispExample}
\end{docCommands}
%-------------------------------------------------------------------------------


%-------------------------------------------------------------------------------
\begin{docCommands}
{
  { doc name = didecsetsum, doc parameter = \oarg{didec var}\marg{sum of didec exp} },
}
Sets \meta{didec var} to the result of computing the given \meta{sum of didec exp}.\par
Here, \meta{sum of didec exp} =  \meta{didec exp\textsubscript{1}}
+ \meta{didec exp\textsubscript{2}} + \ldots + \meta{didec exp\textsubscript{n}}
\begin{dispExample}
\didecset{A}{123}
\didecset{B}{-32.15}
\didecsetsum{A}{ A + B + -22.5 }
\didecuse{A}
\end{dispExample}
\end{docCommands}
%-------------------------------------------------------------------------------



\clearpage
%%%%%%%%%%%%%%%%%%%%%%%%%%%%%%%%%%%%%%%%%%%%%%%%%%%%%%%%%%%%%%%%%%%%%%%%%%%%%%%%
\section{Using didecs}%


%-------------------------------------------------------------------------------
\begin{docCommands}
{
  { doc name = didectoint, doc parameter = \marg{didec var} },
  { doc name = didec_to_int:n, doc parameter = \marg{didec var} },
  { doc name = didec_to_int_check:n, doc parameter = \marg{didec var} },
}
Expresses the \meta{didec var} as Cent integer value, i.e. 100 times the
value.
All functions are expandable.
\begin{dispExample}
\didecset{A}{27123.45}
\didectoint{A}

\didecset{A}{-17}
\didectoint{A}
\end{dispExample}
\end{docCommands}
%-------------------------------------------------------------------------------



%-------------------------------------------------------------------------------
\begin{docCommands}
{
  { doc name = didectofp, doc parameter = \marg{didec var} },
  { doc name = didec_to_fp:n, doc parameter = \marg{didec var} },
  { doc name = didec_to_fp_check:n, doc parameter = \marg{didec var} },
}
Expresses the \meta{didec var} as floating-point value.
All functions are expandable.
\begin{dispExample}
\didecset{A}{27123.45}
\didectofp{A}

\didecset{A}{-17}
\didectofp{A}
\end{dispExample}
\end{docCommands}
%-------------------------------------------------------------------------------



%-------------------------------------------------------------------------------
\begin{docCommands}
{
  { doc name = didectofc, doc parameter = \marg{didec var} },
  { doc name = didec_to_fc:n, doc parameter = \marg{didec var} },
  { doc name = didec_to_fc_check:n, doc parameter = \marg{didec var} },
}
Expresses the \meta{didec var} as floating-comma value.
All functions are expandable.
\begin{dispExample}
\didecset{A}{27123.45}
\didectofc{A}

\didecset{A}{-17}
\didectofc{A}
\end{dispExample}
\end{docCommands}
%-------------------------------------------------------------------------------


\clearpage
%-------------------------------------------------------------------------------
\begin{docCommands}
{
  { doc name = didecuse, doc parameter = \oarg{key list}\marg{didec var} },
  { doc name = didec_use:n, doc parameter = \marg{didec var} },
  { doc name = didec_use_check:n, doc parameter = \marg{didec var} },
}
Expresses the \meta{didec var} as formatted value.
With \refCom{didecsetup}, the standard format can be set.
This standard format can be overwritten by \meta{key list}.
\begin{dispExample}
\didecset{A}{123456.78}
\didecuse{A}

\didecuse[
    currency           = {\pounds}{},
    decimal-separator  = {.},
    grouping-separator = {,},
  ]{A}
\end{dispExample}
\end{docCommands}
%-------------------------------------------------------------------------------



%-------------------------------------------------------------------------------
\begin{docCommands}
{
  { doc name = dideccoluse, doc parameter = \oarg{key list}\marg{didec var} },
  { doc name = didec_color_use:n, doc parameter = \marg{didec var} },
  { doc name = didec_color_use_check:n, doc parameter = \marg{didec var} },
}
Expresses the \meta{didec var} as colorized formatted value.
With \refCom{didecsetup}, the standard format can be set.
This standard format can be overwritten by \meta{key list}.
\begin{dispExample}
\didecset{A}{123456.78}
\dideccoluse{A}

\didecset{A}{-125}
\dideccoluse{A}

\dideccoluse[color-negative=didec-blue]{A}
\end{dispExample}
\end{docCommands}
%-------------------------------------------------------------------------------


%-------------------------------------------------------------------------------
\begin{docCommands}
{
  { doc name = dideccolinvuse, doc parameter = \oarg{key list}\marg{didec var} },
  { doc name = didec_color_inverse_use:n, doc parameter = \marg{didec var} },
  { doc name = didec_color_inverse_use_check:n, doc parameter = \marg{didec var} },
}
Expresses the \meta{didec var} as colorized formatted value.
The coloring is switched between positive and negative.
The standard
coloring and format can be overwritten by \meta{key list}.
\begin{dispExample}
\didecset{A}{123456.78}
\dideccolinvuse{A}

\didecset{A}{-125}
\dideccolinvuse{A}
\end{dispExample}
\end{docCommands}
%-------------------------------------------------------------------------------


\clearpage
%-------------------------------------------------------------------------------
\begin{docCommands}
{
  { doc name = didecformat, doc parameter = \oarg{key list}\marg{didec expr} },
  { doc name = dideccolformat, doc parameter = \oarg{key list}\marg{didec expr} },
  { doc name = dideccolinvformat, doc parameter = \oarg{key list}\marg{didec expr} },
}
Like \refCom{didecuse}, \refCom{dideccoluse}, \refCom{dideccolinvuse},
but accepts a \meta{didec expr} instead of a \meta{didec var}.
If the \meta{didec expr} is a \meta{didec var},
\refCom{didecuse}, \refCom{dideccoluse}, \refCom{dideccolinvuse} are more efficient.
\begin{dispExample}
\didecformat{123456.78}\\
\didecformat{A}\\
\didecformat[english]{123456.78}\\
\dideccolformat{123456.78}\\
\dideccolinvformat{123456.78}
\end{dispExample}
\end{docCommands}
%-------------------------------------------------------------------------------


%-------------------------------------------------------------------------------
\begin{docCommands}
{
  { doc name = didecsetup, doc parameter = \marg{key list} },
}
Sets all keys of the given \meta{key list}. See the following documentation
for available settings.
\begin{dispExample}
\didecsetup{
    currency           = {\pounds}{},
    decimal-separator  = {.},
    grouping-separator = {,},
}
\didecset{A}{123456.78}
\didecuse{A}
\end{dispExample}
\end{docCommands}
%-------------------------------------------------------------------------------


\clearpage
%-------------------------------------------------------------------------------
\begin{docDidecKeys}
{
  {
    doc name = decimal-separator,
    doc parameter = {=\marg{separator}},
    doc description = {initially |,|},
  },
}
Sets some \meta{separator} as decimal separator.
\begin{dispExample}
\didecset{A}{123456.78}
\didecuse[ decimal-separator={\#} ]{A} \par
\end{dispExample}
\end{docDidecKeys}
%-------------------------------------------------------------------------------



%-------------------------------------------------------------------------------
\begin{docDidecKeys}
{
  {
    doc name = grouping-separator,
    doc parameter = {=\marg{separator}},
    doc description = {initially |.|},
  },
}
Sets some \meta{separator} as grouping separator.
\begin{dispExample}
\didecset{A}{123456.78}
\didecuse[ grouping-separator={'} ]{A}
\end{dispExample}
\end{docDidecKeys}
%-------------------------------------------------------------------------------



%-------------------------------------------------------------------------------
\begin{docDidecKeys}
{
  {
    doc name = currency,
    doc parameter = {=\marg{prefix}\marg{postfix}},
    doc description = {initially empty},
  },
}
Sets some \meta{prefix} and \meta{postfix} to denote the currency of the
didec variable.
This also sets\\
\refKey{didec/currency-negative}\brackets{\meta{prefix}-}\marg{postfix}
\begin{dispExample}
\didecset{A}{123456.78}
\didecuse[ currency = {\pounds}{} ]{A} \par
\didecuse[ currency = {}{\:Gulden} ]{A}
\end{dispExample}
\end{docDidecKeys}
%-------------------------------------------------------------------------------



%-------------------------------------------------------------------------------
\begin{docDidecKeys}
{
  {
    doc name = currency-negative,
    doc parameter = {=\marg{prefix}\marg{postfix}},
    doc description = {initially empty},
  },
}
Sets some \meta{prefix} and \meta{postfix} to denote the currency of the
didec variable, if the resulting value is negative. Otherwise, the settings of
\refKey{didec/currency} are used.
Note that you need to set a minus sign |-| explicitly, if you want to see it.
Also note that setting \refKey{didec/currency} overwrites values given
by \refKey{didec/currency-negative}.
\begin{dispExample}
\didecset{A}{-123456.78}
\didecuse{A}\par
\didecuse[ currency-negative = {$-$}{\;€} ]{A} \par
\didecuse[ currency-negative = {(}{\;€)} ]{A} \par
\didecuse[ currency-negative = {(}{)\;€} ]{A} \par
\didecuse[ currency-negative = {-€}{} ]{A} \par
\end{dispExample}
\end{docDidecKeys}
%-------------------------------------------------------------------------------


\clearpage
%-------------------------------------------------------------------------------
\begin{docDidecKeys}[
    doc parameter   = ,
    doc description = style,
]
{
  { doc name = german },
  { doc name = english },
  { doc name = french },
  { doc name = float },
}
Styles to set some format preferences combined.
Any currency settings are removed and should be applied afterwards,
if needed.
Note that \refCom{didectofp} is more efficient than \refCom{didecuse} with
style \refKey{didec/float}.
\begin{center}
\begin{tabular}{>{\ttfamily}l>{\ttfamily}c>{\ttfamily}c>{\ttfamily}c>{\ttfamily}c}\toprule
                   & german & english & french & float\\\midrule
decimal-separator  &   ,    &    .    &    ,   &   .  \\ 
grouping-separator &   .    &    ,    &   |\;|   &      \\\bottomrule
\end{tabular}
\end{center}
\begin{dispExample}
\didecset{A}{12345678.90}
\didecuse[german]{A}  \par
\didecuse[english]{A} \par
\didecuse[french]{A}  \par
\didecuse[float]{A}   \par
\end{dispExample}
\end{docDidecKeys}
%-------------------------------------------------------------------------------


%-------------------------------------------------------------------------------
\begin{docDidecKeys}
{
  {
    doc name = color-positive,
    doc parameter = {=\meta{positive color}},
    doc description = {initially |didec-green|},
  },
  {
    doc name = color-negative,
    doc parameter = {=\meta{negative color}},
    doc description = {initially |didec-red|},
  },
}
Sets \meta{positive color} to denote positive (and zero) values
and \meta{negative color} to denote negative values.
Any valid |l3color| \meta{color expression} can be used.
The package defines additional colors
\begin{itemize}
\item\docColorExample{didec-green}
\item\docColorExample{didec-red}
\item\docColorExample{didec-blue}
\end{itemize}
\begin{dispExample}
\didecset{A}{123456.78}
\dideccoluse[ color-positive = magenta ]{A}
\end{dispExample}
\end{docDidecKeys}
%-------------------------------------------------------------------------------




%-------------------------------------------------------------------------------
\begin{docCommands}
{
  { doc name = didecwrite, doc parameter = \marg{didec var}\marg{stream} },
  { doc name = didec_write:nn, doc parameter = \marg{didec var}\marg{stream} },
  { doc name = didec_write_check:nn, doc parameter = \marg{didec var}\marg{stream} },
}
Writes \refCom{didecset}\marg{didec var}\marg{current value} to the given
already opened output \meta{stream}.
\begin{dispListing}
\didecwrite{A}{output}
% writes to output:
% \didecset{A}{VALUE}
\end{dispListing}
\end{docCommands}
%-------------------------------------------------------------------------------


\clearpage
%%%%%%%%%%%%%%%%%%%%%%%%%%%%%%%%%%%%%%%%%%%%%%%%%%%%%%%%%%%%%%%%%%%%%%%%%%%%%%%%
\section{Didec conditionals}%
%-------------------------------------------------------------------------------
\begin{docCommands}[
    doc parameter = \marg{didec var}\marg{true code}\marg{false code},
  ]
{
  { doc name = didecifpositive },
  { doc name = didec_if_positive_p:n, doc parameter = \marg{didec var} },
  { doc name = didec_if_positive:nTF, },
  { doc name = didec_if_positive:nT, doc parameter = \marg{didec var}\marg{true code} },
  { doc name = didec_if_positive:nF, doc parameter = \marg{didec var}\marg{false code} },
 }
Evaluates the \meta{didec var}
and returns |true| or executes the \meta{true code} if the value is positive,
otherwise returns |false| or executes the \meta{false code}.
\begin{dispExample}
\didecset{A}{2799.50}
\didecuse{A} is \didecifpositive{A}{positive}{not positive}.

\didecset{B}{-584}
\didecuse{B} is \didecifpositive{B}{positive}{not positive}.

\didecset{A}{0}
\didecuse{A} is \didecifpositive{A}{positive}{not positive}.
\end{dispExample}
\end{docCommands}
%-------------------------------------------------------------------------------


%-------------------------------------------------------------------------------
\begin{docCommands}[
    doc parameter = \marg{didec var}\marg{true code}\marg{false code},
  ]
{
  { doc name = didecifnegative },
  { doc name = didec_if_negative_p:n, doc parameter = \marg{didec var} },
  { doc name = didec_if_negative:nTF, },
  { doc name = didec_if_negative:nT, doc parameter = \marg{didec var}\marg{true code} },
  { doc name = didec_if_negative:nF, doc parameter = \marg{didec var}\marg{false code} },
 }
Evaluates the \meta{didec var}
and returns |true| or executes the \meta{true code} if the value is negative,
otherwise returns |false| or executes the \meta{false code}.
\begin{dispExample}
\didecset{A}{2799.50}
\didecuse{A} is \didecifnegative{A}{negative}{not negative}.

\didecset{B}{-584}
\didecuse{B} is \didecifnegative{B}{negative}{not negative}.

\didecset{A}{0}
\didecuse{A} is \didecifnegative{A}{negative}{not negative}.
\end{dispExample}
\end{docCommands}
%-------------------------------------------------------------------------------


\clearpage
%-------------------------------------------------------------------------------
\begin{docCommands}[
    doc parameter = \marg{didec var}\marg{true code}\marg{false code},
  ]
{
  { doc name = didecifzero  },
  { doc name = didec_if_zero_p:n, doc parameter = \marg{didec var} },
  { doc name = didec_if_zero:nTF, },
  { doc name = didec_if_zero:nT, doc parameter = \marg{didec var}\marg{true code} },
  { doc name = didec_if_zero:nF, doc parameter = \marg{didec var}\marg{false code} },
 }
Evaluates the \meta{didec var}
and returns |true| or executes the \meta{true code} if the value is zero,
otherwise returns |false| or executes the \meta{false code}.
\begin{dispExample}
\didecset{A}{2799.50}
\didecuse{A} is \didecifzero{A}{zero}{not zero}.

\didecset{B}{-584}
\didecuse{B} is \didecifzero{B}{zero}{not zero}.

\didecset{A}{0}
\didecuse{A} is \didecifzero{A}{zero}{not zero}.
\end{dispExample}
\end{docCommands}
%-------------------------------------------------------------------------------






%-------------------------------------------------------------------------------
\begin{docCommands}[
    doc parameter = \marg{didec expr\textsubscript{1}}\marg{didec expr\textsubscript{2}}\marg{true code}\marg{false code},
  ]
{
  { doc name = dideciflowerthan },
  { doc name = didecifequal  },
  { doc name = didecifgreaterthan },
  { doc name = didec_compare_p:nNn, doc parameter = \marg{didec expr\textsubscript{1}}\meta{relation}\marg{didec expr\textsubscript{2}} },
  { doc name = didec_compare:nNnTF, doc parameter = \marg{didec expr\textsubscript{1}}\meta{relation}\marg{didec expr\textsubscript{2}}\marg{true code}\marg{false code}  },
  { doc name = didec_compare:nNnT, doc parameter = \marg{didec expr\textsubscript{1}}\meta{relation}\marg{didec expr\textsubscript{2}}\marg{true code} },
  { doc name = didec_compare:nNnF, doc parameter = \marg{didec expr\textsubscript{1}}\meta{relation}\marg{didec expr\textsubscript{2}}\marg{false code} },
 }
Compares the \meta{didec expr\textsubscript{1}} and the \meta{didec expr\textsubscript{2}},
and returns |true| or executes the \meta{true code} if the \meta{relation} (given by function respectively)
is obeyed, otherwise returns |false| or executes the \meta{false code}.
\begin{dispExample}
\didecset{A}{2799.50}
\didecset{B}{-584}
\didecuse{A} is \dideciflowerthan{A}{B}{lower}{not lower} than \didecuse{B}

\didecuse{A} is \didecifequal{A}{B}{equal}{not equal} to \didecuse{B}

\didecuse{A} is \didecifequal{A}{A}{equal}{not equal} to \didecuse{A}

\didecuse{A} is \didecifgreaterthan{A}{B}{greater}{not greater} than \didecuse{B}
\end{dispExample}
\end{docCommands}
%-------------------------------------------------------------------------------


\clearpage
%%%%%%%%%%%%%%%%%%%%%%%%%%%%%%%%%%%%%%%%%%%%%%%%%%%%%%%%%%%%%%%%%%%%%%%%%%%%%%%%
\section{Viewing didecs}%

%-------------------------------------------------------------------------------
\begin{docCommands}
{
  { doc name = didec_show:n, doc parameter = \marg{didec var} },
 }
Displays the content of \meta{didec var} in the terminal.
\begin{dispListing}
\didecset{A}{2799.50}
\ExplSyntaxOn
\didec_show:n{A}
\ExplSyntaxOff
\end{dispListing}
\end{docCommands}
%-------------------------------------------------------------------------------



%%%%%%%%%%%%%%%%%%%%%%%%%%%%%%%%%%%%%%%%%%%%%%%%%%%%%%%%%%%%%%%%%%%%%%%%%%%%%%%%
\clearpage
\phantomsection
\printindex

\end{document}

\usepackage{l3benchmark,siunitx}
\begin{document}
\begin{inctext}
\begin{tcolorbox}[blankest]
\makeatletter
\ExplSyntaxOn
\cs_new_protected:Npn \didec_benchmark:n #1
{
  \benchmark:n { #1 }
  \tablenum[table-format=4.0]{ \fp_to_decimal:n { round( \g_benchmark_time_fp * 1 000 000 ) } }\,\unit{\micro\second}
}
\didecnew{A}\didecnew{B}
\fp_set:Nn \g_benchmark_duration_target_fp { 1 }
\begin{center}
\begin{tabular}{lr}\toprule
\multicolumn{2}{c}{
  kernel:~\texttt{\c_didec_kernel_str},~engine:~\texttt{\c_sys_engine_str}
}\\\midrule
\docAuxCommand*{didec_gset:nn} & \didec_benchmark:n{ \didec_gset:nn{A}{-9876,324} }
\\\docAuxCommand*{didec_gset_check:nn} & \didec_benchmark:n{ \didec_gset_check:nn{A}{-9876,324} }
\\\docAuxCommand*{didecset} & \didec_benchmark:n{ \didecset{A}{-9876,324} }
\\\midrule
\docAuxCommand*{didec_gset_eq:nn} & \didecset{B}{1234.56} \didec_benchmark:n{ \didec_gset_eq:nn{A}{B} }
\\\docAuxCommand*{didec_gset_eq_check:nn} & \didecset{B}{1234.56} \didec_benchmark:n{ \didec_gset_eq_check:nn{A}{B} }
\\\docAuxCommand*{didecsetequal} & \didecset{B}{1234.56} \didec_benchmark:n{ \didecsetequal{A}{B} }
\\\midrule
\docAuxCommand*{didec_gset_fp:nn} & \didec_benchmark:n{ \didec_gset_fp:nn{A}{51234.56} }
\\\docAuxCommand*{didec_gset_fp_check:nn} & \didec_benchmark:n{ \didec_gset_fp_check:nn{A}{51234.56} }
\\\docAuxCommand*{didecsetfp} & \didec_benchmark:n{ \didecsetfp{A}{51234.56} }
\\\midrule
\docAuxCommand*{didec_gadd:nn} & \didecset{A}{-987.34} \didec_benchmark:n{ \didec_gadd:nn{A}{12345} }
\\\docAuxCommand*{didec_gadd_check:nn} &  \didecset{A}{-987.34}\didec_benchmark:n{ \didec_gadd_check:nn{A}{12345} }
\\\docAuxCommand*{didecadd} &  \didecset{A}{-987.34} \didec_benchmark:n{ \didecadd{A}{12345} }
\\\midrule
\docAuxCommand*{didec_gadd_to:nnn} & \didec_benchmark:n{ \didec_gadd_to:nnn{A}{-4564}{12345} }
\\\docAuxCommand*{didec_gadd_to_check:nnn} & \didec_benchmark:n{ \didec_gadd_to_check:nnn{A}{-4564}{12345} }
\\\docAuxCommand*{didecadd} & \didec_benchmark:n{ \didecadd[A]{-4564}{12345} }
\\\midrule
\docAuxCommand*{didec_gmul_fp:nn} & \didecset{A}{-987.34} \didec_benchmark:n{ \didec_gmul_fp:nn{A}{0.99} }
\\\docAuxCommand*{didec_gmul_fp_check:nn} &  \didecset{A}{-987.34}\didec_benchmark:n{ \didec_gmul_fp_check:nn{A}{0.99} }
\\\docAuxCommand*{didecmulfp} &  \didecset{A}{-987.34} \didec_benchmark:n{ \didecmulfp{A}{0.99} }
\\\midrule
%
\docAuxCommand*{didec_if_positive:nTF} & \didecset{A}{-987.34} \didec_benchmark:n{ \didec_if_positive:nTF{A}{}{} }
\\\docAuxCommand*{didecifpositive} & \didecset{A}{-987.34} \didec_benchmark:n{ \didecifpositive{A}{}{} }
\\\docAuxCommand*{didec_compare:nNnTF} & \didecset{A}{-987.34} \didec_benchmark:n{ \didec_compare:nNnTF{A}>{12345}{}{} }
\\\docAuxCommand*{didecifgreaterthan} & \didecset{A}{-987.34} \didec_benchmark:n{ \didecifgreaterthan{A}{12345}{}{} }
\\\midrule
\docAuxCommand*{didec_to_fp:n} & \didecset{A}{-1234987.34} \didec_benchmark:n{ \setbox\z@\hbox{\didec_to_fp:n{A}} }
\\\docAuxCommand*{didec_to_fp_check:n} & \didecset{A}{-1234987.34} \didec_benchmark:n{ \setbox\z@\hbox{\didec_to_fp_check:n{A}} }
\\\docAuxCommand*{didectofp} & \didecset{A}{-1234987.34} \didec_benchmark:n{ \setbox\z@\hbox{\didectofp{A}} }
\\\midrule
\docAuxCommand*{didec_use:n} & \didecset{A}{-1234987.34} \didec_benchmark:n{ \setbox\z@\hbox{\didec_use:n{A}} }
\\\docAuxCommand*{didec_use_check:n} & \didecset{A}{-1234987.34} \didec_benchmark:n{ \setbox\z@\hbox{\didec_use_check:n{A}} }
\\\docAuxCommand*{didecuse} & \didecset{A}{-1234987.34} \didec_benchmark:n{ \setbox\z@\hbox{\didecuse{A}} }
\\\docAuxCommand*{didecformat} & \didec_benchmark:n{ \setbox\z@\hbox{\didecformat{-1234987.34}} }
\\\docAuxCommand*{didec_color_use:n} & \didecset{A}{-1234987.34} \didec_benchmark:n{ \setbox\z@\hbox{\didec_color_use:n{A}} }
\\\docAuxCommand*{didec_color_use_check:n} & \didecset{A}{-1234987.34} \didec_benchmark:n{ \setbox\z@\hbox{\didec_color_use_check:n{A}} }
\\\docAuxCommand*{dideccoluse} & \didecset{A}{-1234987.34} \didec_benchmark:n{ \setbox\z@\hbox{\dideccoluse{A}} }
\\\docAuxCommand*{dideccolformat} & \didec_benchmark:n{ \setbox\z@\hbox{\dideccolformat{-1234987.34}} }
\\\bottomrule
\end{tabular}
\end{center}
\ExplSyntaxOff
\makeatother
\end{tcolorbox}
\end{inctext}
\end{document}
\end{tcblisting}
\end{tcbverbatimwrite}
%
\begin{tcbraster}[raster column skip=0pt]
\begin{tcbverbatimwrite}{\jobname.tempinc}
\usepackage[int]{\didecpackageprefix didec}
\end{tcbverbatimwrite}
\input{\jobname.temp}
%
\begin{tcbverbatimwrite}{\jobname.tempinc}
\usepackage[fp]{\didecpackageprefix didec}
\end{tcbverbatimwrite}
\input{\jobname.temp}
\end{tcbraster}
%-------------------------------------------------------------------------------


\clearpage
If needed, the selected kernel can be questioned by the following:

%-------------------------------------------------------------------------------
\begin{docCommands}[
    % doc parameter = \marg{didec var},
  ]
{
   { doc name = c_didec_kernel_str },
}
The current kernel given as a lower case string: one of
\docValue*{int} or \docValue*{fp}.
\begin{dispExample}
\ExplSyntaxOn
\c_didec_kernel_str
\ExplSyntaxOff
\end{dispExample}
\end{docCommands}
%-------------------------------------------------------------------------------



%-------------------------------------------------------------------------------
\begin{docCommands}[
    % doc parameter = \marg{didec var},
  ]
{
   { doc name = didec_if_kernel_int_p: },
   { doc name = didec_if_kernel_int:T, doc parameter = ~\marg{true code} },
   { doc name = didec_if_kernel_int:TF, doc parameter = ~\marg{true code}\marg{false code} },
   { doc name = didec_if_kernel_fp_p: },
   { doc name = didec_if_kernel_fp:T, doc parameter = ~\marg{true code} },
   { doc name = didec_if_kernel_fp:TF, doc parameter = ~\marg{true code}\marg{false code} },
}
Conditionals which allow kernel-specific code to be used. The names follow
naturally from those of the kernels.
\begin{dispExample}
\ExplSyntaxOn
\didec_if_kernel_int:T { Integer~kernel~used. }\par
Floating~point~kernel
\didec_if_kernel_fp:TF { ~used. }{ ~not~used. }
\ExplSyntaxOff
\end{dispExample}
\end{docCommands}
%-------------------------------------------------------------------------------





%%%%%%%%%%%%%%%%%%%%%%%%%%%%%%%%%%%%%%%%%%%%%%%%%%%%%%%%%%%%%%%%%%%%%%%%%%%%%%%%
\clearpage
\section{Creating didec variables}%

%-------------------------------------------------------------------------------
\begin{docCommands}[
    doc parameter = \marg{didec var},
  ]
{
  { doc name = didecnew },
  { doc name = didec_new:n },
 }
Creates a new \meta{didec var} or raises an error if the name is already taken.
\refCom{didecnew} trims spaces while \refCom{didec_new:n} does not.
\begin{dispExample}
\didecnew{konto}
\didecset{konto}{99.75}
\didecuse{konto}
\end{dispExample}
\end{docCommands}
%-------------------------------------------------------------------------------


%\clearpage
%%%%%%%%%%%%%%%%%%%%%%%%%%%%%%%%%%%%%%%%%%%%%%%%%%%%%%%%%%%%%%%%%%%%%%%%%%%%%%%%
\section{Setting didec variables}%


%-------------------------------------------------------------------------------
\begin{docCommands}[
    doc parameter = \marg{didec var}\marg{didec expr},
  ]
{
  { doc name = didecset },
  { doc name = didec_gset:nn },
  { doc name = didec_gset_check:nn },
 }
Sets \meta{didec var} to the value of \meta{didec expr} which can be
\begin{itemize}
\item a number in floating-point notation,
\item a number in floating-comma notation,
\item another \meta{didec var}.
\end{itemize}
\refCom{didecset} trims spaces and performs an existence check for \meta{didec var}.
\refCom{didec_gset_check:nn} performs an existence check for \meta{didec var}.
Decimals places 3 and beyond are cut not rounded.
If rounding is an issue, use \refCom{didecsetfp} instead.
\begin{dispExample}
\didecset{A}{1234.56}
\didecuse{A}

\didecset{A}{2345,6789}
\didecuse{A}

\didecset{B}{-3500}
\didecset{A}{B}
\didecuse{A}
\end{dispExample}
\end{docCommands}
%-------------------------------------------------------------------------------



%-------------------------------------------------------------------------------
\begin{docCommands}[
    doc parameter = \marg{didec var\textsubscript{1}}\marg{didec var\textsubscript{2}},
  ]
{
  { doc name = didecsetequal },
  { doc name = didec_gset_eq:nn },
  { doc name = didec_gset_eq_check:nn },
 }
Sets the \meta{didec var\textsubscript{1}} to the current value of \meta{didec var\textsubscript{2}}
\begin{dispExample}
\didecset{A}{1234.56}
\didecsetequal{B}{A}
\didecuse{A}
\didecuse{B}
\end{dispExample}
\end{docCommands}
%-------------------------------------------------------------------------------


\clearpage
%-------------------------------------------------------------------------------
\begin{docCommands}[
    doc parameter = \marg{didec var}\marg{didec expr},
  ]
{
  { doc name = didecsetnegative },
  { doc name = didec_gset_negative:nn },
  { doc name = didec_gset_negative_check:nn },
 }
Sets \meta{didec var} to the negated (opposite) value of \meta{didec expr}.
\refCom{didecsetnegative} trims spaces and performs an existence check for \meta{didec var}.
\begin{dispExample}
\didecsetnegative{A}{1234.56}
\didecuse{A}

\didecsetnegative{A}{-42.55}
\didecuse{A}

\didecset{B}{-3500}
\didecsetnegative{A}{B}
\didecuse{A}
\end{dispExample}
\end{docCommands}
%-------------------------------------------------------------------------------



%-------------------------------------------------------------------------------
\begin{docCommands}[
    doc parameter = \marg{didec var}\marg{fp expr},
  ]
{
  { doc name = didecsetfp },
  { doc name = didec_gset_fp:nn },
  { doc name = didec_gset_fp_check:nn },
 }
Sets \meta{didec var} to the value of \meta{fp expr} which can
by any \LaTeX3 floating-point expression.
\refCom{didecsetfp} trims spaces and performs an existence check for \meta{didec var}.
Other didec variables can be used inside \meta{fp expr} if guarded with \refCom{didectofp}.
The result is rounded to 2 decimal places.
\refCom{didec_gset_fp_check:nn} performs an existence check for \meta{didec var}.
\begin{dispExample}
\didecsetfp{A}{2345.6789}
\didecuse{A}

\didecsetfp{A}{ ln( 12345678 ) }
\didecuse{A}

\didecset{A}{123456,78}
\didecsetfp{B}{ \didectofp{A} * 2.35 / 100 }
2.35\% of \didecuse{A} are \didecuse{B}.
\end{dispExample}
\end{docCommands}
%-------------------------------------------------------------------------------


\clearpage
%-------------------------------------------------------------------------------
\begin{docCommands}
{
  { doc name = didecadd,                doc parameter = \oarg{didec var}\marg{didec expr\textsubscript{1}}\marg{didec expr\textsubscript{2}} },
  { doc name = didec_gadd:nn,     doc parameter = \marg{didec var}\marg{didec expr} },
  { doc name = didec_gadd_check:nn,     doc parameter = \marg{didec var}\marg{didec expr} },
  { doc name = didec_gadd_to:nnn, doc parameter = \marg{didec var}\marg{didec expr\textsubscript{1}}\marg{didec expr\textsubscript{2}} },
  { doc name = didec_gadd_to_check:nnn, doc parameter = \marg{didec var}\marg{didec expr\textsubscript{1}}\marg{didec expr\textsubscript{2}} },
 }
\begin{itemize}
\item Adds the result of computing the \meta{didec expr} to the \meta{didec var}
\item or sets \meta{didec var} to the sum of \meta{didec expr\textsubscript{1}} and \meta{didec expr\textsubscript{2}}.
\item For \refCom{didecadd}, if the optional \meta{didec var} is not available,
  the sum is stored into \meta{didec expr\textsubscript{1}} which has to be a \meta{didec var}
  in this case.
\end{itemize}
\begin{dispExample}
\didecset{A}{123}
\didecset{B}{5,88}
\didecadd{A}{B}
\didecuse{A}

\didecadd[A]{B}{1000}
\didecuse{A}

\didecadd{A}{-2750}
\didecuse{A}
\end{dispExample}
\end{docCommands}
%-------------------------------------------------------------------------------


%\clearpage
%-------------------------------------------------------------------------------
\begin{docCommands}
{
  { doc name = didecsub,                doc parameter = \oarg{didec var}\marg{didec expr\textsubscript{1}}\marg{didec expr\textsubscript{2}} },
  { doc name = didec_gsub:nn,     doc parameter = \marg{didec var}\marg{didec expr} },
  { doc name = didec_gsub_check:nn,     doc parameter = \marg{didec var}\marg{didec expr} },
  { doc name = didec_gsub_to:nnn, doc parameter = \marg{didec var}\marg{didec expr\textsubscript{1}}\marg{didec expr\textsubscript{2}} },
  { doc name = didec_gsub_to_check:nnn, doc parameter = \marg{didec var}\marg{didec expr\textsubscript{1}}\marg{didec expr\textsubscript{2}} },
 }
\begin{itemize}
\item Subtracts the result of computing the \meta{didec expr} to the \meta{didec var}
\item or sets \meta{didec var} to the difference of \meta{didec expr\textsubscript{1}} and \meta{didec expr\textsubscript{2}}.
\item For \refCom{didecsub}, if the optional \meta{didec var} is not available,
  the difference is stored into \meta{didec expr\textsubscript{1}} which has to be a \meta{didec var}
  in this case.
\end{itemize}
\begin{dispExample}
\didecset{A}{123}
\didecset{B}{5,88}
\didecsub{A}{B}
\didecuse{A}

\didecsub[A]{B}{1000}
\didecuse{A}

\didecsub{A}{-2750}
\didecuse{A}
\end{dispExample}
\end{docCommands}
%-------------------------------------------------------------------------------


\clearpage
%-------------------------------------------------------------------------------
\begin{docCommands}
{
  { doc name = didecmulfp,                doc parameter = \oarg{didec var\textsubscript{2}}\marg{didec var}\marg{fp expr} },
  { doc name = didec_gmul_fp:nn,     doc parameter = \marg{didec var}\marg{fp expr} },
  { doc name = didec_gmul_fp_check:nn,     doc parameter = \marg{didec var}\marg{fp expr} },
  { doc name = didec_gmul_fp_to:nnn, doc parameter = \marg{didec var\textsubscript{2}}\marg{didec var}\marg{fp expr} },
  { doc name = didec_gmul_fp_to_check:nnn, doc parameter = \marg{didec var\textsubscript{2}}\marg{didec var}\marg{fp expr} },
 }
\begin{itemize}
\item Multiplies \meta{didec var} with the result of computing the \meta{fp expr}
\item and sets \meta{didec var} or respectively \meta{didec var\textsubscript{2}} to the result.
\end{itemize}
\begin{dispExample}
\didecset{A}{123}
\didecmulfp{A}{0.9675}
\didecuse{A}

\didecmulfp[B]{A}{ln(42)}
\didecuse{B}
\end{dispExample}
\end{docCommands}
%-------------------------------------------------------------------------------



%-------------------------------------------------------------------------------
\begin{docCommands}
{
  { doc name = didecdivfp,                doc parameter = \oarg{didec var\textsubscript{2}}\marg{didec var}\marg{fp expr} },
  { doc name = didec_gdiv_fp:nn,     doc parameter = \marg{didec var}\marg{fp expr} },
  { doc name = didec_gdiv_fp_check:nn,     doc parameter = \marg{didec var}\marg{fp expr} },
  { doc name = didec_gdiv_fp_to:nnn, doc parameter = \marg{didec var\textsubscript{2}}\marg{didec var}\marg{fp expr} },
  { doc name = didec_gdiv_fp_to_check:nnn, doc parameter = \marg{didec var\textsubscript{2}}\marg{didec var}\marg{fp expr} },
 }
\begin{itemize}
\item Divides \meta{didec var} by the result of computing the \meta{fp expr}
\item and sets \meta{didec var} or respectively \meta{didec var\textsubscript{2}} to the result.
\end{itemize}
\begin{dispExample}
\didecset{A}{123}
\didecdivfp{A}{0.9675}
\didecuse{A}

\didecdivfp[B]{A}{ln(42)}
\didecuse{B}
\end{dispExample}
\end{docCommands}
%-------------------------------------------------------------------------------


%-------------------------------------------------------------------------------
\begin{docCommands}
{
  { doc name = didecsetsum, doc parameter = \oarg{didec var}\marg{sum of didec exp} },
}
Sets \meta{didec var} to the result of computing the given \meta{sum of didec exp}.\par
Here, \meta{sum of didec exp} =  \meta{didec exp\textsubscript{1}}
+ \meta{didec exp\textsubscript{2}} + \ldots + \meta{didec exp\textsubscript{n}}
\begin{dispExample}
\didecset{A}{123}
\didecset{B}{-32.15}
\didecsetsum{A}{ A + B + -22.5 }
\didecuse{A}
\end{dispExample}
\end{docCommands}
%-------------------------------------------------------------------------------



\clearpage
%%%%%%%%%%%%%%%%%%%%%%%%%%%%%%%%%%%%%%%%%%%%%%%%%%%%%%%%%%%%%%%%%%%%%%%%%%%%%%%%
\section{Using didecs}%


%-------------------------------------------------------------------------------
\begin{docCommands}
{
  { doc name = didectoint, doc parameter = \marg{didec var} },
  { doc name = didec_to_int:n, doc parameter = \marg{didec var} },
  { doc name = didec_to_int_check:n, doc parameter = \marg{didec var} },
}
Expresses the \meta{didec var} as Cent integer value, i.e. 100 times the
value.
All functions are expandable.
\begin{dispExample}
\didecset{A}{27123.45}
\didectoint{A}

\didecset{A}{-17}
\didectoint{A}
\end{dispExample}
\end{docCommands}
%-------------------------------------------------------------------------------



%-------------------------------------------------------------------------------
\begin{docCommands}
{
  { doc name = didectofp, doc parameter = \marg{didec var} },
  { doc name = didec_to_fp:n, doc parameter = \marg{didec var} },
  { doc name = didec_to_fp_check:n, doc parameter = \marg{didec var} },
}
Expresses the \meta{didec var} as floating-point value.
All functions are expandable.
\begin{dispExample}
\didecset{A}{27123.45}
\didectofp{A}

\didecset{A}{-17}
\didectofp{A}
\end{dispExample}
\end{docCommands}
%-------------------------------------------------------------------------------



%-------------------------------------------------------------------------------
\begin{docCommands}
{
  { doc name = didectofc, doc parameter = \marg{didec var} },
  { doc name = didec_to_fc:n, doc parameter = \marg{didec var} },
  { doc name = didec_to_fc_check:n, doc parameter = \marg{didec var} },
}
Expresses the \meta{didec var} as floating-comma value.
All functions are expandable.
\begin{dispExample}
\didecset{A}{27123.45}
\didectofc{A}

\didecset{A}{-17}
\didectofc{A}
\end{dispExample}
\end{docCommands}
%-------------------------------------------------------------------------------


\clearpage
%-------------------------------------------------------------------------------
\begin{docCommands}
{
  { doc name = didecuse, doc parameter = \oarg{key list}\marg{didec var} },
  { doc name = didec_use:n, doc parameter = \marg{didec var} },
  { doc name = didec_use_check:n, doc parameter = \marg{didec var} },
}
Expresses the \meta{didec var} as formatted value.
With \refCom{didecsetup}, the standard format can be set.
This standard format can be overwritten by \meta{key list}.
\begin{dispExample}
\didecset{A}{123456.78}
\didecuse{A}

\didecuse[
    currency           = {\pounds}{},
    decimal-separator  = {.},
    grouping-separator = {,},
  ]{A}
\end{dispExample}
\end{docCommands}
%-------------------------------------------------------------------------------



%-------------------------------------------------------------------------------
\begin{docCommands}
{
  { doc name = dideccoluse, doc parameter = \oarg{key list}\marg{didec var} },
  { doc name = didec_color_use:n, doc parameter = \marg{didec var} },
  { doc name = didec_color_use_check:n, doc parameter = \marg{didec var} },
}
Expresses the \meta{didec var} as colorized formatted value.
With \refCom{didecsetup}, the standard format can be set.
This standard format can be overwritten by \meta{key list}.
\begin{dispExample}
\didecset{A}{123456.78}
\dideccoluse{A}

\didecset{A}{-125}
\dideccoluse{A}

\dideccoluse[color-negative=didec-blue]{A}
\end{dispExample}
\end{docCommands}
%-------------------------------------------------------------------------------


%-------------------------------------------------------------------------------
\begin{docCommands}
{
  { doc name = dideccolinvuse, doc parameter = \oarg{key list}\marg{didec var} },
  { doc name = didec_color_inverse_use:n, doc parameter = \marg{didec var} },
  { doc name = didec_color_inverse_use_check:n, doc parameter = \marg{didec var} },
}
Expresses the \meta{didec var} as colorized formatted value.
The coloring is switched between positive and negative.
The standard
coloring and format can be overwritten by \meta{key list}.
\begin{dispExample}
\didecset{A}{123456.78}
\dideccolinvuse{A}

\didecset{A}{-125}
\dideccolinvuse{A}
\end{dispExample}
\end{docCommands}
%-------------------------------------------------------------------------------


\clearpage
%-------------------------------------------------------------------------------
\begin{docCommands}
{
  { doc name = didecformat, doc parameter = \oarg{key list}\marg{didec expr} },
  { doc name = dideccolformat, doc parameter = \oarg{key list}\marg{didec expr} },
  { doc name = dideccolinvformat, doc parameter = \oarg{key list}\marg{didec expr} },
}
Like \refCom{didecuse}, \refCom{dideccoluse}, \refCom{dideccolinvuse},
but accepts a \meta{didec expr} instead of a \meta{didec var}.
If the \meta{didec expr} is a \meta{didec var},
\refCom{didecuse}, \refCom{dideccoluse}, \refCom{dideccolinvuse} are more efficient.
\begin{dispExample}
\didecformat{123456.78}\\
\didecformat{A}\\
\didecformat[english]{123456.78}\\
\dideccolformat{123456.78}\\
\dideccolinvformat{123456.78}
\end{dispExample}
\end{docCommands}
%-------------------------------------------------------------------------------


%-------------------------------------------------------------------------------
\begin{docCommands}
{
  { doc name = didecsetup, doc parameter = \marg{key list} },
}
Sets all keys of the given \meta{key list}. See the following documentation
for available settings.
\begin{dispExample}
\didecsetup{
    currency           = {\pounds}{},
    decimal-separator  = {.},
    grouping-separator = {,},
}
\didecset{A}{123456.78}
\didecuse{A}
\end{dispExample}
\end{docCommands}
%-------------------------------------------------------------------------------


\clearpage
%-------------------------------------------------------------------------------
\begin{docDidecKeys}
{
  {
    doc name = decimal-separator,
    doc parameter = {=\marg{separator}},
    doc description = {initially |,|},
  },
}
Sets some \meta{separator} as decimal separator.
\begin{dispExample}
\didecset{A}{123456.78}
\didecuse[ decimal-separator={\#} ]{A} \par
\end{dispExample}
\end{docDidecKeys}
%-------------------------------------------------------------------------------



%-------------------------------------------------------------------------------
\begin{docDidecKeys}
{
  {
    doc name = grouping-separator,
    doc parameter = {=\marg{separator}},
    doc description = {initially |.|},
  },
}
Sets some \meta{separator} as grouping separator.
\begin{dispExample}
\didecset{A}{123456.78}
\didecuse[ grouping-separator={'} ]{A}
\end{dispExample}
\end{docDidecKeys}
%-------------------------------------------------------------------------------



%-------------------------------------------------------------------------------
\begin{docDidecKeys}
{
  {
    doc name = currency,
    doc parameter = {=\marg{prefix}\marg{postfix}},
    doc description = {initially empty},
  },
}
Sets some \meta{prefix} and \meta{postfix} to denote the currency of the
didec variable.
This also sets\\
\refKey{didec/currency-negative}\brackets{\meta{prefix}-}\marg{postfix}
\begin{dispExample}
\didecset{A}{123456.78}
\didecuse[ currency = {\pounds}{} ]{A} \par
\didecuse[ currency = {}{\:Gulden} ]{A}
\end{dispExample}
\end{docDidecKeys}
%-------------------------------------------------------------------------------



%-------------------------------------------------------------------------------
\begin{docDidecKeys}
{
  {
    doc name = currency-negative,
    doc parameter = {=\marg{prefix}\marg{postfix}},
    doc description = {initially empty},
  },
}
Sets some \meta{prefix} and \meta{postfix} to denote the currency of the
didec variable, if the resulting value is negative. Otherwise, the settings of
\refKey{didec/currency} are used.
Note that you need to set a minus sign |-| explicitly, if you want to see it.
Also note that setting \refKey{didec/currency} overwrites values given
by \refKey{didec/currency-negative}.
\begin{dispExample}
\didecset{A}{-123456.78}
\didecuse{A}\par
\didecuse[ currency-negative = {$-$}{\;€} ]{A} \par
\didecuse[ currency-negative = {(}{\;€)} ]{A} \par
\didecuse[ currency-negative = {(}{)\;€} ]{A} \par
\didecuse[ currency-negative = {-€}{} ]{A} \par
\end{dispExample}
\end{docDidecKeys}
%-------------------------------------------------------------------------------


\clearpage
%-------------------------------------------------------------------------------
\begin{docDidecKeys}[
    doc parameter   = ,
    doc description = style,
]
{
  { doc name = german },
  { doc name = english },
  { doc name = french },
  { doc name = float },
}
Styles to set some format preferences combined.
Any currency settings are removed and should be applied afterwards,
if needed.
Note that \refCom{didectofp} is more efficient than \refCom{didecuse} with
style \refKey{didec/float}.
\begin{center}
\begin{tabular}{>{\ttfamily}l>{\ttfamily}c>{\ttfamily}c>{\ttfamily}c>{\ttfamily}c}\toprule
                   & german & english & french & float\\\midrule
decimal-separator  &   ,    &    .    &    ,   &   .  \\ 
grouping-separator &   .    &    ,    &   |\;|   &      \\\bottomrule
\end{tabular}
\end{center}
\begin{dispExample}
\didecset{A}{12345678.90}
\didecuse[german]{A}  \par
\didecuse[english]{A} \par
\didecuse[french]{A}  \par
\didecuse[float]{A}   \par
\end{dispExample}
\end{docDidecKeys}
%-------------------------------------------------------------------------------


%-------------------------------------------------------------------------------
\begin{docDidecKeys}
{
  {
    doc name = color-positive,
    doc parameter = {=\meta{positive color}},
    doc description = {initially |didec-green|},
  },
  {
    doc name = color-negative,
    doc parameter = {=\meta{negative color}},
    doc description = {initially |didec-red|},
  },
}
Sets \meta{positive color} to denote positive (and zero) values
and \meta{negative color} to denote negative values.
Any valid |l3color| \meta{color expression} can be used.
The package defines additional colors
\begin{itemize}
\item\docColorExample{didec-green}
\item\docColorExample{didec-red}
\item\docColorExample{didec-blue}
\end{itemize}
\begin{dispExample}
\didecset{A}{123456.78}
\dideccoluse[ color-positive = magenta ]{A}
\end{dispExample}
\end{docDidecKeys}
%-------------------------------------------------------------------------------




%-------------------------------------------------------------------------------
\begin{docCommands}
{
  { doc name = didecwrite, doc parameter = \marg{didec var}\marg{stream} },
  { doc name = didec_write:nn, doc parameter = \marg{didec var}\marg{stream} },
  { doc name = didec_write_check:nn, doc parameter = \marg{didec var}\marg{stream} },
}
Writes \refCom{didecset}\marg{didec var}\marg{current value} to the given
already opened output \meta{stream}.
\begin{dispListing}
\didecwrite{A}{output}
% writes to output:
% \didecset{A}{VALUE}
\end{dispListing}
\end{docCommands}
%-------------------------------------------------------------------------------


\clearpage
%%%%%%%%%%%%%%%%%%%%%%%%%%%%%%%%%%%%%%%%%%%%%%%%%%%%%%%%%%%%%%%%%%%%%%%%%%%%%%%%
\section{Didec conditionals}%
%-------------------------------------------------------------------------------
\begin{docCommands}[
    doc parameter = \marg{didec var}\marg{true code}\marg{false code},
  ]
{
  { doc name = didecifpositive },
  { doc name = didec_if_positive_p:n, doc parameter = \marg{didec var} },
  { doc name = didec_if_positive:nTF, },
  { doc name = didec_if_positive:nT, doc parameter = \marg{didec var}\marg{true code} },
  { doc name = didec_if_positive:nF, doc parameter = \marg{didec var}\marg{false code} },
 }
Evaluates the \meta{didec var}
and returns |true| or executes the \meta{true code} if the value is positive,
otherwise returns |false| or executes the \meta{false code}.
\begin{dispExample}
\didecset{A}{2799.50}
\didecuse{A} is \didecifpositive{A}{positive}{not positive}.

\didecset{B}{-584}
\didecuse{B} is \didecifpositive{B}{positive}{not positive}.

\didecset{A}{0}
\didecuse{A} is \didecifpositive{A}{positive}{not positive}.
\end{dispExample}
\end{docCommands}
%-------------------------------------------------------------------------------


%-------------------------------------------------------------------------------
\begin{docCommands}[
    doc parameter = \marg{didec var}\marg{true code}\marg{false code},
  ]
{
  { doc name = didecifnegative },
  { doc name = didec_if_negative_p:n, doc parameter = \marg{didec var} },
  { doc name = didec_if_negative:nTF, },
  { doc name = didec_if_negative:nT, doc parameter = \marg{didec var}\marg{true code} },
  { doc name = didec_if_negative:nF, doc parameter = \marg{didec var}\marg{false code} },
 }
Evaluates the \meta{didec var}
and returns |true| or executes the \meta{true code} if the value is negative,
otherwise returns |false| or executes the \meta{false code}.
\begin{dispExample}
\didecset{A}{2799.50}
\didecuse{A} is \didecifnegative{A}{negative}{not negative}.

\didecset{B}{-584}
\didecuse{B} is \didecifnegative{B}{negative}{not negative}.

\didecset{A}{0}
\didecuse{A} is \didecifnegative{A}{negative}{not negative}.
\end{dispExample}
\end{docCommands}
%-------------------------------------------------------------------------------


\clearpage
%-------------------------------------------------------------------------------
\begin{docCommands}[
    doc parameter = \marg{didec var}\marg{true code}\marg{false code},
  ]
{
  { doc name = didecifzero  },
  { doc name = didec_if_zero_p:n, doc parameter = \marg{didec var} },
  { doc name = didec_if_zero:nTF, },
  { doc name = didec_if_zero:nT, doc parameter = \marg{didec var}\marg{true code} },
  { doc name = didec_if_zero:nF, doc parameter = \marg{didec var}\marg{false code} },
 }
Evaluates the \meta{didec var}
and returns |true| or executes the \meta{true code} if the value is zero,
otherwise returns |false| or executes the \meta{false code}.
\begin{dispExample}
\didecset{A}{2799.50}
\didecuse{A} is \didecifzero{A}{zero}{not zero}.

\didecset{B}{-584}
\didecuse{B} is \didecifzero{B}{zero}{not zero}.

\didecset{A}{0}
\didecuse{A} is \didecifzero{A}{zero}{not zero}.
\end{dispExample}
\end{docCommands}
%-------------------------------------------------------------------------------






%-------------------------------------------------------------------------------
\begin{docCommands}[
    doc parameter = \marg{didec expr\textsubscript{1}}\marg{didec expr\textsubscript{2}}\marg{true code}\marg{false code},
  ]
{
  { doc name = dideciflowerthan },
  { doc name = didecifequal  },
  { doc name = didecifgreaterthan },
  { doc name = didec_compare_p:nNn, doc parameter = \marg{didec expr\textsubscript{1}}\meta{relation}\marg{didec expr\textsubscript{2}} },
  { doc name = didec_compare:nNnTF, doc parameter = \marg{didec expr\textsubscript{1}}\meta{relation}\marg{didec expr\textsubscript{2}}\marg{true code}\marg{false code}  },
  { doc name = didec_compare:nNnT, doc parameter = \marg{didec expr\textsubscript{1}}\meta{relation}\marg{didec expr\textsubscript{2}}\marg{true code} },
  { doc name = didec_compare:nNnF, doc parameter = \marg{didec expr\textsubscript{1}}\meta{relation}\marg{didec expr\textsubscript{2}}\marg{false code} },
 }
Compares the \meta{didec expr\textsubscript{1}} and the \meta{didec expr\textsubscript{2}},
and returns |true| or executes the \meta{true code} if the \meta{relation} (given by function respectively)
is obeyed, otherwise returns |false| or executes the \meta{false code}.
\begin{dispExample}
\didecset{A}{2799.50}
\didecset{B}{-584}
\didecuse{A} is \dideciflowerthan{A}{B}{lower}{not lower} than \didecuse{B}

\didecuse{A} is \didecifequal{A}{B}{equal}{not equal} to \didecuse{B}

\didecuse{A} is \didecifequal{A}{A}{equal}{not equal} to \didecuse{A}

\didecuse{A} is \didecifgreaterthan{A}{B}{greater}{not greater} than \didecuse{B}
\end{dispExample}
\end{docCommands}
%-------------------------------------------------------------------------------


\clearpage
%%%%%%%%%%%%%%%%%%%%%%%%%%%%%%%%%%%%%%%%%%%%%%%%%%%%%%%%%%%%%%%%%%%%%%%%%%%%%%%%
\section{Viewing didecs}%

%-------------------------------------------------------------------------------
\begin{docCommands}
{
  { doc name = didec_show:n, doc parameter = \marg{didec var} },
 }
Displays the content of \meta{didec var} in the terminal.
\begin{dispListing}
\didecset{A}{2799.50}
\ExplSyntaxOn
\didec_show:n{A}
\ExplSyntaxOff
\end{dispListing}
\end{docCommands}
%-------------------------------------------------------------------------------



%%%%%%%%%%%%%%%%%%%%%%%%%%%%%%%%%%%%%%%%%%%%%%%%%%%%%%%%%%%%%%%%%%%%%%%%%%%%%%%%
\clearpage
\phantomsection
\printindex

\end{document}
